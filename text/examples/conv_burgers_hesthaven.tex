\begin{example}[Viscous Burgers 1D]
\label{ex:burgers_hest}
Based on \cite[Section 7.1.2, Example 7.5,  p. 255]{Hesthaven2008}.
On $\Omega = [-1, 1]$ we will solve the viscous Burgers’ equation
\eqref{eq:ex_burgers} with the zero source function.
%\begin{equation}
%	\pdiff{u}{t} + \pdiff{}{x}\left(\frac{u^2}{2}\right) = D \pdiff{^2u}{x^2}
%\end{equation}
%where $D$ is diffusion coefficient.
This equation has an exact solution of a traveling wave
\begin{equation}
	u_{exact} =  -\tanh\left(-\frac{2 \, t - 2 \, x - 1}{4 \,D}\right) + 1.
\end{equation}
We set boundary conditions to match the solution
\begin{equation}
	\begin{aligned}
	& u(-1, t) = -\tanh\left(-\frac{2 \, t  + 1}{4 \,D}\right) + 1,
	&  u(-1, t) = -\tanh\left(-\frac{2 \, t - 3}{4 \,D}\right) + 1,\\
	&u_x(-1, t) = \frac{1}{2D}\tanh\left(-\frac{2 \, t + 1}{4 \, D}\right)^{2} -
	\frac{1}{2D},
	&u_x(1, t) = \frac{1}{2D}\tanh\left(-\frac{2 \, t - 3}{4 \, D}\right)^{2} -
	\frac{1}{2D}.
	\end{aligned}
\end{equation}
We will study the solution at time $t = 1$ with $D = 0.001$ and $D = 0.01$. The exact
solution is shown in \Cref{fig:burgers_hesthaven_ext}.
\begin{figure}[h]
	\centering
%	\begin{tabular}{p{0.5\textwidth} p{0.5\textwidth}}
%		\vspace{0pt}
%		\includegraphics[width=0.4\textwidth]{../figs/burgess_hesthaven_exact_t1_e001.png}
%		&
%		\vspace{0pt}
%		\includegraphics[width=0.4\textwidth]{../figs/burgess_hesthaven_exact_t1_e01.png}
%	\end{tabular}
%
	\includegraphics[scale=0.45]{../figs/burgers_hesthaven_exact_t1.png}
	\caption{\Cref{ex:burgers_hest}. Exact solution at $t = 1$.}
	\label{fig:burgers_hesthaven_ext}
\end{figure}
Different values of the coefficient $C_w$ in the penalty term yield different convergence
behavior as demonstrated in \Cref{fig:burgers_conv}. In this case increasing $C_w$ is detrimental to the accuracy of the solution for $D=0.001$ as it
develops into a steep step, see \Cref{fig:burgers_hesthaven_ext}, whose approximation
requires discontinuity in the approximate solution. For $D=0.01$ the diffusion leads to a much
more gradual solution and the penalty term counteracting discontinuity between elements
is beneficial, it also helps to counteract oscillations similarly to the limiter in
\Cref{ex:adv1D}. \todo Figure \ref{fig:burgers_sol} demonstrates where the errors come 
from.

\begin{figure}[h!]
    \centering
    \begin{subfigure}{.5\textwidth}
        \centering
        \includegraphics[width=\linewidth]{../figs/sols/burg1D-0002000100000-sol-h4o4}
        \caption{Limit: False}
    \end{subfigure}%
    \begin{subfigure}{.5\textwidth}
        \centering
        \includegraphics[width=\linewidth]{../figs/sols/burg1D-0002000000000-sol-h4o4}
        \caption{Limit: True}
    \end{subfigure}
    \caption{\Cref{ex:burgers_hest}. 4th order solution for $D=0.001$ with and without
        limiting.}
    \label{fig:burgers_sol}
\end{figure}

\end{example}

%\begin{figure}[h!]
%	\centering
%	\begin{tabular}{p{0.5\textwidth} p{0.5\textwidth}}
%		\vspace{0pt}
%
%\includegraphics[width=0.49\textwidth]{../figs/parametric/burgers_1D/orders_unlimited}
%		&
%		\vspace{0pt}
%
%\includegraphics[width=0.49\textwidth]{../figs/parametric/burgers_1D/orders_limited}
%	\end{tabular}
%	\caption{\Cref{ex:burgers_hest} average convergence rates for different choices of
%	$C_w$}
%	\label{fig:burgess_orders}
%\end{figure}

\begin{figure}[p!]
	\centering
	\includegraphics[width=\textwidth]{../figs/parametric/burgers_1D/convergences}
	\caption{\Cref{ex:burgers_hest}. Relative errors for different choices of $C_w$}
	\label{fig:burgers_conv}
\end{figure}
\clearpage

%\begin{figure}
%	\centering
%	\includegraphics[width=\textwidth]{../figs/err-sols/burgess_hesthaven-err-sol-i20cw1_d001_t2}
%	\includegraphics[width=\textwidth]{../figs/err-sols/burgess_hesthaven-err-sol-i20cw10_d001_t2}
%	\caption{Example 9 Exact solution (gray), numerical solution (orange) and their absolute difference (red) for
%	different orders and $h$. The left $y$ axes correspond to the solutions, the right ones to their difference.}
%	\label{fig:err_sol_burges_hest}
%\end{figure}
