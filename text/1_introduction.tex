% !TeX spellcheck = en_US
%%
%% Text of diploma thesis
%%
%% Tomáš Zítka
%%
In this work, we present the implementation of discontinuous Galerkin Finite
Elements Method (DG FEM) in software package \sfepy{} (Simple Finite Elements
for
Python) and results of experimental measurement of convergence of the method.
%and its comparison to grad-div, SUPGand PSPG stabilization of the classical
%FEM
%methods.
We consider several model problems, the basic one being a linear
advection problem for variable $p$ with constant advection velocity $\vec{a}$
$$
\pdiff{p}{t} - \vec{a}\cdot\nabla p = 0,
$$
with a Dirichlet boundary conditions
$$
%\begin{array}{ll}
p(t, \vec{x}) = p_D(t, \vec{x}) \; \text{ for } \vec{x} \text{ in } \partial\Omega,
$$
%$$
%\pdiff{p}{\vec{n}}(t, \vec{x}) = p_N(t, \vec{x}) & \text{ for }\vec{x}\text{
%in }
%\Gamma_N \subset \partial\Omega.
%\end{array}
%$$
which we gradually generalize by adding diffusion $D$, source term $g$ and nonlinearity 
$\vec{f}$
\begin{equation*}
\pdiff{p}{t} + \nabla\cdot \vec{f}(p) - D\Delta p = g.
\end{equation*}
The first main goal of this work is to provide an implementation of
the discontinuous Galerkin
method, which could be used to empirically study the behavior of the method but
also in academic and potential real-world applications and in education. The
second main goal is to use this implementation to analyze the behavior of the
method when applied to chosen model problems based on the equations above,
especially concerning the choice of different fluxes and penalty terms (see below).

The work is divided as follows: The introductory chapter summarizes literature
and basic concepts of DG FEM and introduces \sfepy{}. In the second chapter,
we derive the method for a model problem and explore the theory behind it. The
third chapter describes in detail the \sfepy{} package and the implementation of the
method. The fourth chapter presents the setup and results of numerical
experiments measuring convergence. In the concluding chapter, we discuss the
results and present suggestions for future work.

\section{Basic concepts and literature}
In continuous or classical finite element discretizations of the partial
differential equation, the solution is approximated by a combination of basis
functions whose supports span across multiple geometrical elements of the mesh
discretizing the computational domain. This enforces continuity of the solution
and provides a way of transferring information between the elements. In
discontinuous Galerkin FE methods, on the other hand, the basis functions used
in the approximation of test and state variables have supports limited to the
individual geometrical elements, much like piecewise approximation in finite
volume (FV) methods. This leads to compact discretization stencils and
allows discontinuities in the approximate solution but also requires fluxes at
element interfaces to be introduced in order to transfer information between
elements. As we shall see, these properties of the DG FE methods prove to be
useful in some applications and burdens in others.

The discontinuous approximation and compact stencils make DG FEM appealing for
multi-domain and multi-physics simulations \cite{DiPietro2012}. The possibility to
approximate discontinuous solutions proves useful in modeling so-called shock
waves in nonlinear conservation laws with small dissipation \cite{Kucera}(see
\Cref{ex:burgers_hest}). Inherent discontinuity of solution, however, brings
difficulties for diffusion dominated or otherwise naturally continuous
problems and forces introduction of so-called penalty terms (more in Section
\ref{se:diff_term}, \cite{Antonietti2013} and \cite{Kucera}). The introduction
of fluxes provides great flexibility of DG FEM and allows a straightforward
implementation of conservation laws which endows the method with good stability
properties when approximating advection dominated problems. Among disadvantages of
the use of fluxes belong a complicated theoretical analysis of the methods and
a lack
of an exact solution to Riemann problems for high order approximations in
individual mesh elements. The use of approximate fluxes for solving Riemann
problems with rough initial data with large gradients introduces oscillations
not present in FV methods mandating the use of so-called limiters (more in
\Cref{se:limiters}, \cite[Sec. 3.2.4]{DiPietro2012} and
\cite{Krivodonova2007}). The FE nature of DG FEM and the use of fluxes allows
the DG FEM to be interpreted both as Galerkin projection onto suitable energy
spaces as well as high order classical upwind finite volume schemes
\cite{Georgoulis2011}.

Although studied thoroughly, as literature cited above suggests, DG FE methods
still pose research challenges and promise new and potentially useful results
for numerical modeling. Among challenges are those mentioned above. One of
great promises is the so-called super-convergence observed for certain problems
\cite{Roe2017} which yet awaits to be leveraged in applications.


\section{\sfepy{} -- Simple Finite Elements in Python}

Simple finite elements in Python (\sfepy{},
\url{http://sfepy.org/}) is a software package providing FE based methods
along with a wide range of tools for defining, solving, and post-processing
variety of coupled PDEs in 1D, 2D, and 3D. It can be viewed both as a
black-box PDE solver and as a Python package which can be used for building
custom application \cite{Cimrman_Lukes_Rohan_2019}. The code of the package is
open-source published under New BSD-3 Clause license \cite{bsd3-lic} and is
available on Github (\url{https://github.com/sfepy/sfepy}) %\cite{sfepy-git}.
Detailed documentation with many examples can be found in \cite{sfepy-doc}.

\sfepy{} can use many FE based terms to build the PDEs to be solved. This
approach is reflected in Section \ref{ch:theory} where the discretization of
the equation is divided into discretization of individual terms, these are
then implemented individually in Chapter \ref{ch:implementation}. As of time
of writing \sfepy{} supports classical FEM and isogeometric analysis (IGA)
based FEM and provides tools for setting up, solving and post-processing
problems in applications like homogenization of porous media, acoustic waves
in thin perforated layers, finite element formulation of Schroedinger equation
or flow of a two-phase non-Newtonian fluid medium in a general domain
\cite{Cimrman_Lukes_Rohan_2019}.

There are several other software packages implementing  DG FEM, some of the
currently available codes are: hpGEM \cite{hpgem2007}
(\url{https://hpgem.org/}) which provides implementation of so-called
$hp$-methods in C++; FEniCS project \cite{fenics2015}
(\url{https://fenicsproject.org/}) which is well established numerical
software build on C/C++; PyFR \cite{pyfr2014} (\url{http://www.pyfr.org/})
which is  Python based framework for solving advection-diffusion type problems
that leverages locality of DG FE methods to run computations efficiently on
modern streaming architectures, such as Graphical Processing Units
(GPUs)\cite{pyfr2014}.
