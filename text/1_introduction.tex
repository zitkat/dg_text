In this work we present implementation of discontinuous Galerkin Finite 
Elements Method (DG FEM) in software package SfePy (Simple Finite Elements for 
Python) and results of experimental measurement of convergence of the method. 
%and its comparison to grad-div, SUPGand PSPG stabilization of the classical FEM 
%methods. 
We consider several model problems, the basic one being linear 
advection with constant advection speed $\vec{a}$
$$
\pdiff{p}{t} - \vec{a}\cdot\nabla p = 0,
$$
with boundary conditions
$$
\begin{array}{ll}
p(t, \vec{x}) = p_D(t, \vec{x}) & \text{ for } \vec{x} \text{ in } \Gamma_D \subset 
\partial\Omega, \\
$$
$$
\pdiff{p}{\vec{n}}(t, \vec{x}) = p_N(t, \vec{x}) & \text{ for }\vec{x}\text{ in } \Gamma_N \subset \partial\Omega.
\end{array}
$$
Which we gradually generalize adding source term $g$, nonlinearity $\vec{f}$ 
and diffusion $D$
\begin{equation*}
	\pdiff{p}{t} + \nabla\cdot \vec{f}(p) - D\Delta p = g.
\end{equation*}
First main goal of this work is to provide implementation of discontinuous Galerkin 
method which could be used to empirically study the behavior of the method but also in 
academic and potentially real world applications and in education. 
Second main goal is to use this implementation to analyze behavior of the method when 
applied to chosen model problems based on equations above especially with regard to 
choice of different fluxes and penalty terms (see below).

The work is divided as follows: The introductory chapter summarizes literature on 
and basic concepts of DG FEM and introduces Sfepy. In second chapter we derive the 
method for model problem and explore theory behind it. Third chapter describes 
in detail Sfepy package and implementation of the method. Fourth chapter 
presents setup and results of numerical experiments measuring convergence 
%with regard to the newest theoretical results presented in \cite{Roe2017}. 
%Fifth 
%chapter presents comparison with grad-div, SUPG \cite{Rapin2007} and PSPG 
%stabilization solved by an Oseen solver\cite{} applied to advection-diffusion 
%equation. 
In the concluding chapter we discuss the results and present 
suggestions for future work.

\section{Basic concepts and literature overview}
In continuous or classical finite element discretization of partial differential equation 
the solution is approximated by as combination of basis functions whose supports span 
across multiple geometrical elements of the mesh discretizing computational domain. This 
enforces continuity of the solution and provides a way of transferring information 
between the elements. In discontinuous Galerkin FE methods on the other hand the 
basis functions used in approximation of test and state variable have supports limited to 
the individual geometrical elements, much like piecewise approximation in finite volume 
(FV) methods. This leads to compact discretization stencils and allows for 
discontinuities in solution but also requires fluxes at element interfaces to be 
introduced in order to transfer information between elements. As we shall see these 
properties of the DG FE methods prove to be useful in some applications and burden in 
others.

Discontinuous approximation and compact stencil makes DG FEM appealing for 
multi-domain and multi-physics simulations \cite{DiPietro2012}.
Possibility to approximate discontinuous solutions proves useful in modeling so-called 
shock waves in nonlinear conservation laws with small dissipation \cite{Kucera}(see 
\Cref{ex:burgers_hest}). Inherent discontinuity of solution, however, brings difficulties 
for diffusion 
dominated or otherwise naturally continuous problems and forces introduction of so-called 
penalty terms (more in Section \ref{se:diff_term}, \cite{Antonietti2013} and  
\cite{Kucera}). Introduction of fluxes provides great flexibility of DG FEM and allows 
for straightforward implementation of conservation laws which endows method with 
good stability properties when approximating advection dominated problems. 
Disadvantages of the use of fluxes are complicated theoretical analysis of the 
methods and lack of exact solution to Riemann problem for high order approximation in 
individual mesh elements. Use of approximate fluxes for solving Riemann problems with 
rough initial data with large gradients introduces oscillations not present in FV methods 
mandating use of so-called limiters (more in \Cref{se:limiters}, \cite[Sec. 
3.2.4]{DiPietro2012} and \cite{Krivodonova2007}). The FE nature of DG FEM and use of 
fluxes allows the DG FEM to be interpreted both as Galerkin projection onto suitable 
energy spaces as well as high order classical upwind finite volume schemes 
\cite{Georgoulis2011}. 

Although studied thoroughly, as wast array of literature cited above suggests, DG FE 
methods still pose research challenges and promise new and potentially useful results for 
numerical modeling. Among challenges are those mentioned above. Among great promises is 
so called super-convergence observed for certain problems \cite{Roe2017} which yet awaits 
to be leveraged in applications.


\section{SfePy -- Simple Finite Elements for Python}

Simple finite elements for Python (SfePy, 
\url{http://sfepy.org/}) is a software 
package providing FE based methods along with wide range of tools for defining, solving 
and post-processing variety of coupled PDEs in 1D, 2D and 3D. It can be viewed both as a 
black-box PDE solver, and as a Python package which can be used for building custom 
application \cite{Cimrman_Lukes_Rohan_2019}. The code of the package is open-source 
published under New BSD-3 Clause license \cite{bsd3-lic} and is available on Github 
(\url{https://github.com/sfepy/sfepy}) %\cite{sfepy-git}.
Detailed documentation with many examples can be found in 
\cite{sfepy-doc}.

SfePy can use many FE based terms to 
build the PDEs to be solved. This approach is reflected in Section \ref{ch:theory}
where the discretization of the equation is divided into discretization of individual 
terms, these are then implemented individually in Chapter \ref{ch:implementation}. As of 
time of writing Sfepy supports classical FEM and isogeomteric analysis (IGA) based FEM 
and 
provides tools for setting up, solving and post-processing problems in applications like 
homogenization of porous media, acoustic waves in thin perforated layers,  finite element 
formulation of Schroedinger equation or flow of a two-phase non-Newtonian fluid medium in 
a general domain \cite{Cimrman_Lukes_Rohan_2019}.

There are several other software packages implementing  DG FEM, some of the currently 
available codes are:
hpGEM \cite{hpgem2007} (\url{https://hpgem.org/}) which provides implementation of 
so-called $hp$-methods in C++; 
FEniCS project \cite{fenics2015} (\url{https://fenicsproject.org/}) which is well 
established numerical software build on C/C++;
PyFR \cite{pyfr2014} (\url{http://www.pyfr.org/}) which is  Python based framework for 
solving advection-diffusion type problems that leverages locality of DG FE methods to run 
computations efficiently on modern streaming architectures, such as Graphical 
Processing Units (GPUs).






