% !TeX spellcheck = en_US
%%
%% Text of diploma thesis
%%
%% Tomáš Zítka
%%
\providecommand{\elint}{\int_{T^k}}
\providecommand{\elbint}{\int_{\partial T^k}}
\section{Terms and equations}
Basic equation we will be concerned with is the partial hyperbolic-elliptic 
equation for the unknown function $p$, $p : \Omega \rightarrow \realc$, where 
$\Omega \subset \realc$ is physical domain with boundary $\partial\Omega$ in 
the stationary form
\begin{equation}
\label{eq:hyp_diff}
\nabla\cdot \vec{f}(p) -  \nabla \cdot (D \nabla p) = g
\end{equation}
where $f$ is sufficiently smooth vector function $f: \realc 
\rightarrow\realc^n$, with the boundary conditions
\begin{align}\label{eq:diff_bcs}
p(t, \vec{x}) = p_D(t, \vec{x}) & \text{ for } \vec{x} \text{ in } 
                \Gamma_{Dir} \subset \partial\Omega \\
%\pdiff{p}{\vec{n}}(t, \vec{x}) = p_N(t, \vec{x}) & \text{ for }\vec{x}\text{ 
%in 
%} \Gamma_{New} \subset \partial\Omega
\end{align}
or in the transient form
\begin{equation}
    \label{eq:hyp_diff_trans}
    \pdiff{p}{t} + \nabla\cdot \vec{f}(p) -  \nabla \cdot (D \nabla p) = g
\end{equation}
with boundary conditions of the same form and initial condition
\begin{equation}
    \label{eq:diff_ic}
    p(0, \vec{x}) = p_0(\vec{x}).
\end{equation}
For this problem we will be concerned with discretization of the generally 
nonlinear hyperbolic term covered in \Cref{se:hyp_term}
\begin{equation}
    \nabla\cdot \vec{f}(p),
\end{equation}
which also covers discretization of linear advection term
\begin{equation}
\vec{a} \cdot \nabla p;
\end{equation}
diffusion term covered in \Cref{se:diff_term}
\begin{equation}
    -  \nabla \cdot (D \nabla p);
\end{equation}
source term in \Cref{se:source_term}
\begin{equation}
    g;
\end{equation}
and finally in \Cref{se:time_theory} we will 
treat discretization of temporal derivative term
\begin{equation}
\pdiff{p}{t}.
\end{equation}


\section{Finite dimensional discontinuous approximation space}
In order to discretize the terms and further the equations we first need to 
establish approximation space we will use, similarly to continuous FEM. We 
start by choosing suitable computational domain $\Omega_h$ which approximates 
domain $\Omega$. Since SfePy supports simplex and tensor product meshes, we 
will be concerned with space filling tessellations containing line segments 
for 1D problems respective only triangles or only quadrangles for 2D 
problems. Subscript $h$ denotes average diameter of the elements, 
$N$ will denote the number of elements of $\Omega_h$ and individual elements 
will be denoted by $T^k$ for $k=0, \ldots, N - 1$. Creating this suitable 
tessellation for arbitrary computational domain is by no means trivial task, 
however for the time being we will assume selected computational domain and 
mesh satisfies all conditions required bellow. First we define space of 
piecewise continuous functions  on $\Omega_h$ as 
\begin{equation}
    C^1(\Omega_h) =  \left\{v;\; v\vert_{T^k} \in C^1 \quad \forall T^k \in 
    \Omega_h \right\}
\end{equation}
and broken Sobolev space on $\Omega_h$ as
\begin{equation}\label{eq:sobh}
    W^{1, 2}(\Omega_h) = \left\{v;\; v\vert_{T^k} \in W^{1, 2}\quad \forall T^k 
    \in \Omega_h \right\}
\end{equation}
In finite dimensional discretization we will work in finite dimensional 
subspaces of $W^{1, 2}(\Omega_h)$. On each element $T^k$ we express 
solution $p(t, \vec{x})$ locally as a linear combination of polynomial basis functions
\begin{equation}
\label{eq:el_lin_comb}
    p(t, \vec{x})|_{T^k} \approx p_h^k(t, \vec{x}) = \sum\limits_{n=0}^{N_{base} - 1} 
    P_n^k\psi^k_n(\vec{x}),
\end{equation}
i.e. as a function in local space
\begin{equation}
    V_{T^k} = \text{span}\big\{ \psi_n^k(\vec{x}), \; n = 0,1, \dots  
    N_{base}-1\big\},
\end{equation}
where $N_{base}$ is number of basis functions we use in approximation and hence 
dimension of approximation space. This number is directly tied to approximation 
order and is dependent on the type of mesh elements. We also require that
\begin{equation}
    \text{supp}\big\{\psi_n^k(\vec{x}\big)\} = T^k \quad n \in \{0,1, 
    \dots  N_{base}-1\}.
\end{equation}
This means that basis functions are localized to individual elements and allow 
us to represent discontinuous solution, unlike in the classical FEM, where 
supports of basis functions overlap, spanning multiple elements and  thus 
enforcing continuity of solution.

In 1D setting $\psi_n^k(\vec{x})$ is composed of Legendre polynomials shifted 
and truncated to interval $[0, 1]$, we denote $L^r(x)$ the Legendre 
polynomial of order $r$. These Legendre polynomials are orthogonal and hence 
the set  $\{\psi_n^k(x)| \; \psi_n = L^n, \; n = 0,1, \dots  N_{base} - 
1\}$ forms basis of  $N_{base}$ dimensional space we denote $M = N_{base} - 
1$ maximal order of used Legendre polynomials.

In 2D the basis functions $\psi_n^k(\vec{x})$ are composed from Legendre 
polynomials  in such a way that set $\{\psi_n^k(\vec{x})| \; n = 0,1, \dots  
N_{base} - 1\}$ is orthogonal with respect to the local 
scalar product
\begin{equation}
    \label{eq:scalar_prod_dk}
    (p, v)_{T^k} = \int_{T^k} p \cdot v.
\end{equation}
hence forming basis of $N_{base}$-dimensional space. 

%Setting 
%\begin{equation}
%N_{base} =  \frac{(Ord + 1) \cdot (Ord + 2) \cdot ... \cdot (Ord + d)}{d!},
%\end{equation}
%where $Ord$ is desired order of approximation -- 0 for finite volumes, 1 for 
%linear approximation, 2 for quadratic approximation etc. and $d$ is dimension 
%of $\Omega_h$; as number of basis functions, provides maximal number of basis 
%function for given order which remain orthogonal. 

Concrete shape of $\psi_n^k(\vec{x})$ depends on topology of the mesh elements. For 
tensor product meshes i.e. quadrilateral elements we use straight-forward 
tensor product of Legendre polynomials. If we denote $M$ the order of 
approximation, $d$ dimension of geometric space (i.e., 2 in our case), we get 
quadrilateral element basis functions in the form
\begin{equation}
\psi_n^k(\vec{x}) = L^r(x)L^s(y)\quad r, s = 0,1, \dots, M
\end{equation}
and dimension of the approximation space is
\begin{equation}
    N_{base} = (M + 1)^d.
\end{equation}
%and for "cube" is
%\begin{equation}
%\psi_n^k(\vec{x}) = L^r(x)L^s(y)L^d(z)\quad r,s,d = 0,1, \dots, Ord \text{ 
%s.t. } r + s + d = n \leq Ord.
%\end{equation}

In case of simplex meshes, the shape of $\psi_n^k(\vec{x})$ is result of 
Gram-Schmidt orthogonalization process on canonical basis 
$$
\left\{ x^ry^s,  \quad r, s = 0,1, \dots, M \text{ s.t. } r + s \leq M\right\}
$$
with respect to scalar product \eqref{eq:scalar_prod_dk}
and its shape is much more elaborate. The Jacobi polynomials are needed to 
represent the basis. If we denote $J^{\alpha, \beta}_m$ the $m$-th order 
Jacobi polynomial, the individual basis functions can be written in the form 
\cite{Hesthaven2008}
\begin{equation}
    \psi_n^k(\vec{x}) = J_r(a)J^{2s+1, 0}_s(b)(1 - b)^r\quad r, s = 0,1, \dots, 
    Ord \text{ s.t. } r + s = n \leq Ord
\end{equation}
where
\begin{equation}
    a = 2 \frac{1 + x}{1 - y} - 1, b = y.
\end{equation}
We get local polynomial space of dimension
\begin{equation}
N_{base} =  \frac{(M + 1) \cdot (M + 2) \cdot ... \cdot (M + d)}{d!}.
\end{equation}
In both 2D cases the mapping between $n$ and $r$ and $s$ (i.e. ordering of 
basis functions) 
\begin{equation}
\label{eq:bindx}
n = indx(r, s)
\end{equation}
with its reverses
\begin{equation}
r = indx_1(n),\; s = indx_2(n)
\end{equation}
is theoretically arbitrary, in practice we choose it so that the basis 
functions are ordered hierarchically i.e.
\begin{equation}
n \leq m \Leftrightarrow indx_1(n) + indx_2(n) \leq indx_1(m) + indx_2(m)
\end{equation}
Used mapping is probably best expressed using procedural programming language,
see Listing \ref{lst:iter_by_order}.
In the whole computational domain $\Omega_h$ the solution can be than thought 
of as being a member of direct sum of local spaces
\begin{equation}
    Le_{\Omega_h}^{M} = \bigoplus\limits_{T^k \in \Omega_h} V_{T^k}
\end{equation}
which is finite dimensional subspace of broken Sobolev space defined in 
\eqref{eq:sobh}, 
that is $Le_{\Omega_h}^{M} \subset  W^{1,2}(\Omega_h)$, $Le$ stands for Legendre, it has 
dimension
\begin{equation}\label{eq:dim_legh}
    N_{dof} = \text{dim}(Le_{\Omega_h}^{M}) = N\cdot N_{base}
\end{equation}
This is local basis commonly used in literature \cite{Hesthaven2008}, 
\cite{Bokhove2008}, however there are also other usable bases, which must not 
be orthogonal or polynomial \cite{Yuan2006}. We will always use full 
basis of the functions, however implementations contains mechanism to omit 
some of them for testing purposes.

\newpage
\section{Spatial discretization}
We can now start formulating discretization in space domain. To discretize 
equation \eqref{eq:hyp_diff} in finite elements manner we first devise weak 
formulation of the problem. First we choose the unknown $p$ and arbitrary test 
function $w$ to be from $C^1(\Omega_h)$. And multiply equation 
\eqref{eq:hyp_diff} by a test function $w$ also from $C^1(\Omega_h)$, we get
\begin{equation}
    \nabla\cdot \vec{f}(p)\cdot w(\vec{x}) 
    -  \nabla \cdot (D \nabla p)\cdot w( 
    \vec{x}) = g\cdot w(\vec{x}),
\end{equation}
after integration over the domain $\Omega$ we get
\begin{equation}\label{eq:hyp_int}
     \int_{\Omega}\nabla\cdot \vec{f}(p)\cdot w(\vec{x}) 
     - \int_{\Omega}\nabla \cdot (D \nabla p)\cdot w(\vec{x}) 
     = \int_{\Omega}g\cdot w(\vec{x}).
\end{equation}
This holds for every cauchy sequence of functions ${p_n}$, ${w_n}$ and using 
Lebesgue dominated convergence theorem we can formulate the problem on closure 
of $C^1(\Omega_h)$ i.e. for $p \in W^{1,2}(\Omega_h)$ and  $w \in 
W^{1,2}(\Omega_h)$, obtaining the equation \eqref{eq:hyp_int} in the form (we 
drop 
independent variables $t$ and $\vec{x}$ notations for brevity) 
\begin{equation}
    \label{eq:sum_int_hyp}
    \sum\limits_{k=0}^{N}\left(
    \elint\nabla\cdot \vec{f}(p)\cdot w
    - \elint\nabla \cdot (D \nabla p)\cdot w 
        \right)
     =\sum\limits_{k=0}^{N}\left(\elint g\cdot w\right).
\end{equation}
Having arrived to the "broken" integral formulation of the equation we will now 
focus on discretization of individual terms within mesh elements.

\subsection{Hyperbolic term discretization}
\label{se:hyp_term}

Using Green's theorem on the first integral term in \eqref{eq:sum_int_hyp} we 
get
\begin{equation}
    \label{eq:hyp_term}
    \int_{T^k} \nabla\cdot \vec{f}(p)\cdot w = % \int_{T^k} \vec{f}\; \nabla p w = 
    \int_{\partial{T^k}} \vec{n}\vec{f}(p)w - \int_{T^k} \vec{f}(p)\cdot\nabla w,
\end{equation}
where $\vec{n}$ is the normal vector to the boundary of $T^k$ we denote 
$\partial T^k$. Approximation of the value of $\vec{f}$ on the boundary of the 
element plays key role in discretization using DG FE methods, the issue is 
that the approximate solution is discontinuous across the boundary of an 
element and two values are actually present, $p_{in}$ inner to the element and 
$p_{out}$ outer, coming from its neighbor across particular part of the 
boundary. Since we deal with 2D 
elements with polygonal boundary the integral over the boundary can be expressed as sum 
of integrals over line segments ${F^k_i}$, $i=0,1,2$ for triangular meshes or 
$i=0,1,2,3$ for quadrilateral meshes, forming the boundary
\begin{equation}
    \sum_{i=0}^{N_f} \int_{F^k_i} \vec{n_f}\vec{f}(p)w
\end{equation}
If we denote $T^{k'}$ element sharing line segment $F^k_i$ with element $T^k$
value $p_{out}$ corresponds to approximation in this element i.e. $p_{out} = 
p^{k'}_h$. For simplicity of notation we continue using integral over the whole 
boundary of $T^k$ implicitly assuming that $p_{out}$ changes as described above. To 
approximate unknown value of $\vec{f}(p)$  we will use approximate flux $f^*(p_{in}, 
p_{out})$ obtaining the first term on the right-hand side in the form
\begin{equation}
    \label{eq:flux_integral}
    \int_{\partial{T^k}} \vec{n}\vec{f}(p)w = \int_{\partial{T^k}} \vec{n} 
    \cdot f^{*} (p_{in}, p_{out})\cdot w.
\end{equation}
In our setting we will use so-called local Lax-Friedrichs flux 
exclusively, although there are many other possible fluxes, for examples see 
\cite{Kucera, Cockburn2001a}, their later implementation should be straightforward (more 
in Section \ref{se:adv_flux_term_imp}). Lax-Friedrichs flux as given in 
\cite{Hesthaven2008} 
is of the form
\begin{equation}
    \label{eq:lax-frieflux}
    f^{*}(p_{in}, p_{out}) =   \frac{\vec{f}(p_{in}) + \vec{f}(p_{out})}{2}  + (1-\alpha) \vec{n}\frac{C}{2}(p_{in} - 
    p_{out}),
\end{equation}
where $\alpha \in [0, 1]$ is parameter adjusting nature of the flux, $\alpha = 0$ for purely upwind scheme, 
$\alpha = 1$ for central scheme,  and
\begin{equation}
    C = \max_{p \in [p_{in} : p_{out}]} \abs{n_x \pdiff{f_1}{p} + n_y \pdiff{f_2}{p}} =
        \max_{p \in [p_{in} : p_{out}]} \abs{\vec{n}\cdot\frac{d\vec{f}}{dp}(p)},
\end{equation}
where $[p_{in} : p_{out}]$ denotes closed interval 
$$\big[\min_{\partial{T^k}} (\min(p_{in}, p_{out})), 
       \max_{\partial{T^k}}(\max(p_{in}, p_{out}))\big].$$ 
Note that for linear case where $f(p) = \vec{a}p$ and $\frac{d\vec{f}}{dp}(p) = \vec{a}$  
the $C$ reduces to
\begin{equation}
    C = |\vec{n}\vec{a}|.
\end{equation}
In this formulation $C$ constitutes upper bound on wave speed at the boundary interface.
In order to simplify notation we denote "jump in" 
quantity $p$ across boundary 
\begin{equation}
    [p] = p_{in} - p_{out}
\end{equation}
and the average of $p$ across boundary  
\begin{equation}
    \label{eq:avrg}
    \manglebracs{p} = \frac{p_{in}+p_{out}}{2}.
\end{equation}
Using this notation we can write
\begin{equation}
    f^*(p_{in}, p_{out}) = \manglebracs{p} + (1-\alpha) \vec{n}\frac{C}{2}[p].
\end{equation}



After approximating $p$ and $w$ on element $T^k$ as linear combinations of 
basis functions as in 
\eqref{eq:el_lin_comb},
\begin{align}\label{eq:state_epprox}
    p(t, \vec{x}) \approx \sum\limits_{i=0}^{N_{base}-1} P_i^k\psi_i(\vec{x})
\end{align}
\begin{align}\label{eq:test_approx}
    w(t, \vec{x}) \approx \sum\limits_{j=0}^{N_{base}-1} W_j^k\psi_j(\vec{x})
\end{align}
and substituting \eqref{eq:flux_integral} to \eqref{eq:hyp_term} we arrive to 
%\begin{equation}
%    \sum\limits_{k=0}^{N}
%        \left(
%            \int_{T^k}\pdiff{p}{t}\cdot w + \int_{T^k} \vec{f}(p)\cdot\nabla w - 
%\int_{\partial{T^k}} \vec{n} \cdot f^{*} (p_{in}, p_{out})\cdot w
%        \right)
%    = 0.
%\end{equation}

%\begin{align}
%    \sum\limits_{k=0}^{N}
%        \left(\vphantom{\sum\limits_{i=0}^{N_{base}-1}} \right.
%        & \int_{T^k}\pdiff{}{t}\sum\limits_{i=0}^{N_{base}-1} P_i^k(t)\psi_i\cdot 
%\sum\limits_{j=0}^{N_{base}-1} 
%W_j^k\psi_j\nonumber\\
%       +&\int_{T^k} \vec{f}\Big(\sum\limits_{i=0}^{N_{base}-1} 
%P_i^k(t)\psi_i\Big)\cdot\nabla\Big(\sum\limits_{j=0}^{N_{base}-1} 
%W_j^k\psi_j\Big)\nonumber\\ 
%       -&\left.\int_{\partial{T^k}} \vec{n} \cdot f^{*} (p_{in}, p_{out})\cdot 
%\sum\limits_{j=0}^{N_{base}-1} 
%W_j^k\psi_j \right) = 0.
%\end{align}
\begin{multline}
    \label{eq:disc_hyp_term}
    \int_{T^k} \nabla\cdot \vec{f}(p)\cdot w \approx \int_{T^k} 
    \vec{f}\Big(\sum\limits_{i=0}^{N_{base} - 1} 
    P_i^k\psi_i\Big)\cdot\nabla\Big(\sum\limits_{j=0}^{N_{base} - 1} 
    W_j^k\psi_j\Big)\\    
    -\int_{\partial{T^k}} \vec{n} \cdot f^{*} (p_{in}, p_{out})\cdot 
    \sum\limits_{j=0}^{N_{base} - 1} W_j^k\psi_j
\end{multline}
since this approximation holds for every test function $w \in 
Le_{\Omega_h}^{M}$ we can choose $W_j^k = 1 \; \forall \; 
j, k$, using summation notation for clarity we can then write terms on the  right-hand 
side of \eqref{eq:disc_hyp_term} as 
\begin{equation}\label{eq:hyp_stiff_app}
    a^C_\mathrm{hyp}(\mathbf{p}) = \int_{T^k} \vec{f}(P_i^k\psi_i)\cdot\nabla\psi_j, 
\end{equation}
\begin{equation}\label{eq:hyp_flux_app}
    a^F_\mathrm{hyp}(\mathbf{p}) = \int_{\partial{T^k}} \vec{n} \cdot f^{*} (p_{in}, 
    p_{out})\cdot\psi_j.
\end{equation}
where $\mathbf{p}$ denotes vector of unknowns $P^k_i$ grouped by element as follows
\begin{equation}
    \mathbf{p} = \left(P^0_0, P^0_1, P^0_2, \ldots, P^0_{N_{base}-1}, \; \ldots \; ,
    P^{N}_0, P^{N}_1, P^{N}_2, \ldots, P^{N}_{N_{base} - 1}  \right).
\end{equation}
Note that in \eqref{eq:hyp_flux_app} we do not include the minus sign in front 
of the flux term.
This finalizes discretization of general hyperbolic term 
$\nabla\cdot\vec{f}(p) 
\cdot w$, the two terms -- integral over element $T^k$ (often called stiffness 
term) and integral over its surface $\partial T^k$ -- are implemented in SfePy 
as \pysauce{AdvectDGFluxTerm} and \pysauce{ScalarDotMGradScalarTerm} in special 
case $f = \vec{a}p$ and as \pysauce{NonlinearHyperDGFluxTerm} and 
\pysauce{NonlinScalarDotGradTerm} in general case. See Section 
\ref{se:adv_flux_term_imp} in Chapter \ref{ch:implementation} for details on 
implementation.


%\begin{equation}
%    \sum\limits_{k=0}^{N}
%    \left(
%        \int_{T^k}\pdiff{}{t}P_i^k\psi^i\cdot W_j^k\psi^j + 
%        \int_{T^k} \vec{f}(P_i^k\psi^i)\cdot\nabla W_j^k\psi^j 
%        - \int_{\partial{T^k}} \vec{n} \cdot f^{*} (p_{in}, p_{out})\cdot W_j^k\psi^j
%    \right) = 0.
%\end{equation}
%and using orthogonality of basis ${\psi_i}$ we get
%\begin{equation}
%    \sum\limits_{k=0}^{N}
%    \left(
%        \int_{T^k}\pdiff{}{t}P_i^k \psi^2_i\cdot W^k_j + 
%        \int_{T^k} \vec{f}(P_i^k\psi^i)\cdot\nabla W_j^k\psi^j 
%        - \int_{\partial{T^k}} \vec{n} \cdot f^{*} (p_{in}, p_{out})\cdot W_j^k\psi^j
%    \right) = 0.
%\end{equation}
%For simplicity we will be now concerned with the linear scalar advection equation with constant coefficient $\vec{a}$ 
%representing advection velocity of the quantity $p$ i.e. $\vec{f}(p) = \vec{a}p$. This allows us to transform the formulation to
%simpler form
%\begin{equation}
%    \label{eq:adv_disc1}
%    \sum\limits_{k=0}^{N}
%    \left(
%        \int_{T^k}\pdiff{}{t}P_i^k \psi^2_i\cdot W^k_j + 
%        \int_{T^k} \vec{a}P_i^k\psi^i\cdot\nabla W_j^k\psi^j 
%        - \int_{\partial{T^k}} \vec{n} \cdot f^{*} (p_{in}, p_{out})\cdot W_j^k\psi^j
%    \right) = 0.
%\end{equation}
%\begin{eqnarray}
%    \label{eq:adv_disc2}
%    \sum\limits_{k=0}^{N}
%    \left(
%        \int_{T^k}\pdiff{}{t}P_i^k \psi^2_i\cdot W^k_j + 
%        \int_{T^k} \vec{a}P_i^kW_j^k\psi^i\cdot\nabla \psi^j 
%        - \int_{\partial{T^k}} \vec{n} \cdot f^{*} (p_{in}, p_{out})\cdot W_j^k\psi^j
%    \right) = 0.
%\end{eqnarray}
%...

%This provides us with discrete operator $\mathcal{L}$ such that
%$$
%\pdiff{u}{t} = \mathcal{L}(u, t)
%$$
%we use in time discretization.


\subsection{Elliptic term discretization}
\label{se:diff_term}
To discretize the elliptic diffusion term
$$
\elint\nabla\cdot(D\nabla p) w
$$
we use Green's theorem as well, obtaining
\begin{equation}
    \label{eq:diff_after_green}
    \elint\nabla \cdot (D \nabla p)\cdot w  = \elbint D(\nabla p\cdot\vec{n}) w - \elint D\nabla p \nabla w
\end{equation}
on the boundary $\partial T^k$ we define, using notation from \eqref{eq:avrg},
\begin{equation}
    \nabla p = \frac{\nabla p_{in} + \nabla p_{out}}{2} = \manglebracs{\nabla p}.
    \label{eq:avrg_grad_state}
\end{equation}
Substituting \eqref{eq:avrg_grad_state} to \eqref{eq:diff_after_green} we get 
so-called incomplete scheme
\begin{equation}
        \elbint D \manglebracs{\nabla p} \cdot \vec{n} [w] - \elint D\nabla p 
        \nabla w.
\end{equation}
Due to regularity of $p$ the $[p(t, \cdot)] = 0$ holds \cite[p. 14]{Kucera} and term
\begin{equation}\label{eq:diff_right}
    \elbint D \manglebracs{\nabla w }\cdot \vec{n} [p]
\end{equation}
vanishes.
We can then create symmetric resp. non-symmetric scheme by adding term 
\eqref{eq:diff_right} with "$+$" resp. "$-$" sign \cite{Kucera}. However either of these 
schemes is not stable and we need to compensate for the discontinuities of the $p$ across 
element boundaries by adding interior penalty term \cite{Kucera, 
Antonietti2013}
\begin{equation}
    \nu \elbint C_w \cdot \frac{Ord^2}{d(\partial T^k)} [p][w]
\end{equation}
where constant $\nu$ captures properties of diffusion tensor $D$, in case $D = 
\varepsilon, \; \varepsilon > 0$ we set $\nu = \varepsilon$ and $C_w$ is 
parameter 
at our disposal used to fine tune the penalty term.
See examples \ref{ex:laplace} \ref{ex:quart1}, \ref{ex:quart2}, 
\ref{burgers_hest} and \ref{ex:kucera} in chapter 
\ref{ch:convergence} for effects of different choices of the value of $C_w$. 
And finally $d(\partial T^k)$ is the volume of the boundary of $T^k$. To 
simplify notation 
we denote
\begin{equation}\label{eq:diff_penalty_sigma}
\sigma = \nu C_w \cdot \frac{Ord^2}{d(\partial T^k)}.
\end{equation} 
We further proceed as for hyperbolic term. By replacing $p$ and $w$ by their 
finite dimensional approximations \eqref{eq:state_epprox} and 
\eqref{eq:test_approx} and using the fact that the test function $\psi^k_i$ 
vanishes outside element $T^k$ and hence
\begin{eqnarray}
    \manglebracs{\psi^k_i} = \frac{\psi^k_i}{2},
\end{eqnarray}
\begin{eqnarray}
    [\psi^k_i] = \psi^k_i
\end{eqnarray}
holds, we obtain individual terms needed to discretize diffusion term in forms
\begin{equation}\label{eq:diff_left_approx}
    \elbint D \manglebracs{\nabla p} \cdot \vec{n} [w] 
    \approx
    a^L_\mathrm{diff}(\mathbf{p}) = 
        \elbint D \manglebracs{P^k_i\nabla\psi_i} \cdot \vec{n}[\psi_j] =
        \elbint D \manglebracs{P^k_i\nabla\psi_i} \cdot \vec{n}\psi_j,
\end{equation}
\begin{equation}\label{eq:diff_right_approx}
        \elbint D \manglebracs{\nabla w }\cdot \vec{n} [p] 
        \approx
        a^R_\mathrm{diff}(\mathbf{p}) =
            \elbint D \manglebracs{\nabla \psi_j }\cdot \vec{n} [P^k_i\psi_i] =
            \elbint D \frac{\nabla \psi_j}{2}\cdot \vec{n} 
            [P^k_i\psi_i],
\end{equation}
\begin{equation}\label{eq:diff_laplace_approx}
    \elint D\nabla p \nabla w 
    \approx
    a^C_\mathrm{diff}(\mathbf{p})=
        \elint D\nabla P^k_i\psi_i \nabla \psi_j,
\end{equation}
\begin{equation}\label{eq:diff_penalty_approx}
        \elbint \sigma [p][w] 
        \approx
        a^P_\mathrm{diff}(\mathbf{p})=
         \elbint \sigma [P^k_i\psi_i][\psi_j] 
         = \elbint \sigma [P^k_i\psi_i]\psi_j.
\end{equation}


\subsection{Source term discretization}
\label{se:source_term}
Discretization of source term is for DG FEM same as for continuous FEM, 
in term
\begin{equation}
    \int_{\Omega_h} g\cdot w
\end{equation}
we take test function from the broken Legendre polyspace obtaining 
\begin{equation}
    \sum\limits_{k=0}^{N}\left(\elint g\cdot w\right)
\end{equation}
after substituting \eqref{eq:test_approx} and choosing $W_j^k = 1 \; \forall \; 
j, k$ we get
\begin{equation}
\sum\limits_{k=0}^{N - 1}\sum\limits_{j=0}^{N_{base} - 1}\left(\elint g\cdot 
\psi_j \right)
\end{equation}
hence for element $T^k$ and test function $\psi_j$ the source term coefficient 
is
\begin{equation}
    b_{source} = \elint g\cdot \psi_j.
\end{equation}

\newpage
\section{Temporal discretization}
\label{se:time_theory}
%\begin{equation} \label{eq:test_theq_time}
%\pdiff{p}{t}\cdot w(t, \vec{x}) + \nabla\cdot \vec{f}(p)\cdot w(t, \vec{x}) 
%-  \nabla \cdot (D \nabla p)\cdot w(t, 
%\vec{x}) = g\cdot w(t, \vec{x}),
%\end{equation}
%
%\begin{equation}
%\label{eq:sum_int_hyp_time}
%\sum\limits_{k=0}^{N}\left(    \elint\pdiff{p}{t}\cdot w
%\elint\nabla\cdot \vec{f}(p)\cdot w
%- \elint\nabla \cdot (D \nabla p)\cdot w 
%\right)
%=\sum\limits_{k=0}^{N}\left(\elint g\cdot w\right).
%\end{equation}
In discretizing transient equation \eqref{eq:hyp_diff_trans} we use the 
discretization of terms devised for stationary equation, with the important 
difference that the discretization coefficients in \eqref{eq:state_epprox} now 
depend on time, that is
\begin{equation}
    P^k_i = P^k_i(t).
\end{equation}
By applying analogous [\todo approximation] to the transient term we obtain
\begin{equation}
    \elint\pdiff{p}{t}(t)w \approx  \fdiff{P^k_i}{t}(t)\elint\psi_i\psi_j. 
\end{equation}
Using this discretization in \eqref{eq:hyp_diff_trans} we obtain system of 
ordinary differential equations for unknown coefficients 
$P^k_i(t)$ in variable $t$
\begin{equation}\label{eq:time_diff}
    \mathbf{M}  \fdiff{\mathbf{p}}{t}(t) + \mathcal{L}(\mathbf{p}(t)) = 0,
\end{equation}
where $\mathcal{L}$  is composed from discretized terms  derived in previous 
sections and $\mathbf{M}$ is a matrix composed of blocks $M^{k,l},\; k,l=0, 
\ldots , N$ of the 
form
\begin{equation}
(M^{k,l})_{ij} = ( \psi^k_i,\psi^l_j)_{T^k}.
\end{equation}
Since the basis functions are orthogonal with respect to the scalar product 
\eqref{eq:scalar_prod_dk} individual blocks are diagonal and since basis 
function $\psi^k_i$ vanish outside of the element $T^k$, matrix $\mathbf{M}$ 
is block diagonal 
\begin{equation}
\mathbf{M}  = \begin{pmatrix}
        M^{0,0}      &    0   &\cdots&  0   &\cdots&0\\
        0         &   M^{1,1}  &      &      &      &\\
        \vdots     &    0   &\ddots&      &  0   &\vdots\\
      \vdots     & \vdots &      & M^{k,k}  &      &\\
        \vdots   & \vdots & 0    &      &\ddots&0\\
        0         &   0    &\cdots&\cdots&   0  &M^{N,N} 
    \end{pmatrix}.
\end{equation}
Thanks to this the inverse of $\mathbf{M}$ is trivial and we can rewrite 
\eqref{eq:time_diff} as
\begin{equation}
     \fdiff{\mathbf{p}}{t}(t) + \mathbf{M}^{-1}\mathcal{L}(\mathbf{p}(t)) = 0,
\end{equation}
denoting
\begin{equation}
    \bar{\mathcal{L}} =\mathbf{M}^{-1}\mathcal{L}, 
\end{equation}
we can write equation 
\eqref{eq:time_diff} in the form
\begin{equation}\label{eq:time_simple}
\fdiff{\mathbf{p}}{t}(t) + \bar{\mathcal{L}}(\mathbf{p}(t)) = 0.
\end{equation}
There is plethora of different schemes for evolving equation 
\eqref{eq:time_simple} we will only present basic forward Euler scheme and so called 
total variations diminishing Runge-Kutta scheme of 3rd order.


\paragraph{Forward Euler scheme} In forward euler scheme we approximate time 
derivation by forward difference i.e.
\begin{equation}
    \fdiff{\mathbf{p}}{t}(t) \approx     \frac{\mathbf{p}^{n + 1} - 
    \mathbf{p}^{n}}{\Delta t},
\end{equation}
where $n$ denotes time step. Substituting into equation \eqref{eq:time_simple} 
yields
\begin{equation}
    \frac{\mathbf{p}^{n + 1} - \mathbf{p}^{n}}{\Delta t} + 
     \bar{\mathcal{L}}(\mathbf{p}^n, t^n) = 0
\end{equation}
and after rearranging to obtain explicit equation for $\mathbf{p}^{n + 1}$ we 
get
\begin{equation}
\mathbf{p}^{n + 1} = \mathbf{p}^{n} - {\Delta t} 
\bar{\mathcal{L}}(\mathbf{p}^n, t^n).
\end{equation}
Forward Euler scheme is first order in time. We use it to define the so-called 
total variation diminishing property of $\bar{\mathcal{L}}$, that is the total 
variation of the numerical solution in one dimension
\begin{equation}\label{eq:TV}
    TV(p) = \sum_{k} \abs{p_{k + 1} - p_{k}},    
\end{equation}
where $k$ ranges over subsequent 1D mesh elements, does not increase with time i.e.
\begin{equation}
    TV(\mathbf{p}^{n + 1}) \leq TV(\mathbf{p}^{n})
\end{equation}
under update by forward Euler scheme.
This motivates usage of following TVD Runge-Kutta method \cite[p. 73]{Gottlieb2002}. 



\paragraph{TVD Runge-Kutta 3rd order scheme}
Third order total variations diminishing Runge-Kutta scheme \cite{Gottlieb2002} 
is three step scheme that maintains the TVD property while achieving 3rd order 
accuracy in time.
\begin{equation}    
    \begin{aligned}
        \mathbf{p}^{(1)} &= \mathbf{p}^n - \Delta t  
        \bar{\mathcal{L}}(\mathbf{p}^n), \\
        \mathbf{\mathbf{p}}^{(2)} &= \frac{3}{4}\mathbf{p}^n 
        +\frac{1}{4}\mathbf{p}^{(1)} - \frac{1}{4}\Delta t 
         \bar{\mathcal{L}}(\mathbf{p}^{(1)}),\\
        \mathbf{p}^{(n+1)} &= \frac{1}{3}\mathbf{p}^n 
        +\frac{2}{3}\mathbf{p}^{(2)} - \frac{2}{3}\Delta t 
         \bar{\mathcal{L}}(\mathbf{p}^{(2)}).
    \end{aligned}
\end{equation}


\subsection{Hyperbolic term stability requirement}
\todo Courant-Fridrichs-Levy condition adjusted for high order approximation 
mandates 
\begin{equation}
    \Delta t \leq c\frac{\Delta x^2}{D} \cdot \frac{1}{2M + 1},
\end{equation} 
where $c \leq 1$ is adjustable parameter.


\subsection{Elliptic term stability requirement}
\todo
\begin{equation}\label{eq:cfl_cond}
    \Delta t \leq c\frac{\Delta x}{\norm{\vec{a}}} \cdot \frac{1}{2M + 1},
\end{equation}
where $c \leq 1$ is adjustable parameter.

\newpage
\section{Initial condition discretization}
Initial condition 
$$
p_0(\vec{x})
$$
is discretized in straight-forward manner as an orthogonal projection into the 
finite dimensional space $Le_{\Omega_h}^{M}$ on domain $\Omega_h$. That is
by solving
\begin{equation}\label{eq:init_proj}
    (P^k_i)^0\elint\psi_i\psi_j = \elint p_0(\vec{x})\psi_j,
\end{equation}
for $(P^k_i)^0$. We use the mass matrix notation and the fact that 
it is diagonal and get
\begin{equation}
    (P^k_i)^0 = \frac{1}{(\psi^k_i,\psi^k_i)_{T^k}}\elint p_0(\vec{x})\psi_i.
\end{equation}




\section{Boundary conditions}
Unlike it is common in literature we postponed treatment of the boundary 
conditions (BCs) until now. The reason is to keep the theoretical discussion 
closely tied with implementation. This allows us to clearly demonstrate how 
method works, hopefully providing reader with enough information and 
understanding to modify it. In our implementation treatment of boundary 
condition is separated from term implementation i.e. terms do not have any 
information about boundary conditions, they are merely passed data, which 
already satisfy BCs.
In expressions
\begin{equation}
      \manglebracs{p} = \frac{p_{in} + p_{out}}{2}
\end{equation}
and 
\begin{equation}
    [p] = p_{in} - p_{out}
\end{equation}
the missing outer value $p_{out}$ in boundary elements is substituted by
\begin{itemize}
    \item value of Dirichlet boundary condition, 
    \item value in corresponding neighbor cell on periodic boundary,
\end{itemize} 
whenever there is no direct neighbor.
In implementations this is ensured in terms themselves by getting corresponding values 
from \pysauce{DGField} through method \pysauce{get_both_facet_base_vals}.

\newpage
\section{Limiters}
\label{se:limiters}
In high order DG FEM oscillations, which can significantly decrease quality of 
solution, occur even when Courant-Friedrichs-Levy condition \eqref{eq:cfl_cond} is met. 
To combat this limiter needs to be used, in the following section we present moment 
limiters for 1D and 
2D problems.

\subsection{Moment limiter}
Moment limiter due to Krivodonova \cite{Krivodonova2007} leverages idea that 
coefficients for higher order basis functions in hierarchically ordered 
Legendre basis represent derivatives of lower order data and uses this to 
limit the derivative of order $i$ in a given cell using derivatives of order 
$i - 1$ in neighboring cells. This kind of limiter unlike others does not 
reduce the solution to first-order accuracy. Unfortunately this kind of 
limiting is so far available only in one dimensional problems and in two 
dimensional problems with tensor product meshes.

\subsubsection{One-dimensional limiting}
\label{sse:moment_lim_1D}
We limit the solution in cell $T^k$
\begin{equation}
\label{eq:el_lin_comb_lim}
p_h^k(t, \vec{x}) = \sum\limits_{i=0}^{N_{base}} P_i^k\psi_i(\vec{x})
\end{equation}
by limiting its coefficients $P_i^k$, starting with the coefficients of 
highest order i.e. $i = N_{base}$ we subsequently replace 
$P_i^k$ with
\begin{equation}
\label{eq:limiter_1D}
    \tilde{P}_i^k = \text{minmod}\left(P_i^k, 
                        \alpha_i(P_{i-1}^{k+1} - P_{i-1}^k), 
                        \alpha_i (P_{i-1}^k - P_{i-1}^{k-1})\right)
\end{equation}
stopping when $P_i^k = \tilde{P}_i^k$. In the definition of limiter \eqref{eq:limiter_1D} 
$\text{minmod}$ is a function of three 
variables
\begin{equation}
    \text{minmod}(a,b,c) = 
    \begin{cases}
            \text{sign}(a)\min(\abs{a}, \abs{b}, \abs{c}) & \;\text{if}\;     
                                                            \text{sign}(a) =
                                                            \text{sign}(b) = 
                                                            \text{sign}(c)\\
                                                        0 & \;\text{otherwise}
    \end{cases}.
\end{equation}
and $\alpha_i$ is limiting coefficient dependent on the order, Krivodonova 
\cite{Krivodonova2007} proposes to take $\alpha_i$ from range
\begin{equation}
    \frac{1}{2(2n -1)} \leq \alpha_n \leq 1.
\end{equation}
Choosing $\alpha_i$ outside this region results in either loss of accuracy or numerical 
instability \cite[p. 882]{Krivodonova2007}. Lower bound of the interval corresponds to 
the strictest limiting, whereas $\alpha_i = 1$ is the mildest limiter possible \cite[p. 
882]{Krivodonova2007}. The one-dimensional limiter is implemented in 
\pysauce{dg.limiters.MomentLimiter1D}, see Section \ref{se:i_moment_lim_1D} for details.

\subsubsection{Two-dimensional limiting}
In this section we describe extension of moment limiter to regular 
tensor-product meshes in two dimensions. We limit solution coefficient in 
individual cells much like in 1D, this time we have to take into account 
derivations in four directions, we also introduce some new notation
\begin{multline}
\tilde{P}^{k,m}_{r,s} =
    \text{minmod}\big(P^{k,m}_{r,s}, 
                      \alpha_s(P^{k,m+1}_{r,s-1} - P^{k,m}_{r,s-1}),
                      \alpha_s(P^{k,m}_{r,s-1} - P^{k,m-1}_{r,s-1}),\\
                      \alpha_r(P^{k+1,m}_{r-1,s} - P^{k+1,m}_{r-1,s}),
                      \alpha_r(P^{k-1,m}_{r-1,s} - P^{k,m}_{r-1,s})\big),
\end{multline}
where $P^{k, m}_{r,s}$ denotes coefficient at basis function $\phi_n$ of the form
$$
\psi_n = L^r(x)L^s(y)
$$
i.e.,
$$
n = indx(r, s).
$$
Further we leveraged the fact that we assume the mesh to be regular uniform 
Cartesian grid and introduced new notation indexing mesh elements as $T^{k,m}$ where $k$ 
ranges over rows and $m$ over columns.
The region of stability for $\alpha_n$ is different due to normalization of 
basis functions
\begin{equation}
\frac{1}{2\sqrt{4n^2 - 1}} \leq \alpha_n \leq \sqrt{\frac{2n - 1}{2n + 1}},
\end{equation}
we choose upper bound as $\alpha$ again. The two-dimensional limiter is implemented in 
class\\ \pysauce{dg.limiters.MomentLimiter2D}, see \Cref{se:i_moment_lim_2D}. 


%$$
%I(x) = \Bigg\lbrace
%\begin{array}{ll}
%1 & \text{pro } c < x < d \\
%0 & \text{jinde}.
%\end{array}
%$$

