% !TeX spellcheck = en_US
%%
%% Text of diploma thesis
%%
%% Tomáš Zítka
%%
\paragraph{Implementation}
In this work, we implemented the discontinuous Galerkin method into \sfepy{} 
package. \sfepy{} uses term based syntax for building discretizations of 
equations, we implemented several new terms, namely linear advection flux 
term, general hyperbolic flux term, diffusion flux and diffusion penalty term, 
and the general term for computing integral
$$
\int_{T^k} \vec{f}(P_i^k\psi_i)\cdot\nabla\psi_j. 
$$
Along with a wide range of terms already present in \sfepy{} this allows users 
to discretize a variety of useful equations. To enable solving transient 
equations we implemented two explicit time-stepping solvers, forward Euler and 
TVD Runge-Kutta of 3rd order solver. Moreover, we implemented the moment 
limiters for 1D and 2D transient problems.

\paragraph{Analysis}
To study properties of the method we measured convergence rates for seven 
example problems chosen from the literature. For some of them, we present 
results which complement already published parametric studies. For transient 
problems, the performance of the method is hindered by used time stepping 
solvers, nevertheless, the method still performs to the expectations.
[\todo more on examples when they are finished]

This fulfills all the major goals of this work. There are, however, still many 
possible improvements and opportunities for future work. Although from the 
time and memory requirements perspective the implementation of the method 
scales well enough with \sfepy{} capabilities, there is still room for 
improvement. Calls of \pysauce{numpy.einsum} could use an optimized tensor 
contraction path, which could be retained between individual term evaluations 
(i.e., between time steps). The implementation of limiters is rather ad-hoc and 
refactoring it to bring it in line with the design of other \sfepy{} elements 
would help to make their code more readable and their usage simpler and more 
versatile. 

One important feature available in \sfepy{} missing from this DG FEM 
implementation is the ability to solve systems of PDEs. This lack could 
inspire future work as it would require substantial modification of flux terms 
as well as the implementation of  strategies for evolving systems of interest 
like Euler or Navier-Stokes equations \cite{Hesthaven2008}. Further besides 
implemented Lax-Friedrichs flux, there is a variety of other fluxes for 
example Godunov flux \cite{DiPietro2012} or fluxes designed specifically for 
solving Euler equations \cite[Section 3.3]{Kucera} which could be added to 
broaden the selection of tools available in \sfepy{}. Thanks to the versatility of 
problem specification this would also unlock large potential for further study of the 
method behavior.