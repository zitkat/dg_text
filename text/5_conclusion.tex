% !TeX spellcheck = en_US
%%
%% Text of diploma thesis
%%
%% Tomáš Zítka
%%
\paragraph{Implementation}
In this work we implemented discontinuous Galerkin method into SfePy package. 
SfePy uses term based syntax for building discretizations of equations, we implemented 
several new terms, namely linear advection flux term, general hyperbolic flux term, 
diffusion flux and diffusion penalty term, and general term for computing integral
$$
\int_{T^k} \vec{f}(P_i^k\psi_i)\cdot\nabla\psi_j. 
$$
Along with wide range of terms already present in SfePy this allows users to 
discretize variety of useful equations. To enable solving transient equations we 
implemented two explicit time stepping solvers, forward Euler and TVD Runge-Kutta of 3rd 
order solver. Moreover we implemented moment limiters for 1D and 2D transient 
problems.

\paragraph{Analysis}
To study properties of the method we measured convergence rates for seven 
example 
problems chosen from literature. For some of them we present results to our 
knowledge not 
available in literature. For transient problems the performance of the method 
is hindered 
by used time stepping solvers, nevertheless the method still performs to the 
expectations.
[\todo more on examples when they are finished]

This fulfills all the major goals of this work. There are, however, still many possible 
improvements and opportunities for future work. Although from the time and memory 
requirements perspective the implementation of the method scales well enough with SfePy 
capabilities, there is still room for improvement. Calls of \pysauce{numpy.einsum} could 
use optimized tensor contraction path, which could be retained between individual term 
evaluation (i.e. between time steps). The implementation of limiters is rather ad-hoc and 
refactoring it to bring it in line with design of other SfePy elements would help making 
code more readable and their usage simpler and more versatile. 

One important feature available in SfePy missing from this DG FEM 
implementation is ability to solve systems of PDEs. This lack could inspire future work 
as it would requires substantial modification of flux terms as well as implementation of  
strategies for evolving systems of interest like Euler or Navier-Stokes 
equations. Further besides implemented Lax-Friedrichs flux there is variety of other 
fluxes for examples Godunov flux or some problem specific fluxes like [\todo cite] 
designed specifically for solving Euler equation Thanks to versatility of problem 
specification this would also unlock large potential for further study of the method 
behavior. 
