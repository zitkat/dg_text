% !TeX spellcheck = en_US
%%
%% Text of diploma thesis
%%
%% Tomáš Zítka
%%
\section{DG method components}
Having laid out the structure of the \sfepy{} problem and objects needed to create
it and work with it we now present classes needed to implement the DG FEM. Following the
architecture of \sfepy{}, the DG FE method implementation comprises of:
\begin{itemize}
    \item \pysauce{DGField},
    \item \pysauce{LegendrePolySpace} and its subclasses,
    \item \pysauce{LegendreTensorProductPolySpace} and
    \item \pysauce{LegendreSimplexPolySpace};
\end{itemize}
DG specific terms are summarized in the top portion of \Cref{tab:terms}.
\begin{landscape}
    \begin{table}[p!]
        \centering
        \caption{Table of terms used in DG method}
        \label{tab:terms}
        \begin{tabular}{lccc}
            \toprule
            Class & Name & Symbol & Expression  \\
            \midrule
            \pysauce{AdvectionDGFluxTerm} &
            \pysauce{"dw_dg_advect_laxfrie_flux"}
            & $a^F_\mathrm{adv}(\mathbf{p})$
            & $\int_{\partial{T^k}} \vec{n} \cdot f^{*} (p_{in},
            p_{out})\cdot\psi_j$ \\
            \pysauce{NonlinearHyperbolicDGFluxTerm} &
            \pysauce{"dw_dg_nonlinear_laxfrie_flux"}
            & $a^F_\mathrm{hyp}(\mathbf{p})$
            & $\int_{\partial{T^k}} \vec{n} \cdot f^{*} (p_{in},
            p_{out})\cdot\psi_j$ \\
            \pysauce{NonlinearScalarDotGradTerm} &
            \pysauce{"dw_ns_dot_grad_s"}
            & $a^C_\mathrm{hyp}(\mathbf{p})$
            & $\int_{T^k} \vec{f}(P_i^k\psi_i)\cdot\nabla\psi_j$ \\
            \pysauce{DiffusionDGFluxTerm} & \pysauce{"dw_dg_diffusion_flux"}
            & \begin{tabular}{c}
                $a^R_\mathrm{diff}(\mathbf{p})$\\
                $a^L_\mathrm{diff}(\mathbf{p})$
            \end{tabular}
            & \begin{tabular}{c}
                $\elbint D \frac{\nabla \psi_j}{2}\cdot \vec{n}[P^k_i\psi_i]$\\
                $\elbint D \manglebracs{P^k_i\nabla\psi_i} \cdot \vec{n}\psi_j$
            \end{tabular}  \\
            \pysauce{DiffusionInteriorPenaltyTerm}&
            \pysauce{"dw_dg_interior_penalty"}
            & $a^P_\mathrm{diff}(\mathbf{p})$ & $\elbint \sigma
            [P^k_i\psi_i]\psi_j$ \\
            \midrule

            \pysauce{ScalarDotMGradScalarTerm} & \pysauce{"dw_s_dot_mgrad_s"}
            & $a^C_\mathrm{adv}(\mathbf{p})$
            & $\int_{T^k} \vec{a}P_i^k\psi_i\cdot\nabla\psi_j$
            \\
            \pysauce{LaplaceTerm} & \pysauce{"dw_laplace"}
            & $a^C_\mathrm{diff}(\mathbf{p})$
            & $\elint D\nabla
            P^k_i\psi_i \nabla \psi_j$ \\
            \pysauce{DotProductVolumeTerm} & \pysauce{"dw_volume_dot"} & -- &
            $\fdiff{P^k_i}{t}(t)\elint\psi_i\psi_j$ \\
            \bottomrule
        \end{tabular}
    \end{table}
\end{landscape}
\noindent DG specific boundary conditions:
\begin{itemize}
    \item \pysauce{DGEssentialBC},
    \item \pysauce{DGPeriodicBC};
\end{itemize}
multistage time-stepping solvers:
\begin{itemize}
    \item abstract base class \pysauce{DGMultiStageTS} and two solvers used in numerical experiments:
    \item \pysauce{EulerStepSolver},
    \item \pysauce{TVDRK3StepSolver}.
\end{itemize}
Finally limiters were implemented as subclasses of \pysauce{DGLimiter} abstract class (which has no
counterpart in \sfepy{}):
\begin{itemize}
    \item \pysauce{IdentityLimiter} -- provided for convenience to enable
    easily disabling limiter without changing
    syntax,
    \item \pysauce{MomentLimiter1D} -- for 1D problems only,
    \item \pysauce{MommentLimiter2D} -- only for 2D problems on regular tensor
    product
    meshes.
\end{itemize}
The limiters are used in the problem composition as post-stage hooks passed to time-stepping solvers. For
technical reasons we also created the \pysauce{DGVariable} class in order to bypass the classical FE
treatment of boundary conditions, otherwise it is similar to the original \sfepy{} \pysauce{Variable}
class and we omit its detailed description.


\section{DG Field}
The \pysauce{DGField} class inherits from the \pysauce{Field} base class. This provides it with the basic
functionality needed to be used in problem specification. From methods implemented in
\pysauce{DGField}, the most relevant to DG FEM are:
\begin{itemize}
    \item \pysauce{get_both_facet_state_vals} -- which returns values of state on opposing sides of
    the boundary for each element
    \item \pysauce{get_both_facet_base_vals} -- which returns values of basis functions on opposing
    sides of the boundary for each element
    \item \pysauce{get_facet_neighbor_idx} -- which returns indices of cell neighbors for individual
    facets along with index of the facet within the neighboring cell
    \item \pysauce{get_bc_facet_values}
    \item \pysauce{get_facet_boundary_idx}
    \item \pysauce{get_facet_vols}
    \item \pysauce{get_facet_qp}
    \item \pysauce{get_nodal_values}
\end{itemize}

\subsection{Legendre polynomial spaces implementation}
Legendre polynomial spaces are implemented in two classes
\pysauce{LegendreTensorProductPolySpace} and
\pysauce{LegendreSimplexPolySpace}. Both are derived from the abstract class
\pysauce{LegendrePolySpace} which inherits from \sfepy{}
\pysauce{PolySpace}. It implements the method \pysauce{_eval_base} which is used to get values of basis
functions as well as their derivatives. It also contains methods for evaluating Legendre and Jacobi
polynomials common to tensor-product and simplex subclasses. These classes are accompanied by the
function \pysauce{get_n_el_nod}, which returns number of basis functions for the given order, dimension
and type of basis, and the generator \pysauce{iter_by_order} (\ref{lst:iter_by_order}) which generates
tuples of $r$ and $s$ in desired hierarchical order. For example, for the approximation order $2$ and the
tensor-product basis this is:
\pysauce{(0, 0),
    (0, 1),
    (1, 0),
    (0, 2),
    (1, 1),
    (2, 0),
    (2, 1),
    (1, 2),
    (2, 2)}.
\setcounter{lstannotation}{0}
\begin{lstlisting}[language=Python, caption= Iteration over $r$ and $s$
indicies of basis functions \label{lst:iter_by_order}]
for k in range(porder):
  for r in range(k + 1):
    yield r, k - r /*!\lann{lsta:yield}!*/
  if not extended: return /*!\lann{lsta:extended}!*/
  for s in range(1, porder):
    for r in range(1, porder):
      if r + s <= porder - 1:
        continue
      yield r, s
\end{lstlisting}
\begin{itemize}
    \item[\ref{lsta:yield}] \pysauce{yield} keyword turns a function into a generator usable in for
    cycles, for example in Listing \ref{lst:limiter_2D}.
    \item[\ref{lsta:extended}] \pysauce{extended} flag distinguishes the simplex basis from
    tensor-product one which uses more basis functions.
\end{itemize}
To obtain values of Jacobi polynomials, we used implementations provided by SciPy in the
\pysauce{special} module.


\section{DG Terms}
(\todo \pysauce{function} method) is called whenever value of the term is needed either
to build residual vector or right-hand side of an equation or to get terms contribution
to (\todo the matrix) (in case of implicit problems). The method returns values for
individual DOFs and in matrix mode also indices to build sparse matrix.

\subsection{Hyperbolic flux term implementation}
\label{se:adv_flux_term_imp}
\pysauce{AdvectionDGFluxTerm} corresponds to the discretized term \eqref{eq:hyp_flux_app}
where $\vec{f}(p) = \vec{a}p$. The part of the \pysauce{function} capturing computation of
cell fluxes can be seen in Listing \ref{lst:adv_flux} below.
\setcounter{lstannotation}{0}
\begin{lstlisting}[language=Python, caption=Computation of advection cell
fluxes \label{lst:adv_flux}]
fc_n = field.get_cell_normals_per_facet(region)
# get maximal wave speeds at facets
C = nm.abs(nm.einsum("ifk,ik->if", fc_n, advelo))

facet_base_vals = field.get_facet_base(base_only=True)
in_fc_v, out_fc_v, weights = field.get_both_facet_state_vals(state,
                                                             region)
# reshape facet base
fc_b = facet_base_vals[:, 0, :, 0, :].T
# (n_el_nod, n_el_facet, n_qp)

fc_v_avg = (in_fc_v + out_fc_v)/2.
fc_v_jmp = in_fc_v - out_fc_v

central = nm.einsum("ik,ifq->ifkq", advelo, fc_v_avg) /*!\lann{lsta:aflx_centr}!*/
upwind = (1 - self.alpha)/2. * nm.einsum("if,ifk,ifq->ifkq",
                                         C, fc_n, fc_v_jmp)

cell_fluxes = nm.einsum("ifk,ifkq,dfq,ifq->id",
                        fc_n, central + upwind, fc_b, weights)
\end{lstlisting}
\begin{itemize}
    \item[\ref{lsta:aflx_centr}] \pysauce{numpy.einsum} uses the Einstein summation convention for
    expressing tensor contractions, for details see \cite{einsum-doc}.
\end{itemize}
The general hyperbolic term is implemented in the class
\pysauce{NonlinearHyperbolicDGFluxTerm}.



\subsection{Diffusion flux term implementation}
\label{se:diff_flux_term_imp}
\pysauce{DiffusionDGFluxTerm} implements both terms in
\eqref{eq:diff_left_approx} and \eqref{eq:diff_right_approx}. This is thanks to two modes
in which it can be used in an equation --- this has already been demonstrated for the Laplace
equation in Listing \ref{lst:laplace} where \pysauce{"dw_dg_diffusion_flux.i.Omega(D.val,
p, v)"} corresponds to $a^R_\mathrm{diff}(\mathbf{p})$ and mode \pysauce{'avg_state'}
(\ref{lsta:avg_state}), and \pysauce{"dw_dg_diffusion_flux.i.Omega(D.val, v, p)"}
corresponds to $a^L_\mathrm{diff}(\mathbf{p})$ and mode \pysauce{'avg_virtual'}
(\ref{lsta:avg_virtual}).
\setcounter{lstannotation}{0}
\begin{lstlisting}[language=Python, caption=Computation of diffusion cell
fluxes]
if self.mode == 'avg_state': /*!\lann{lsta:avg_state}!*/
  avgDdState = (nm.einsum("ikl,ifkq->ifkq",
                          D, inner_facet_state_d) +
                nm.einsum("ikl,ifkq->ifkq",
                          D, outer_facet_state_d)) / 2.
  # outer_facet_base is in DG zero
  # hence the jump is inner value
  jmpBase = inner_facet_base

  cell_fluxes = nm.einsum("ifkq ,ifk,idfq,ifq->id",
                          avgDdState, fc_n, jmpBase, weights)

elif self.mode == 'avg_virtual': /*!\lann{lsta:avg_virtual}!*/
  avgDdbase = (nm.einsum("ikl,idfkq->idfkq",
                         D, inner_facet_base_d)) / 2.

  jmpState = inner_facet_state - outer_facet_state
  cell_fluxes = nm.einsum("idfkq, ifk, ifq , ifq -> id",
                          avgDdbase, fc_n, jmpState, weights)
\end{lstlisting}


\subsection{Difusion penalty term implementation}
\label{se:diff_penal_term_imp}
\setcounter{lstannotation}{0}
\begin{lstlisting}[language=Python, caption=Computation of penalty cell
fluxes]
approx_order = field.approx_order

inner_facet_base, outer_facet_base, whs = \
    field.get_both_facet_base_vals(state, region, derivative=False)
facet_vols = nm.sum(whs, axis=-1)

# nu characterizes diffusion tensor, so far we user diagonal average
nu = nm.trace(diff_tensor, axis1=-2, axis2=-1)[..., None] / \
                        diff_tensor.shape[1]
sigma = nu * Cw * approx_order ** 2 / facet_vols

inner_facet_state, outer_facet_state, whs = \
    field.get_both_facet_state_vals(state, region,
                                    derivative=False)

inner_facet_base, outer_facet_base, _ = \
    field.get_both_facet_base_vals(state, region,
                                   derivative=False)

jmp_state = inner_facet_state - outer_facet_state
jmp_base = inner_facet_base  # - outer_facet_base

n_el_nod = nm.shape(inner_facet_base)[1]
cell_penalty = nm.einsum("nf,nfq,ndfq,nfq->nd",
                         sigma, jmp_state, jmp_base, whs)

\end{lstlisting}

\section{Limiters implementation}
Following design patterns used in \sfepy{} and Python in general, the limiters are
implemented as classes. The base class providing only the constructor is called \pysauce{DGLimiter}, its
subclasses then implement the abstract method \pysauce{__call__} --- this makes all limiters callable
objects, allowing one to pass them as post-step or post-stage or other hooks to time-stepping
solvers. For convenience the neutral limiter is implemented in \pysauce{IdentityLimiter}.

\subsubsection{Moment limiter -- 1D}
\label{se:i_moment_lim_1D}
The code listing below shows the implementation of the moment limiter introduced in Section~\ref{sse:moment_lim_1D},
omitting some details for brevity.
\setcounter{lstannotation}{0}
\begin{lstlisting}[language=Python, caption=Moment limiter for 1D]
idx = nm.arange(nm.shape(u[0, 1:-1])[0])

nu = nm.copy(u)
tilu = nm.zeros(u.shape[1:])
for ll in range(self.n_el_nod - 1, 0, -1):
  tilu[idx] = minmod(nu[ll, 1:-1][idx],
                     nu[ll-1, 2:][idx] - nu[ll-1, 1:-1][idx],
                     nu[ll-1, 1:-1][idx] - nu[ll-1, :-2][idx]) /*!\lann{lsta:lim1}!*/

  idx = idx[nm.where(abs(tilu[idx] - nu[ll, 1:-1][idx])
            > MACHINE_EPS)[0]] /*!\lann{lsta:lim2}!*/
  if len(idx) == 0:
    break /*!\lann{lsta:lim3}!*/
  nu[ll, 1:-1][idx] = tilu[idx] /*!\lann{lsta:lim4}!*/
\end{lstlisting}
\begin{itemize}
    \item [\ref{lsta:lim1}] Compute the limiting value $\tilde{u}$.
    \item [\ref{lsta:lim2}] Extract indicies where the limiting value is
    larger than the
    current solution.
    \item[\ref{lsta:lim3}] If none of the coefficients requires limiting we
    stop.
    \item [\ref{lsta:lim3}] Replace old values with limited ones.

\end{itemize}

\subsubsection{Moment limiter -- 2D}
\label{se:i_moment_lim_2D}
We list the implementation of the 2D limiter for reference in Listing \ref{lst:limiter_2D}. The Limiter is
implemented according to \Cref{se:limiters}.
\setcounter{lstannotation}{0}
\begin{lstlisting}[language=Python, caption=Moment limiter for
cartesian grid \label{lst:limiter_2D}]
for ll, (ii, jj) in enumerate(
                     iter_by_order(self.field.approx_order,
                                   2,   # dim
                                   extended=ex)):
  nu[ii, jj, ...] = u[ll] /*!\lann{lsta:indx_to}!*/

for ii, jj in reversed(list(
                        iter_by_order(
                            self.field.approx_order, 2,
                            extended=ex))):
  minmod_args = [nu[ii, jj, idx]]
  nbrhs = nbrhd_idx[idx]
  if ii - 1 >= 0:
    alf = nm.sqrt((2 * ii-1) / (2 * ii + 1))
    # right difference in x axis
    dx_r = alf*(nu[ii-1, jj, nbrhs[:, 1]] - nu[ii-1, jj, idx])
    # left differnce in x axis
    dx_l = alf*(nu[ii-1, jj, idx] - nu[ii-1, jj, nbrhs[:, 3]])
    minmod_args += [dx_r, dx_l]
  if jj - 1 >= 0:
    alf = nm.sqrt((2 * jj - 1) / (2 * jj + 1))
    # right i.e. element "up" difference in y axis
    dy_up = alf*(nu[ii, jj-1, nbrhs[:, 2]] - nu[ii, jj-1,  idx])
    # left i.e. element "down" difference in y axis
    dy_dn = alf*(nu[ii, jj-1,  idx] - nu[ii, jj-1,  nbrhs[:, 0]])
    minmod_args += [dy_up, dy_dn]

  tilu[idx] = minmod_seq(minmod_args)
  idx = idx[nm.where(abs(tilu[idx] - nu[ii, jj, idx]) > MACHINE_EPS)[0]]

  if len(idx) == 0:
    break
  nu[ii, jj, idx] = tilu[idx]

resu = nm.zeros(u.shape)
for ll, (ii, jj) in enumerate(
                     iter_by_order(self.field.approx_order,
                                   2,   # dim
                                   extended=ex)):
  resu[ll] = nu[ii, jj] /*!\lann{lsta:indx_from}!*/
\end{lstlisting}
\begin{itemize}
    \item[\ref{lsta:indx_to}] Reshape the solution array for indexing using $r$ and $s$ indicies,
    effectively removing need for the explicit inverse of index mapping from \eqref{eq:bindx}.
    \item [\ref{lsta:indx_from}] Convert back to the linear index.
\end{itemize}


\section{Time-stepping solvers implementation}
As demonstrated in \Cref{se:time_theory}, the explicit DG FEM requires explicit time stepping solvers
with multiple stages in one time step. These had not been part of the rich collection of time-stepping
solvers included in \sfepy{}, so two new solvers were implemented: the basic Euler solver, the total-variations
diminishing Runge-Kutta of the 3rd order (TVD RK-3). Again following the structure of \sfepy{}, they are
implemented as subclasses of \pysauce{TimeSteppingSolver}. The abstract class
\pysauce{DGMultiStageTS} extends the basic \pysauce{TimeSteppingSolver} with the option to provide
pre-stage and post-stage hooks, allowing to apply limiters between stages. The two time-stepping
solvers are then implemented in classes \pysauce{EulerStepSolver} and \pysauce{TVDRK3StepSolver}.
