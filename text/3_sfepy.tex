% !TeX spellcheck = en_US
%%
%% Text of diploma thesis
%%
%% Tomáš Zítka
%%
\section{Problem specification}
Before we delve into inner workings of Sfepy numerical code lets introduce the so-called 
declarative problem specification format. The format relies on Python dictionaries, see 
Listing \ref{lst:laplace} for model problem specification for 2D Laplace equation on 
domain $[0, 1] \times [0, 1]$ simplified from \Cref{ex:laplace}. It 
contains dictionaries declaring components of the problem like regions in geometric 
domain, field governing the used FE method, state and test variables, boundary 
conditions, material constants and functions etc. Detailed and more general treatment of 
the format can be found in \cite{Cimrman_Lukes_Rohan_2019} here we focus on specification 
of a problem to be solved using DG FE.

\setcounter{lstannotation}{0}
\begin{lstlisting}[language=Python, caption=\pysauce{example\_dg\_laplace.py}, 
label={lst:laplace}]
regions = {'Omega'     : 'all',  /*!\annotation{lsta:laplace_reg}!*/
		   'left' : ('vertices in x == 0', 'edge'),
		   'right': ('vertices in x == 1', 'edge'),
		   'top' : ('vertices in y == 1', 'edge'),
		   'bottom': ('vertices in y == 0', 'edge')}
fields = { /*!\annotation{lsta:laplace_fields}!*/
  'f': ('real', 'scalar', 'Omega', 
        str(approx_order) + 'd', 'DG', 'legendre')}

variables = {'p': ('unknown field', 'f', 0, 1),
			 'v': ('test field', 'f', 'p')} /*!\annotation{lsta:laplace_vars}!*/

def analytic_sol(coors): 
	x_1, x_2 = coors[..., 0], coors[..., 1]
	res = 1/2*x_1**2 - 1/2*x_2**2 - a*x_1 + b*x_2 + c
	return res

@local_register_function
def bcs(ts, coors, bc, problem): /*!\annotation{lsta:laplace_bcf}!*/
	x_1, x_2 = coors[..., 0], coors[..., 1]
	res = nm.zeros(x_1.shape)
	if bc.diff == 0:
		res[:] = analytic_sol(coors)
	elif bc.diff == 1:
		res = nm.stack((x_1 - a, -x_2 + b), axis=-2)
	return res

dgebcs = { /*!\annotation{lsta:laplace_bcs}!*/
  'p_left' : ('left', {'p.all': "bcs", 'grad.p.all': "bcs"}),
  'p_right' : ('right', {'p.all': "bcs", 'grad.p.all': "bcs"}),
  'p_bottom' : ('bottom', {'p.all': "bcs", 'grad.p.all': "bcs"}),
  'p_top' : ('top', {'p.all': "bcs", 'grad.p.all': "bcs"})}

materials = {'D': ({'val': [diffcoef], '.Cw': cw},)} /*!\annotation{lsta:laplace_mat}!*/
integrals = {'i': 2 * approx_order}

equations = {'the_equation':    /*!\annotation{lsta:laplace_eq}!*/
		"dw_laplace.i.Omega(D.val, v, p) " 
	    " - dw_dg_diffusion_flux.i.Omega(D.val, p, v)" 
	    " - dw_dg_diffusion_flux.i.Omega(D.val, v, p)" 
	    " + dw_dg_interior_penalty.i.Omega(D.val, D.Cw, v, p)"
	    "= 0"}

solvers = {'ls': ('ls.auto_direct', {}),   /*!\annotation{lsta:laplace_solv}!*/
    	   'newton': ('nls.newton', {})}
options = {'nls'             : 'newton',   /*!\annotation{lsta:laplace_opts}!*/
	       'ls'              : 'ls',
		   'output_format'   : 'msh'
		   'format_variant'  : 'gmsh-dg'}
\end{lstlisting}


\begin{itemize}
	\item[\ref{lsta:laplace_reg}] \pysauce{regions} dictionary specifies different 
	regions 
	used in	boundary conditions specification, \pysauce{Omega} region is required for 
	setting up fields, \pysauce{'edge'} regions are needed for BCs.
	\item[\ref{lsta:laplace_fields}]  \pysauce{fields} determine discretization spaces of 
	variables and are defined using tuple of strings.
	\shellcmd{(data type, number of components, region name, approximation order, field 
	type, polyspace)} here field \pysauce{'f'} is field for discretization of real scalar 
	variable in region \pysauce{"Omega"} using discontinuous Galerkin method of order 
	\pysauce{approx_order} in space of Legendre polynomials. 
	\item[\ref{lsta:laplace_vars}]  Here \pysauce{'p'} is unknown state variable we are 
	solving for and \pysauce{'v'} is test variable, they are both discretized using field 
	\pysauce{'f'} defined above.
	\item[\ref{lsta:laplace_bcf}]  Function \pysauce{bcs} is called during solution, it 
	is supposed to produce values of boundary conditions.
	\item[\ref{lsta:laplace_bcf}] \pysauce{dgebcs} dictionary sets up boundary 
	conditions specifically for DG FE methods, it creates map between boundary regions 
	and, variables and values of functions that determine boundary conditions.
	\item[\ref{lsta:laplace_mat}] In materials dictionary we specify diffusion 
	coefficient $D$, the dot notation \pysauce{'.Cw'} 
	causes material not to be broad-casted to quadrature points, which is convenient for 
	constants parameterizing terms like $C_w$ or $\alpha$ in 
	\eqref{eq:diff_penalty_sigma} 
	resp. \eqref{eq:lax-frieflux}.
	\item[\ref{lsta:laplace_eq}] Equation to solve is composed from terms derived in 
	Chapter \ref{ch:theory}.
	
	\item[\ref{lsta:laplace_solv}] Linear and non-linear solvers to use, Sfepy supports 
	various solvers including \pysauce{mumps} \cite{MUMPS:2}
	
	\item[\ref{lsta:laplace_opts}] Various options, \pysauce{'output_format' : 'msh'} and 
	\pysauce{'format_variant' : 'gmsh'} ensure output in format suitable for 
	post-processing using \pysauce{Gmsh} (\url{http://gmsh.info/}, \cite{Remacle2007}).
\end{itemize}
All the test problems studied in Chapter \ref{ch:convergence} are specified using this 
declarative approach, codes can be found in (\todo Attachment? 
\url{https://github.com/zitkat/dg_examples})

\section{SfePy architecture}
Components in the problem specification file are parsed into various Python objects  
brought together in the \pysauce{Problem} object, the most important are:
\begin{itemize}
	\item \pysauce{Equation} -- representing the equation to be solved,
	\item \pysauce{EssentialBC} -- representing Dirichlet boundary conditions,
	\item \pysauce{PeriodicBC} -- representing periodic boundary conditions,
	\item \pysauce{InitialCondition} -- representing the initial condition, in case of transient problems,
	\item \pysauce{TimeSteppingSolver} -- specifying time discretization scheme, in case of transient problems.
\end{itemize}
The \pysauce{Equation} object is build by combining \pysauce{Term} objects these 
represent individual integral terms which are evaluated in course of solving a problem.
Due to interpreted nature of CPython (\url{https://github.com/python/cpython}) in which 
Sfepy is mainly run and which is generally too slow for high-performance numerical 
computation due to overhead from the interpreter, Sfepy relies on various approaches to 
speed up the computation, in general it uses fast vectorized operations provided by NumPy 
and Scipy \cite{SciPy-NMeth2020}. C and Cython are used in places where vectorization is 
not possible, or is too difficult or unreadable \cite{Cimrman_Lukes_Rohan_2019}. 
Our implementation relies on NumPy vectorization, especially \pysauce{einsum} 
function (mode details bellow). Terms keep references to other objects:
\begin{itemize}
	\item \pysauce{Variable} -- representing state and test variables,
	\item \pysauce{Material} -- representing various material constants or functions,
	\item \pysauce{Integral} -- representing gauss quadrature rule.
\end{itemize}
The \pysauce{Variable} object in turn contains reference to \pysauce{Field} object that 
manages the chosen discretization -- finite dimensional space represented by 
\pysauce{PolySpace} object and provides method needed to work with it along with the 
computational domain (\pysauce{Domain} object) which stores geometry including
\pysauce{Region} objects used in definition of \pysauce{EssentialBC} and 
\pysauce{PeriodicBC}.










