\section{Example PDEs}
In the following examples we will busy ourselves solving following equations:\\
Transient advection equation in one resp. two dimensions
\begin{equation}
	\label{eq:ex_advection}
	\pdiff{p}{t} + a\pdiff{p}{x} = 0.
\end{equation}
resp.
\begin{equation}
	\pdiff{p}{t} + \vec{a}\cdot\nabla{p}{x} = 0
\end{equation}
in Sfepy declarative notation these equations have the same form. See Listing 
\ref{lst:advection}
\setcounter{lstannotation}{0}
\begin{lstlisting}[language=Python, caption={Advection equation}
\label{lst:advection}]
equations = {'Advection': 
  # transient
  "dw_volume_dot.i.Omega(v, p)"
 
  "- dw_s_dot_mgrad_s.i.Omega(a.val, p[-1], v)" /*!\lann{lsta:history}!*/
  "+ dw_dg_advect_laxfrie_flux.i.Omega(a.flux, a.val, v, p[-1])"  /*!\lann{lsta:flux}!*/
  "= 0"
  }
\end{lstlisting}
\begin{itemize}
	\item[\ref{lsta:history}] \pysauce{"p[-1]"} ensures the variable object stores 
	history one step backwards in time facilitating forward nature of time stepping 
	solvers
	\item[\ref{lsta:flux}] \pysauce{"a.flux"} is an optional material argument 
	representing $\alpha$ coefficient in \eqref{eq:lax-frieflux}
\end{itemize}
Laplace equation in two dimension
\begin{equation}
	\label{eq:ex_laplace}
	\pdiff{^2 p}{x^2} + \pdiff{^2 p}{y^2} = 0
\end{equation}
discretization of this equation using Sfepy terms can be found in Listing 
\ref{lst:laplace};\\
Static advection-diffusion equation with right hand side of the form
\begin{equation}
\label{eq:ex_advdiff}
\pdiff{p}{x} + \pdiff{p}{y} - D \cdot \left( \pdiff{^2 p}{x^2} + \pdiff{^2 
p}{y^2} \right) = g
\end{equation}
i.e.,
\begin{equation}
\vec{a} \cdot \nabla p - D \Delta p = g,
\end{equation}
where $\vec{a} = [1, 1]^T$ is advection velocity, $D$ is diffusion coefficient and $g$ is 
a source function. Sfepy discretization using symmetric scheme and penalty term can be 
found in Listing 
\ref{lst:advdiff}.
\setcounter{lstannotation}{0}
\begin{lstlisting}[language=Python, caption=Static advection-diffusion equation
\label{lst:advdiff}]
equations = {'adv_diff' :
	# advection
	"- dw_s_dot_mgrad_s.i.Omega(a.val, p, v)"
	"+ dw_dg_advect_laxfrie_flux.i.Omega(a.flux, a.val, v, p)"
	# diffusion
	"+ dw_laplace.i.Omega(D.val, v, p) "
	"+ dw_dg_diffusion_flux.i.Omega(D.val, p, v)"
	"+ dw_dg_diffusion_flux.i.Omega(D.val, v, p)"
	# penalty
	"+ dw_dg_interior_penalty.i.Omega(D.val, D.cw, v, p)"
	# source
	"- dw_volume_lvf.i.Omega(g.val, v)"
	"= 0"
	}
\end{lstlisting}
Transient viscous Burgers equation in one resp. two dimensions
\begin{equation}
\pdiff{p}{t} + \frac{1}{2}\pdiff{p^2}{x} - D \cdot\pdiff{^2 p}{x^2} = g,
\end{equation}
resp.
\begin{equation}
\label{eq:ex_burgers}
	\pdiff{p}{t} + \frac{1}{2}\left(\pdiff{p^2}{x} + \pdiff{p^2}{y}\right)  - 
	D \cdot \left( \pdiff{^2 p}{x^2} + \pdiff{^2 p}{y^2} \right) 
	= g,
\end{equation}
i.e.,
\begin{equation}
	\pdiff{p}{t} + \nabla \cdot \vec{f}(p) - D\Delta p = g.
\end{equation}
with $\vec{f}(p) = \frac{1}{2}[p^2, p^2]^T = \frac{1}{2}\vec{a} p^2$. In SfePy 
declarative notation these equations have again the same form represented in Listing 
\ref{lst:burgers}.
\setcounter{lstannotation}{0}
\begin{lstlisting}[language=Python, caption=Viscous Burgers equation \label{lst:burgers}]
burg_velo = nm.array([1., 1.])

def bf(p):
	return .5*burg_velo * p[..., None] ** 2

def bf_d(p):
	return burg_velo * p[..., None]

equations = {'Burgers':
	# transient
	"dw_volume_dot.i.Omega(v, p)"
	#  non-linear hyperbolic terms
	"- dw_ns_dot_grad_s.i.Omega(bf, bf_d, p[-1], v)" /*!\lann{lsta:nonlin}!*/
	"+ dw_dg_nonlinear_laxfrie_flux.i.Omega(bf, bf_d, v, p[-1])"
	#  diffusion
	"+ dw_laplace.i.Omega(D.val, v, p[-1])"
	"+ dw_dg_diffusion_flux.i.Omega(D.val, p, v)"
	"+ dw_dg_diffusion_flux.i.Omega(D.val, v, p)"
	# penalty
	"+ dw_dg_interior_penalty.i.Omega(D.val, D.Cw, v, p[-1])"
	# source
	"- dw_volume_lvf.i.Omega(g.val, v)"
	" = 0"
	}
\end{lstlisting}
\begin{itemize}
	\item[\ref{lsta:nonlin}] Nonlinear terms require as parameters function and its 
	derivative.
\end{itemize}