% !TeX spellcheck = en_US
%%
%% Text of diploma thesis
%%
%% Tomáš Zítka
%%
\section{Example PDEs}
In the following examples we will busy ourselves solving following equations:\\
\paragraph{Transient advection equation in one resp. two dimensions}
\begin{equation}
    \label{eq:ex_advection}
    \pdiff{p}{t} + a\pdiff{p}{x} = 0,
\end{equation}
resp.
\begin{equation}
    \pdiff{p}{t} + \vec{a}\cdot\nabla{p} = 0.
\end{equation}
After applying discretizations devised in \Cref{ch:theory} we obtain both
equations in the same form
\begin{equation}
         \fdiff{P^k_i}{t}(t)\elint\psi_i\psi_j
         - \int_{T^k} \vec{a}P_i^k\psi_i\cdot\nabla\psi_j
         + \int_{\partial{T^k}} \vec{n}
        \cdot f^{*} (p_{in}, p_{out})\cdot\psi_j = 0.
\end{equation}
The Equation expressed in \sfepy{} declarative notation can be found in Listing
\ref{lst:advection} below;
\setcounter{lstannotation}{0}
\begin{lstlisting}[language=Python, caption={Advection equation}
\label{lst:advection}]
equations = {'Advection':
  # transient
  "dw_volume_dot.i.Omega(v, p)"

  "- dw_s_dot_mgrad_s.i.Omega(a.val, p[-1], v)" /*!\lann{lsta:history}!*/
  "+ dw_dg_advect_laxfrie_flux.i.Omega(a.flux, a.val, v, p[-1])"  /*!\lann{lsta:flux}!*/
  "= 0"
  }
\end{lstlisting}
\begin{itemize}
    \item[\ref{lsta:history}] \pysauce{"p[-1]"} ensures the variable object
    stores history one step backwards in time facilitating forward nature of
    time stepping solvers,
    \item[\ref{lsta:flux}] \pysauce{"a.flux"} is an optional material argument
    representing $\alpha$ coefficient in \eqref{eq:lax-frieflux}.
\end{itemize}
\paragraph{Laplace equation in two dimensions}
\begin{equation}
    \label{eq:ex_laplace}
    - D \left(\pdiff{^2 p}{x^2} + \pdiff{^2 p}{y^2} \right)= 0.
\end{equation}
Employing the symmetric discretization of the diffusion term and adding the diffusion
penalty term yields the equation in the form
\begin{multline}
    \elint D\nabla P^k_i\psi_i \nabla \psi_j
    - \elbint D \manglebracs{P^k_i\nabla\psi_i} \cdot \vec{n}\psi_j
    - \elbint D \frac{\nabla \psi_j}{2}\cdot \vec{n} [P^k_i\psi_i] \\
    + \nu \elbint \sigma [P^k_i\psi_i]\psi_j
    = 0.
\end{multline}
The discretization of this equation using \sfepy{} terms can be found in Listing
\ref{lst:laplace};\\
\paragraph{Static advection-diffusion equation with a right hand side}
\begin{equation}
\label{eq:ex_advdiff}
\pdiff{p}{x} + \pdiff{p}{y} - D \cdot \left( \pdiff{^2 p}{x^2} + \pdiff{^2
p}{y^2} \right) = g,
\end{equation}
i.e.,
\begin{equation}
\vec{a} \cdot \nabla p - D \Delta p = g,
\end{equation}
where $\vec{a} = [1, 1]^T$ is the advection velocity, $D$ is the diffusion coefficient
and $g$ is a source function. Combining discretizations of the two previous
equations we obtain the discretized form
\begin{multline}
- \int_{T^k} \vec{a}P_i^k\psi_i\cdot\nabla\psi_j
+ \int_{\partial{T^k}} \vec{n}\cdot f^{*} (p_{in}, p_{out})\cdot\psi_j\\
+ \elint D\nabla P^k_i\psi_i \nabla \psi_j
- \elbint D \manglebracs{P^k_i\nabla\psi_i} \cdot \vec{n}\psi_j
- \elbint D \frac{\nabla \psi_j}{2}\cdot \vec{n} [P^k_i\psi_i] \\
+ \nu \elbint \sigma [P^k_i\psi_i]\psi_j
- \elint g\cdot \psi_j
= 0.
\end{multline}
In \sfepy{} declarative notation this equation has the form presented in
Listing \ref{lst:advdiff}.
\setcounter{lstannotation}{0}
\begin{lstlisting}[language=Python, caption=Static advection-diffusion equation
\label{lst:advdiff}]
equations = {'adv_diff' :
    # advection
    "- dw_s_dot_mgrad_s.i.Omega(a.val, p, v)"
    "+ dw_dg_advect_laxfrie_flux.i.Omega(a.flux, a.val, v, p)"
    # diffusion
    "+ dw_laplace.i.Omega(D.val, v, p) "
    "- dw_dg_diffusion_flux.i.Omega(D.val, p, v)"
    "- dw_dg_diffusion_flux.i.Omega(D.val, v, p)"
    # penalty
    "+ dw_dg_interior_penalty.i.Omega(D.val, D.cw, v, p)"
    # source
    "- dw_volume_lvf.i.Omega(g.val, v)"
    "= 0"
    }
\end{lstlisting}
\paragraph{Transient viscous Burgers equation in one resp. two dimensions}
\begin{equation}
\pdiff{p}{t} + \frac{1}{2}\pdiff{p^2}{x} - D \cdot\pdiff{^2 p}{x^2} = g,
\end{equation}
resp.
\begin{equation}
\label{eq:ex_burgers}
    \pdiff{p}{t} + \frac{1}{2}\left(\pdiff{p^2}{x} + \pdiff{p^2}{y}\right)  -
    D \cdot \left( \pdiff{^2 p}{x^2} + \pdiff{^2 p}{y^2} \right)
    = g,
\end{equation}
i.e.,
\begin{equation}
    \pdiff{p}{t} + \nabla \cdot \vec{f}(p) - D\Delta p = g.
\end{equation}
with $\vec{f}(p) = \frac{1}{2}[p^2, p^2]^T = \frac{1}{2}\vec{a} p^2$.
Discretizing using all the terms derived before we get the same form for both 1D and 2D
\begin{multline}
    \fdiff{P^k_i}{t}(t)\elint\psi_i\psi_j
    - \int_{T^k} \vec{f}(P_i^k\psi_i)\cdot\nabla\psi_j
    + \int_{\partial{T^k}} \vec{n} \cdot f^{*} (p_{in}, p_{out})\cdot\psi_j\\
    + \elint D\nabla P^k_i\psi_i \nabla \psi_j
    - \elbint D \manglebracs{P^k_i\nabla\psi_i} \cdot \vec{n}\psi_j
    - \elbint D \frac{\nabla \psi_j}{2}\cdot \vec{n} [P^k_i\psi_i] \\
    + \nu \elbint \sigma [P^k_i\psi_i]\psi_j
    - \elint g\cdot \psi_j
    = 0.
\end{multline}
In \sfepy{} declarative notation this equation has the form presented in
Listing \ref{lst:burgers}.
\setcounter{lstannotation}{0}
\begin{lstlisting}[language=Python, caption=Viscous Burgers equation \label{lst:burgers}]
burg_velo = nm.array([1., 1.])

def f(p):
    return .5*burg_velo * p[..., None] ** 2

def f_d(p):
    return burg_velo * p[..., None]

equations = {'burgers':
    # transient
    "dw_volume_dot.i.Omega(v, p)"
    #  non-linear hyperbolic terms
    "- dw_ns_dot_grad_s.i.Omega(f, f_d, p[-1], v)" /*!\lann{lsta:nl1}!*/
    "+ dw_dg_nonlinear_laxfrie_flux.i.Omega(f, f_d, v, p[-1])" /*!\lann{lsta:nl2}!*/
    #  diffusion
    "+ dw_laplace.i.Omega(D.val, v, p[-1])"
    "- dw_dg_diffusion_flux.i.Omega(D.val, p[-1], v)"
    "- dw_dg_diffusion_flux.i.Omega(D.val, v, p[-1])"
    # penalty
    "+ dw_dg_interior_penalty.i.Omega(D.val, D.Cw, v, p[-1])"
    # source
    "- dw_volume_lvf.i.Omega(g.val, v)"
    " = 0"
    }
\end{lstlisting}
\begin{itemize}
    \item[\ref{lsta:nl1}, \ref{lsta:nl2}] Nonlinear terms require as parameters the function
    and its derivative.
\end{itemize}


\section{Examples}

\paragraph{Measuring convergence} We define convergence rate $r$ as is common
in literature
\begin{equation}
%    num_order = nm.log(row["diff_l2"] / last_err) \
%    / nm.log(row["h"] ** dim / last_h ** dim)
r = \frac{\log\left(\dfrac{\Ltwonorm{p - p_{h_{n}}}}{\Ltwonorm{p - p_{h_{n-1}}}}\right)}
{\log\left(\dfrac{h^d_{n}}{h^d_{n-1}}\right)}.
\end{equation}
Inspired by \cite{Kucera} we present plots depicting average convergence rate
over several mesh refinements, this might not be an ideal measure of the method
behavior, nevertheless it still provides us with a convenient indicator.
Accompanied with plots of $L^2$ error it allows us to reason about the method over
several varying parameters, notably it reveals the relationship between the diffusion
coefficient and the penalty term in examples including the diffusion terms.

In \Cref{ex:adv1D} we explore behavior of DG FE method for the 1D pure advection,
the time dependent problem with and without limiting, in \Cref{ex:adv2D_kriv} we do the same
for the 2D problem although this time we omit the convergence study in favor of exploring
effectiveness of different approximation orders while keeping the number of DOFs
constant. Examples \ref{ex:laplace}, \ref{ex:quart1}, \ref{ex:quart2} and \ref{ex:quart3}
demonstrate importance of diffusion penalty terms. Final two examples
\ref{ex:burgers_hest} and
\ref{ex:kucera} show usage of the method on the Burgers equation. All the test problems
studied further are specified using this declarative approach introduced in
\ref{ch:convergence}, codes can be found in (\todo Attachment?
\url{https://github.com/zitkat/dg_examples})
