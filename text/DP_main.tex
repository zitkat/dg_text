% !TeX spellcheck = en_US
%%
%% Text of diploma thesis
%%
%% Tomáš Zítka
%%
\documentclass{book}
\usepackage[english]{babel} 
\usepackage{a4wide}
\usepackage{graphicx}
\usepackage{caption}
\usepackage{subcaption}
\usepackage[utf8]{inputenc}
\usepackage{enumerate}
\usepackage{amssymb, amsmath}
\usepackage{amsthm}
\usepackage{natbib}
\usepackage{ae}
\usepackage{pifont}
\usepackage[all]{xy}
\usepackage{thmtools}
\renewcommand{\listtheoremname}{List of Examples}

% colors
\usepackage[usenames,dvipsnames]{xcolor}
\definecolor{deepblue}{rgb}{0, 0, 0.5}
\definecolor{deepred}{rgb}{0.6, 0, 0}
\definecolor{deepgreen}{rgb}{0, 0.5, 0}
\usepackage{tikz}

% links style
\usepackage[unicode]{hyperref}
\hypersetup{
	colorlinks,
	linkcolor={red!50!black},
	citecolor={blue!50!black},
	urlcolor={blue!80!black}
}
\usepackage{cleveref}

% acronyms
%\usepackage[acronym, xindy]{glossaries}
%\makenoidxglossaries
%\input{./texts/zkratky}

% code typeseting
\usepackage{listings}
\lstset{numbers=none,
		stepnumber=2,
		numbersep=5pt,
		numbers=none,
		tabsize=4,	
		frame=lines,
		language=Python,
		otherkeywords={self},
		breaklines=true,
		breakatwhitespace=true,
		basicstyle=\ttfamily,%\footnotesize
		keywordstyle=\bfseries\color{deepblue},
		commentstyle=\small\color{gray},
		stringstyle=\color{deepgreen},
		showstringspaces=false
}

\lstset{escapeinside={/*!}{!*/}}

\newcounter{lstannotation}
\setcounter{lstannotation}{0}
\renewcommand{\thelstannotation}{\ding{\number\numexpr181+\arabic{lstannotation}}}
\newcommand{\annotation}[1]{\refstepcounter{lstannotation}\label{#1}\thelstannotation}
\newcommand{\lann}[1]{\refstepcounter{lstannotation}\label{#1}\thelstannotation}

%inline code typesseting
\newcommand{\shellcmd}[1]{\texttt{#1}}
\newcommand{\pysauce}[1]{\lstinline!#1!}

% some logos 
\providecommand{\Microsoft}{{\sffamily Microsoft}}	
\providecommand{\Windows}{{\sffamily Windows}}	

%TO-DOcomand
\providecommand{\todo}{\textbf{TODO }}


% math stuff
\newtheorem{theorem}{Theorem}
\newtheorem{princ}{Principle}
\newtheorem{axiom}{Axiom}[chapter]
\newtheorem{claim}{Proposition}
\newtheorem{Lemma}{Lemma}

\theoremstyle{definition}
\newtheorem{definition}{Definition}
\newtheorem{example}{Example}

%\renewcommand{\listtheoremname}{List of examples}

\providecommand{\bracesanddots}[4]{$\langle$#1$\rangle$:$\langle$#2$\rangle$:$\langle$#3$\rangle$:$\langle$#4$\rangle$}
\providecommand{\anglebracs}[1]{$\langle$#1$\rangle$}
\providecommand{\manglebracs}[1]{\langle#1\rangle}
\providecommand{\norm}[1]{\left\lVert#1\right\rVert}
\providecommand{\abs}[1]{\left\lvert#1\right\rvert}
\providecommand{\doteq}{\,\dot = \,}
\newcommand{\pdiff}[2]{\frac{\partial#1}{\partial#2}}
\newcommand{\fdiff}[2]{\frac{\text{d}#1}{\text{d}#2}}
\newcommand{\Laplace}{\Delta}
\newcommand{\DAlambert}{\mathop{}\!\mathbin\Box}

% number classes
\providecommand{\realc}{\mathbb{R}}
\providecommand{\naturc}{\mathbb{N}}
\providecommand{\finaturc}{\mathbb{FN}}
\providecommand{\fisnaturc}{\mathbb{F^*N}}
\providecommand{\comc}{\mathbb{C}}

% convenient math functions
\newcommand\tg{\qopname\relax o{tg}}
\newcommand\tgh{\qopname\relax o{tgh}}
\newcommand\sign{\qopname\relax o{sign}}
\newcommand\cotg{\qopname\relax o{cotg}}
\newcommand\atg{\qopname\relax o{arctg}}
\newcommand\im{\qopname\relax o{Im}}
\newcommand\re{\qopname\relax o{Re}}

\newcommand{\Cls}{\mathrm{Cls}}
\newcommand{\Set}{\mathrm{Set}}
\newcommand{\Fin}{\mathrm{Fin}}
\newcommand{\Nat}{\mathrm{Nat}}

\newcommand{\dom}{\mathrm{dom}}
\newcommand{\rng}{\mathrm{rng}}


\numberwithin{equation}{section}
\usepackage{chngcntr}
\counterwithout{figure}{chapter}
\numberwithin{table}{section}



\title{Discontinuous Galerkin method, its analysis and implementation into SfePy package}
\author{Tomáš Zítka}
%\university{University of West Bohemia in Pilsen}
%\faculty{Faculty of Applied Science}
%\department{Department of Mathematics}
%\subject{Diploma Thesis}
%\town{Pilsen}
\date{$2018/2019$}


\begin{document}
\pagenumbering{gobble}
\begin{titlepage}	 
	\begin{figure}[h]
 	 \centering
 	 \includegraphics[scale=1]{./fav_cmyk.pdf}
 	\end{figure}
 	\center %{Fakulta aplikovaných věd Západočeské univerzity v Plzni}\\[4.5cm]
 	
	\textit{{\large Diploma Thesis}}\\[1.5cm]
	\textsc{\LARGE  Discontinuous Galerkin method, its analysis and implementation into SfePy package}\\

	\vfill
	\begin{minipage}{0.4 \textwidth}
		\begin{flushleft}
		Tomáš \textsc{Zítka}\\
		\end{flushleft}	
	\end{minipage}
	\begin{minipage}{0.4\textwidth}	
		\begin{flushright}
		Pilsen $2020$
		\end{flushright}
	\end{minipage}
\end{titlepage}


%\chapter*{Prohlášení}
%\thispagestyle{empty}
%\begin{otherlanguage}{czech}
%Prohlašuji, že jsem závěrečnou práci vypracoval samostatně 
%s~použitím odborné literatury a~pramenů, uvedených
%v seznamu, který tvoří přílohu této práce.\\[1.5cm]
%%\vfill
%\begin{minipage}{0.4 \textwidth}
%	\begin{flushleft}
%		V Plzni dne $31.\,6.\,2020$
%	\end{flushleft}
%\end{minipage}
%\begin{minipage}{0.4\textwidth}	
%	\begin{flushright}
%		\vskip 2em
%		Tomáš \textsc{Zítka}
%	\end{flushright}
%\end{minipage}\\[1.5cm]
%\end{otherlanguage}
%
%
%
%\newpage
%\thispagestyle{empty}
%\begin{otherlanguage}{czech}
%\chapter*{Poděkování}
%Rád bych poděkoval svému školiteli Mgr. Robertu Cimrmanovi, Ph.D. za trpělivost a~cenné 
%rady při vedení této práce.
%\end{otherlanguage}
%
%\thispagestyle{empty}
%\chapter*{Abstract}
%
%{\let\clearpage\relax\chapter*{Abstrakt}}
%\begin{otherlanguage}{czech}
%	
%\end{otherlanguage}

\newpage
\pagenumbering{roman}
\tableofcontents
\listoffigures
\listoftheorems

\makeatletter
\if@twoside \ifodd\value{page}
\clearpage\mbox{}\thispagestyle{empty} \fi \fi
\makeatother

\clearpage
\pagenumbering{arabic}
\chapter{Introduction}
\label{ch:introduction}
% !TeX spellcheck = en_US
%%
%% Text of diploma thesis
%%
%% Tomáš Zítka
%%
In this work we present implementation of discontinuous Galerkin Finite 
Elements Method (DG FEM) in software package SfePy (Simple Finite Elements for 
Python) and results of experimental measurement of convergence of the method. 
%and its comparison to grad-div, SUPGand PSPG stabilization of the classical FEM 
%methods. 
We consider several model problems, the basic one being linear 
advection with constant advection speed $\vec{a}$
$$
\pdiff{p}{t} - \vec{a}\cdot\nabla p = 0,
$$
with boundary conditions
$$
\begin{array}{ll}
p(t, \vec{x}) = p_D(t, \vec{x}) & \text{ for } \vec{x} \text{ in } \Gamma_D \subset 
\partial\Omega, \\
$$
$$
\pdiff{p}{\vec{n}}(t, \vec{x}) = p_N(t, \vec{x}) & \text{ for }\vec{x}\text{ in } \Gamma_N \subset \partial\Omega.
\end{array}
$$
Which we gradually generalize adding source term $g$, nonlinearity $\vec{f}$ 
and diffusion $D$
\begin{equation*}
	\pdiff{p}{t} + \nabla\cdot \vec{f}(p) - D\Delta p = g.
\end{equation*}
First main goal of this work is to provide implementation of discontinuous Galerkin 
method which could be used to empirically study the behavior of the method but also in 
academic and potentially real world applications and in education. 
Second main goal is to use this implementation to analyze behavior of the method when 
applied to chosen model problems based on equations above especially with regard to 
choice of different fluxes and penalty terms (see below).

The work is divided as follows: The introductory chapter summarizes literature on 
and basic concepts of DG FEM and introduces Sfepy. In second chapter we derive the 
method for model problem and explore theory behind it. Third chapter describes 
in detail Sfepy package and implementation of the method. Fourth chapter 
presents setup and results of numerical experiments measuring convergence 
%with regard to the newest theoretical results presented in \cite{Roe2017}. 
%Fifth 
%chapter presents comparison with grad-div, SUPG \cite{Rapin2007} and PSPG 
%stabilization solved by an Oseen solver\cite{} applied to advection-diffusion 
%equation. 
In the concluding chapter we discuss the results and present 
suggestions for future work.

\section{Basic concepts and literature overview}
In continuous or classical finite element discretization of partial differential equation 
the solution is approximated by as combination of basis functions whose supports span 
across multiple geometrical elements of the mesh discretizing computational domain. This 
enforces continuity of the solution and provides a way of transferring information 
between the elements. In discontinuous Galerkin FE methods on the other hand the 
basis functions used in approximation of test and state variable have supports limited to 
the individual geometrical elements, much like piecewise approximation in finite volume 
(FV) methods. This leads to compact discretization stencils and allows for 
discontinuities in solution but also requires fluxes at element interfaces to be 
introduced in order to transfer information between elements. As we shall see these 
properties of the DG FE methods prove to be useful in some applications and burden in 
others.

Discontinuous approximation and compact stencil makes DG FEM appealing for 
multi-domain and multi-physics simulations \cite{DiPietro2012}.
Possibility to approximate discontinuous solutions proves useful in modeling so-called 
shock waves in nonlinear conservation laws with small dissipation \cite{Kucera}(see 
\Cref{ex:burgers_hest}). Inherent discontinuity of solution, however, brings difficulties 
for diffusion 
dominated or otherwise naturally continuous problems and forces introduction of so-called 
penalty terms (more in Section \ref{se:diff_term}, \cite{Antonietti2013} and  
\cite{Kucera}). Introduction of fluxes provides great flexibility of DG FEM and allows 
for straightforward implementation of conservation laws which endows method with 
good stability properties when approximating advection dominated problems. 
Disadvantages of the use of fluxes are complicated theoretical analysis of the 
methods and lack of exact solution to Riemann problem for high order approximation in 
individual mesh elements. Use of approximate fluxes for solving Riemann problems with 
rough initial data with large gradients introduces oscillations not present in FV methods 
mandating use of so-called limiters (more in \Cref{se:limiters}, \cite[Sec. 
3.2.4]{DiPietro2012} and \cite{Krivodonova2007}). The FE nature of DG FEM and use of 
fluxes allows the DG FEM to be interpreted both as Galerkin projection onto suitable 
energy spaces as well as high order classical upwind finite volume schemes 
\cite{Georgoulis2011}. 

Although studied thoroughly, as wast array of literature cited above suggests, DG FE 
methods still pose research challenges and promise new and potentially useful results for 
numerical modeling. Among challenges are those mentioned above. Among great promises is 
so called super-convergence observed for certain problems \cite{Roe2017} which yet awaits 
to be leveraged in applications.


\section{SfePy -- Simple Finite Elements for Python}

Simple finite elements for Python (SfePy, 
\url{http://sfepy.org/}) is a software 
package providing FE based methods along with wide range of tools for defining, solving 
and post-processing variety of coupled PDEs in 1D, 2D and 3D. It can be viewed both as a 
black-box PDE solver, and as a Python package which can be used for building custom 
application \cite{Cimrman_Lukes_Rohan_2019}. The code of the package is open-source 
published under New BSD-3 Clause license \cite{bsd3-lic} and is available on Github 
(\url{https://github.com/sfepy/sfepy}) %\cite{sfepy-git}.
Detailed documentation with many examples can be found in 
\cite{sfepy-doc}.

SfePy can use many FE based terms to 
build the PDEs to be solved. This approach is reflected in Section \ref{ch:theory}
where the discretization of the equation is divided into discretization of individual 
terms, these are then implemented individually in Chapter \ref{ch:implementation}. As of 
time of writing Sfepy supports classical FEM and isogeomteric analysis (IGA) based FEM 
and 
provides tools for setting up, solving and post-processing problems in applications like 
homogenization of porous media, acoustic waves in thin perforated layers,  finite element 
formulation of Schroedinger equation or flow of a two-phase non-Newtonian fluid medium in 
a general domain \cite{Cimrman_Lukes_Rohan_2019}.

There are several other software packages implementing  DG FEM, some of the currently 
available codes are:
hpGEM \cite{hpgem2007} (\url{https://hpgem.org/}) which provides implementation of 
so-called $hp$-methods in C++; 
FEniCS project \cite{fenics2015} (\url{https://fenicsproject.org/}) which is well 
established numerical software build on C/C++;
PyFR \cite{pyfr2014} (\url{http://www.pyfr.org/}) which is  Python based framework for 
solving advection-diffusion type problems that leverages locality of DG FE methods to run 
computations efficiently on modern streaming architectures, such as Graphical 
Processing Units (GPUs).








\chapter{Discontinous Galerkin Method}
\label{ch:theory}
In this chapter we lay theoretical background for the concepts necessary to describe the 
method and then derive the discretization of advection term $a\cdot\nabla(u)$, general 
nonlinear hyperbolic term $\nabla\cdot f(u)$, diffusion term $\nabla \cdot (D 
\nabla u)$ and source term $g(x)$ occurring in various PDEs namely, from the 
simplest one: linear advection equation with constant and non-constant velocity, 
advection with diffusion and sources, general nonlinear hyperbolic equation.
Using the discretized terms we will formulate method for these equations. Towards the end 
of the chapter we introduce limiters necessary for stabilization of the high order 
versions of the scheme.

\providecommand{\elint}{\int_{T^k}}
\providecommand{\elbint}{\int_{\partial T^k}}
\section{Terms and equations}
Basic equation we will be concerned with is the partial hyperbolic-elliptic 
equation for the unknown function $p$, $p : \Omega \rightarrow \realc$, where 
$\Omega \subset \realc$ is physical domain with boundary $\partial\Omega$ in 
the stationary form
\begin{equation}
\label{eq:hyp_diff}
\nabla\cdot \vec{f}(p) -  \nabla \cdot (D \nabla p) = g
\end{equation}
where $f$ is sufficiently smooth vector function $f: \realc 
\rightarrow\realc^n$, with the boundary conditions
\begin{align}\label{eq:diff_bcs}
p(t, \vec{x}) = p_D(t, \vec{x}) & \text{ for } \vec{x} \text{ in } 
				\Gamma_{Dir} \subset \partial\Omega \\
\pdiff{p}{\vec{n}}(t, \vec{x}) = p_N(t, \vec{x}) & \text{ for }\vec{x}\text{ in 
} \Gamma_{New} \subset \partial\Omega
\end{align}
or in the transient form
\begin{equation}
	\label{eq:hyp_diff_trans}
	\pdiff{p}{t} + \nabla\cdot \vec{f}(p) -  \nabla \cdot (D \nabla p) = g
\end{equation}
with boundary conditions of the same form and initial condition
\begin{equation}
	\label{eq:diff_ic}
	p(0, \vec{x}) = p_0(\vec{x}).
\end{equation}
For this problem we will be concerned with discretization of the generally 
nonlinear hyperbolic term covered in \Cref{se:hyp_term}
\begin{equation}
	\nabla\cdot \vec{f}(p)
\end{equation}
which also covers discretization of linear advection term
\begin{equation}
\vec{a} \cdot \nabla p;
\end{equation}
diffusion term covered in \Cref{se:diff_term}
\begin{equation}
	-  \nabla \cdot (D \nabla p)
\end{equation}
source term in \Cref{se:source_term}
\begin{equation}
	g
\end{equation}
and finally in \Cref{se:time_theory} we will 
treat discretization of temporal derivative term
\begin{equation}
\pdiff{p}{t}.
\end{equation}


\section{Finite dimensional discontinuous approximation space}
In order to discretize the terms and further the equations we first need to 
establish approximation space we will use similarly to continuous FEM. We start 
by choosing suitable computational domain $\Omega_h$ which approximates domain 
$\Omega$. Since SfePy supports simplex and tensor product meshes, we will be 
concerned with space filling tessellations containing only triangles or 
only quadrangles. Subscript $h$ denotes average diameter of the elements, $N$ 
number of elements of $\Omega_h$ and individual elements will be denoted by 
$T^k$ for $k=0, \ldots, N - 1$. Creating this suitable tessellation for 
arbitrary computational domain is by no means trivial task, however for the 
time being we will assume selected computational domain and mesh satisfies all 
conditions required bellow. First we define piecewise continuous functions 
space on $\Omega_h$ as 
\begin{equation}
	C^1(\Omega_h) =  \left\{v;\; v\vert_{T^k} \in C^1 \quad \forall T^k \in 
	\Omega_h \right\}
\end{equation}
and broken Sobolev space on $\Omega_h$ as
\begin{equation}\label{eq:sobh}
	W^{1, 2}(\Omega_h) = \left\{v;\; v\vert_{T^k} \in W^{1, 2}\quad \forall T^k 
	\in \Omega_h \right\}
\end{equation}
In finite dimensional discretization we will work in finite dimensional 
subspaces of $W^{1, 2}(\Omega_h)$. On each element $T^k$ we express local 
solution as linear combination of polynomial basis functions
\begin{equation}
\label{eq:el_lin_comb}
	p_h^k(t, \vec{x}) \approx \sum\limits_{n=0}^{N_{base} - 1} 
	P_n^k\psi^k_n(\vec{x}),
\end{equation}
i.e. as function from local space
\begin{equation}
	V_{T^k} = \text{span}\big\{ \psi_n^k(\vec{x}), \; n = 0,1, \dots  
	N_{base}-1\big\},
\end{equation}
where $N_{base}$ is number of basis functions we use in approximation and hence 
dimension of approximation space. This number is directly tied to approximation 
order and is dependent on the type of mesh element. We also require that
\begin{equation}
	\text{supp}\big\{\psi_n^k(\vec{x}\big)\} = T^k \quad n \in \{0,1, 
	\dots  N_{base}-1\}.
\end{equation}
This means that basis functions are localized to individual elements and allow 
us to represent discontinuous solution, unlike in the classical FEM, where 
supports of basis functions overlap, spanning multiple elements and  thus 
enforcing continuity of solution.

In our setting $\psi_n^k(\vec{x})$ is composed of Legendre polynomials shifted 
to interval $[0, 1]$ in such a way that set $\{\psi_n^k(\vec{x}),\; i = 0,1, 
\dots  N_{base} - 1\}$ is orthogonal with respect to local scalar product
\begin{equation}
	\label{eq:scalar_prod_dk}
	(p, v)_{T^k} = \int_{T^k} p \cdot v.
\end{equation}
hence forming basis of $N_{base}$-dimensional space. 

%Setting 
%\begin{equation}
%N_{base} =  \frac{(Ord + 1) \cdot (Ord + 2) \cdot ... \cdot (Ord + d)}{d!},
%\end{equation}
%where $Ord$ is desired order of approximation -- 0 for finite volumes, 1 for 
%linear approximation, 2 for quadratic approximation etc. and $d$ is dimension 
%of $\Omega_h$; as number of basis functions, provides maximal number of basis 
%function for given order which remain orthogonal. 

Concrete shape of $\psi_n^k(\vec{x})$ depends on topology of the elements. For 
tensor product meshes i.e. quadrilateral elements we use straight-forward 
tensor product of Legendre polynomials. If we denote $M$ the order of 
approximation, $d$ dimension of geometric space and $L^\alpha(\xi)$ the shifted 
Legendre polynomial of order $\alpha$ in variable $\xi$, then we get 
quadrilateral element basis functions in the form
\begin{equation}
\psi_n^k(\vec{x}) = L^r(x)L^s(y)\quad r, s = 0,1, \dots, M
\end{equation}
and dimensions of the approximation space is
\begin{equation}
	N_{base} = (M + 1)^d.
\end{equation}

%and for "cube" is
%\begin{equation}
%\psi_n^k(\vec{x}) = L^r(x)L^s(y)L^d(z)\quad r,s,d = 0,1, \dots, Ord \text{ 
%s.t. } r + s + d = n \leq Ord.
%\end{equation}

In case of simplex meshes, the shape of $\psi_n^k(\vec{x})$ is result of 
Gram-Schmidt orthogonalization process on canonical basis 
$$
\left\{ x^ry^s,  \quad r, s = 0,1, \dots, M \text{ s.t. } r + s \leq M\right\}
$$
with respect to scalar product \eqref{eq:scalar_prod_dk}
and its shape is much more elaborate, the Jacobi polynomials are needed to 
represent the basis. If we denote $J^{\alpha, \beta}_m$ the $m$-th order 
Jacobi polynomial, the individual basis functions can be written in the form 
\cite{Hesthaven2008}
\begin{equation}
	\psi_n^k(\vec{x}) = J_r(a)J^{2s+1, 0}_s(b)(1 - b)^r\quad r, s = 0,1, \dots, 
	Ord \text{ s.t. } r + s = n \leq Ord
\end{equation}
where
\begin{equation}
	a = 2 \frac{1 + x}{1 - y} - 1, b = y.
\end{equation}
that is we get local polynomial space of dimension
\begin{equation}
N_{base} =  \frac{(M + 1) \cdot (M + 2) \cdot ... \cdot (M + d)}{d!},
\end{equation}

In the whole computational domain $\Omega_h$ the solution can be than thought 
of as being a member of direct sum of local spaces
\begin{equation}
	Le_{\Omega_h}^{M} = \bigoplus\limits_{T^k \in \Omega_h} V_{T^k}
\end{equation}
which is finite dimensional subspace of broken Sobolev space defined in 
\eqref{eq:sobh}, 
that is $Le_{\Omega_h}^{M} \subset  W^{1,2}(\Omega_h)$, $Le$ stands for Legendre, it has 
dimension
\begin{equation}\label{eq:dim_legh}
	N_{dof} = \text{dim}(Le_{\Omega_h}^{M}) = N\cdot N_{base}
\end{equation}

This is local basis commonly used in literature \cite{Hesthaven2008}, 
\cite{Bokhove2008}, however there are also other usable bases, which must not 
be orthogonal [\todo cite] or polynomial [\todo cite]. We will always use full 
basis of the functions, however implementations contains mechanism to omit some 
of them for testing purposes.

\newpage
\section{Spatial discretization}
We can now start formulating discretization in space domain. To discretize 
equation \eqref{eq:hyp_diff} in finite elements manner we first devise weak 
formulation of the problem. First we choose the unknown $p$ and arbitrary test 
function $w$ to be from $C^1(\Omega_h)$. And multiply equation 
\eqref{eq:hyp_diff} by test function, we get
\begin{equation}
	\nabla\cdot \vec{f}(p)\cdot w(\vec{x}) 
	-  \nabla \cdot (D \nabla p)\cdot w( 
	\vec{x}) = g\cdot w(\vec{x}),
\end{equation}
after integration over the domain we get
\begin{equation}\label{eq:hyp_int}
	 \int_{\Omega}\nabla\cdot \vec{f}(p)\cdot w(\vec{x}) 
	 - \int_{\Omega}\nabla \cdot (D \nabla p)\cdot w(\vec{x}) 
	 = \int_{\Omega}g\cdot w(\vec{x}).
\end{equation}
This holds for every cauchy sequence of functions ${p_n}$, ${w_n}$ and using 
Lebesgue dominated convergence theorem we can formulate the problem on closure 
of $C^1(\Omega_h)$ i.e. for $p \in W^{1,2}(\Omega_h)$ and  $w \in 
W^{1,2}(\Omega_h)$, obtaining the equation \eqref{eq:hyp_int} in the form (we 
drop 
independent variables $t$ and $\vec{x}$ notations for brevity) 
\begin{equation}
	\label{eq:sum_int_hyp}
	\sum\limits_{k=0}^{N}\left(
	\elint\nabla\cdot \vec{f}(p)\cdot w
	- \elint\nabla \cdot (D \nabla p)\cdot w 
		\right)
 	=\sum\limits_{k=0}^{N}\left(\elint g\cdot w\right).
\end{equation}
Having arrived to the "broken" integral formulation of the equation we will now 
focus on discretization of individual terms within mesh elements.

\subsection{Hyperbolic term discretization}
\label{se:hyp_term}

Using Green's theorem on the first integral term in \eqref{eq:sum_int_hyp} we 
get
\begin{equation}
	\label{eq:hyp_term}
	\int_{T^k} \nabla\cdot \vec{f}(p)\cdot w = % \int_{T^k} \vec{f}\; \nabla p w = 
	\int_{\partial{T^k}} \vec{n}\vec{f}(p)w - \int_{T^k} \vec{f}(p)\cdot\nabla w,
\end{equation}
where $\vec{n}$ is a normal vector to the boundary $\partial T^k$. 
Approximation of the value of $\vec{f}$ on the boundary of the element plays 
key role in 
discretization using DG FE methods, the issue is that the 
approximate solution is discontinuous across the boundary of an element and two 
values are actually present, $p_{in}$ inner to the element and $p_{out}$ outer, 
coming from its neighbor across particular part of the boundary. Since we deal with 2D 
elements with polygonal boundary the integral over the boundary can be expressed as sum 
of integrals over line segments forming the boundary
\begin{equation}
	\sum_{i=0}^{N_f} \int_{F^k_i} \vec{n_f}\vec{f}(p)w
\end{equation}
If we denote $T^{k'}$ element sharing line segment $F^k_i$ with element $T^k$
value $p_{out}$ corresponds to approximation in this element i.e. $p_{out} = 
p^{k'}_h$. For simplicity of notation we continue using integral over the whole 
boundary of $T^k$ implicitly assuming that $p_{out}$ changes as described above. To 
approximate unknown value of $\vec{f}(p)$  we will use approximate flux $f^*(p_{in}, 
p_{out})$ obtaining the first term on the right-hand side in the form
\begin{equation}
	\label{eq:flux_integral}
	\int_{\partial{T^k}} \vec{n}\vec{f}(p)w = \int_{\partial{T^k}} \vec{n} 
	\cdot f^{*} (p_{in}, p_{out})\cdot w.
\end{equation}
In our setting we will use so-called local Lax-Friedrichs flux 
exclusively, although there are many other possible fluxes, for examples see 
\cite{Kucera, Cockburn2001a}, their later implementation should be straightforward (more 
in Section \ref{se:adv_flux_term_imp}). Lax-Friedrichs flux as given in 
\cite{Hesthaven2008} 
is of the form
\begin{equation}
	\label{eq:lax-frieflux}
	f^{*}(p_{in}, p_{out}) =   \frac{\vec{f}(p_{in}) + \vec{f}(p_{out})}{2}  + (1-\alpha) \vec{n}\frac{C}{2}(p_{in} - 
	p_{out}),
\end{equation}
where $\alpha \in [0, 1]$ is parameter adjusting nature of the flux, $\alpha = 0$ for purely upwind scheme, 
$\alpha = 1$ for central scheme,  and
\begin{equation}
	C = \max_{p \in [p_{in} : p_{out}]} \abs{n_x \pdiff{f_1}{p} + n_y \pdiff{f_2}{p}} =
	    \max_{p \in [p_{in} : p_{out}]} \abs{\vec{n}\cdot\frac{d\vec{f}}{dp}(p)},
\end{equation}
where $[p_{in} : p_{out}]$ denotes closed interval 
$$\big[\min_{\partial{T^k}} (\min(p_{in}, p_{out})), 
	   \max_{\partial{T^k}}(\max(p_{in}, p_{out}))\big].$$ 
Note that for linear case where $f(p) = \vec{a}p$ and $\frac{d\vec{f}}{dp}(p) = \vec{a}$  
the $C$ reduces to
\begin{equation}
	C = |\vec{n}\vec{a}|.
\end{equation}
In this formulation $C$ constitutes upper bound on wave speed at the boundary interface.
In order to simplify notation we denote "jump in" 
quantity $p$ across boundary 
\begin{equation}
	[p] = p_{in} - p_{out}
\end{equation}
and the average of $p$ across boundary  
\begin{equation}
	\label{eq:avrg}
	\manglebracs{p} = \frac{p_{in}+p_{out}}{2}.
\end{equation}
Using this notation we can write
\begin{equation}
	f^*(p_{in}, p_{out}) = \manglebracs{p} + (1-\alpha) \vec{n}\frac{C}{2}[p].
\end{equation}



After approximating $p$ and $w$ on element $T^k$ as linear combinations of 
basis functions as in 
\eqref{eq:el_lin_comb},
\begin{align}\label{eq:state_epprox}
	p(t, \vec{x}) \approx \sum\limits_{i=0}^{N_{base}-1} P_i^k\psi_i(\vec{x})
\end{align}
\begin{align}\label{eq:test_approx}
	w(t, \vec{x}) \approx \sum\limits_{j=0}^{N_{base}-1} W_j^k\psi_j(\vec{x})
\end{align}
and substituting \eqref{eq:flux_integral} to \eqref{eq:hyp_term} we arrive to 
%\begin{equation}
%	\sum\limits_{k=0}^{N}
%		\left(
%			\int_{T^k}\pdiff{p}{t}\cdot w + \int_{T^k} \vec{f}(p)\cdot\nabla w - 
%\int_{\partial{T^k}} \vec{n} \cdot f^{*} (p_{in}, p_{out})\cdot w
%		\right)
%	= 0.
%\end{equation}

%\begin{align}
%	\sum\limits_{k=0}^{N}
%		\left(\vphantom{\sum\limits_{i=0}^{N_{base}-1}} \right.
%		& \int_{T^k}\pdiff{}{t}\sum\limits_{i=0}^{N_{base}-1} P_i^k(t)\psi_i\cdot 
%\sum\limits_{j=0}^{N_{base}-1} 
%W_j^k\psi_j\nonumber\\
%	   +&\int_{T^k} \vec{f}\Big(\sum\limits_{i=0}^{N_{base}-1} 
%P_i^k(t)\psi_i\Big)\cdot\nabla\Big(\sum\limits_{j=0}^{N_{base}-1} 
%W_j^k\psi_j\Big)\nonumber\\ 
%	   -&\left.\int_{\partial{T^k}} \vec{n} \cdot f^{*} (p_{in}, p_{out})\cdot 
%\sum\limits_{j=0}^{N_{base}-1} 
%W_j^k\psi_j \right) = 0.
%\end{align}
\begin{multline}
	\label{eq:disc_hyp_term}
	\int_{T^k} \nabla\cdot \vec{f}(p)\cdot w \approx \int_{T^k} 
	\vec{f}\Big(\sum\limits_{i=0}^{N_{base} - 1} 
	P_i^k\psi_i\Big)\cdot\nabla\Big(\sum\limits_{j=0}^{N_{base} - 1} 
	W_j^k\psi_j\Big)\\	
	-\int_{\partial{T^k}} \vec{n} \cdot f^{*} (p_{in}, p_{out})\cdot 
	\sum\limits_{j=0}^{N_{base} - 1} W_j^k\psi_j
\end{multline}
since this approximation holds for every test function $w \in 
Le_{\Omega_h}^{M}$ we can choose $W_j^k = 1 \; \forall \; 
j, k$, using summation notation for clarity we can then write terms on the  right-hand 
side of \eqref{eq:disc_hyp_term} as 
\begin{equation}\label{eq:hyp_stiff_app}
		a^C_{hyp}(\mathbf{p}) = \int_{T^k} \vec{f}(P_i^k\psi^i)\cdot\nabla\psi^j, 
\end{equation}
\begin{equation}\label{eq:hyp_flux_app}
	a^F_{hyp}(\mathbf{p}) = \int_{\partial{T^k}} \vec{n} \cdot f^{*} (p_{in}, 
	p_{out})\cdot\psi^j.
\end{equation}
where $\mathbf{p}$ denotes vector of unknowns $P^k_i$, note that we do not include the 
sign in front of the flux term.
This finalizes discretization of general hyperbolic term $\nabla\cdot\vec{f}(p) 
\cdot w$, the two terms -- integral over element $T^k$ (often called stiffness 
term) and integral over its surface $\partial T^k$ -- are implemented in SfePy 
as \pysauce{AdvectDGFluxTerm} and \pysauce{ScalarDotMGradScalarTerm} in special 
case $f = \vec{a}p$ and as \pysauce{NonlinearHyperDGFluxTerm} and 
\pysauce{NonlinScalarDotGradTerm} in general case. See Section 
\ref{se:adv_flux_term_imp} in Chapter \ref{ch:implementation} for details on 
implementation.


%\begin{equation}
%	\sum\limits_{k=0}^{N}
%	\left(
%		\int_{T^k}\pdiff{}{t}P_i^k\psi^i\cdot W_j^k\psi^j + 
%		\int_{T^k} \vec{f}(P_i^k\psi^i)\cdot\nabla W_j^k\psi^j 
%		- \int_{\partial{T^k}} \vec{n} \cdot f^{*} (p_{in}, p_{out})\cdot W_j^k\psi^j
%	\right) = 0.
%\end{equation}
%and using orthogonality of basis ${\psi_i}$ we get
%\begin{equation}
%	\sum\limits_{k=0}^{N}
%	\left(
%		\int_{T^k}\pdiff{}{t}P_i^k \psi^2_i\cdot W^k_j + 
%		\int_{T^k} \vec{f}(P_i^k\psi^i)\cdot\nabla W_j^k\psi^j 
%		- \int_{\partial{T^k}} \vec{n} \cdot f^{*} (p_{in}, p_{out})\cdot W_j^k\psi^j
%	\right) = 0.
%\end{equation}
%For simplicity we will be now concerned with the linear scalar advection equation with constant coefficient $\vec{a}$ 
%representing advection velocity of the quantity $p$ i.e. $\vec{f}(p) = \vec{a}p$. This allows us to transform the formulation to
%simpler form
%\begin{equation}
%	\label{eq:adv_disc1}
%	\sum\limits_{k=0}^{N}
%	\left(
%		\int_{T^k}\pdiff{}{t}P_i^k \psi^2_i\cdot W^k_j + 
%		\int_{T^k} \vec{a}P_i^k\psi^i\cdot\nabla W_j^k\psi^j 
%		- \int_{\partial{T^k}} \vec{n} \cdot f^{*} (p_{in}, p_{out})\cdot W_j^k\psi^j
%	\right) = 0.
%\end{equation}
%\begin{eqnarray}
%	\label{eq:adv_disc2}
%	\sum\limits_{k=0}^{N}
%	\left(
%		\int_{T^k}\pdiff{}{t}P_i^k \psi^2_i\cdot W^k_j + 
%		\int_{T^k} \vec{a}P_i^kW_j^k\psi^i\cdot\nabla \psi^j 
%		- \int_{\partial{T^k}} \vec{n} \cdot f^{*} (p_{in}, p_{out})\cdot W_j^k\psi^j
%	\right) = 0.
%\end{eqnarray}
%...

%This provides us with discrete operator $\mathcal{L}$ such that
%$$
%\pdiff{u}{t} = \mathcal{L}(u, t)
%$$
%we use in time discretization.


\subsection{Elliptic term discretization}
\label{se:diff_term}
To discretize the elliptic diffusion term
$$
\elint\nabla\cdot(D\nabla p) w
$$
we use Green's theorem as well, obtaining
\begin{equation}
	\label{eq:diff_after_green}
	\elint\nabla \cdot (D \nabla p)\cdot w  = \elbint D(\nabla p\cdot\vec{n}) w - \elint D\nabla p \nabla w
\end{equation}
on the boundary $\partial T^k$ we define, using notation from \eqref{eq:avrg},
\begin{equation}
	\nabla p = \frac{\nabla p_{in} + \nabla p_{out}}{2} = \manglebracs{\nabla p}.
	\label{eq:avrg_grad_state}
\end{equation}
Substituting \eqref{eq:avrg_grad_state} to \eqref{eq:diff_after_green} we get 
so called incomplete scheme
\begin{equation}
		\elbint D \manglebracs{\nabla p} \cdot \vec{n} [w] - \elint D\nabla p \nabla w
\end{equation}
we get symmetric resp. non-symmetric scheme by adding term
\begin{equation}
	\elbint D \manglebracs{\nabla w }\cdot \vec{n} [p]
\end{equation}
with "$+$" resp. "$-$" sign \cite{Kucera}. However such scheme is not stable and we need 
to compensate for
the discontinuities of the $p$ across elements by adding interior penalty term \cite{Kucera, Antonietti2013}
\begin{equation}
	\nu \elbint C_w \cdot \frac{Ord^2}{d(\partial T^k)} [p][w]
\end{equation}
where constant $\nu$ captures properties of diffusion tensor $D$, in case $D = 
\varepsilon, \; \varepsilon > 0$ we set $\nu = \varepsilon$; $C_w$ is parameter 
at our disposal used to fine tune the penalty term.
See section ... in chapter 
\ref{ch:convergence} for details on choice of the value of $C_w$. And finally 
$d(\partial T^k)$ is the volume of the boundary of $T^k$. To simplify notation 
we denote
\begin{equation}\label{eq:diff_penalty_sigma}
\sigma = \nu C_w \cdot \frac{Ord^2}{d(\partial T^k)}.
\end{equation} 
We further proceed as for hyperbolic term. By replacing $p$ and $w$ by their 
finite dimensional approximations \eqref{eq:state_epprox} and 
\eqref{eq:test_approx} and using the fact that the test function $\psi_i$ 
vanishes outside element $T^k$ and hence
\begin{eqnarray}
	\manglebracs{\psi^k_i} = \frac{\psi^k_i}{2},
\end{eqnarray}
\begin{eqnarray}
    [\psi^k_i] = \psi^k_i
\end{eqnarray}
holds, we obtain
\begin{equation}\label{eq:diff_left_approx}
	a^L_{diff}(\mathbf{p}) = \elbint D \manglebracs{\nabla p} \cdot \vec{n} [w] \approx
		\elbint D \manglebracs{P^k_i\nabla\psi_i} \cdot \vec{n}[\psi_j] =
		\elbint D \manglebracs{P^k_i\nabla\psi_i} \cdot \vec{n}\psi_j,
\end{equation}
\begin{equation}\label{eq:diff_right_approx}
		a^R_{diff}(\mathbf{p}) =\elbint D \manglebracs{\nabla w }\cdot \vec{n} [p] \approx
			\elbint D \manglebracs{\nabla \psi_j }\cdot \vec{n} [P^k_i\psi_i] =
			\elbint D \frac{\nabla \psi_j}{2}\cdot \vec{n} 
			[P^k_i\psi_i],
\end{equation}\label{eq:diff_laplace_approx}
\begin{equation}\label{key}
	a^C_{diff}(\mathbf{p})=\elint D\nabla p \nabla w \approx
		\elint D\nabla P^k_i\psi_i \nabla \psi_j,
\end{equation}
\begin{equation}\label{eq:diff_penalty_approx}
		a^P_{diff}(\mathbf{p})=\nu \elbint \sigma [p][w] \approx
	 	\nu \elbint \sigma [P^k_i\psi_i][\psi_j] 
	 	= \nu \elbint \sigma [P^k_i\psi_i]\psi_j.
\end{equation}


\subsection{Source term discretization}
\label{se:source_term}
Discretization of source term is for DG FEM same as for continuous FEM, 
in term
\begin{equation}
	\int_{\Omega_h} g\cdot w
\end{equation}
we take test function from the broken Legendre polyspace obtaining 
\begin{equation}
	\sum\limits_{k=0}^{N}\left(\elint g\cdot w\right)
\end{equation}
after substituting \eqref{eq:test_approx} and choosing $W_j^k = 1 \; \forall \; 
j, k$ we get
\begin{equation}
\sum\limits_{k=0}^{N - 1}\sum\limits_{j=0}^{N_{base} - 1}\left(\elint g\cdot 
\psi_j \right)
\end{equation}
hence for element $T^k$ and test function $\psi_j$ the source term coefficient 
is
\begin{equation}
	b_{source} = \elint g\cdot \psi_j.
\end{equation}

\newpage
\section{Temporal discretization}
\label{se:time_theory}
%\begin{equation} \label{eq:test_theq_time}
%\pdiff{p}{t}\cdot w(t, \vec{x}) + \nabla\cdot \vec{f}(p)\cdot w(t, \vec{x}) 
%-  \nabla \cdot (D \nabla p)\cdot w(t, 
%\vec{x}) = g\cdot w(t, \vec{x}),
%\end{equation}
%
%\begin{equation}
%\label{eq:sum_int_hyp_time}
%\sum\limits_{k=0}^{N}\left(	\elint\pdiff{p}{t}\cdot w
%\elint\nabla\cdot \vec{f}(p)\cdot w
%- \elint\nabla \cdot (D \nabla p)\cdot w 
%\right)
%=\sum\limits_{k=0}^{N}\left(\elint g\cdot w\right).
%\end{equation}
In discretizing transient equation \eqref{eq:hyp_diff_trans} we use the 
discretization of terms devised for stationary equation, with the important 
difference that the discretization coefficients in \eqref{eq:state_epprox} now 
depend on time, that is
\begin{equation}
	P^k_i = P^k_i(t).
\end{equation}
By applying analogous [approximation] to the transient term we obtain
\begin{equation}
	\elint\pdiff{p}{t}(t)w \approx  \fdiff{P^k_i}{t}(t)\elint\psi_i\psi_j. 
\end{equation}
Using this discretization in \eqref{eq:hyp_diff_trans} we obtain system of 
ordinary differential equations for unknown coefficients 
$P^k_i$ in variable $t$
\begin{equation}\label{eq:time_diff}
	M \fdiff{\mathbf{p}}{t}(t) +(\mathbf{p}(t)) = 0
\end{equation}
where 
\begin{equation}
	M_{kij} = \elint\psi_i\psi_j
\end{equation}
is matrix of scalar products of basis functions, often called mass matrix and  
$\mathcal{L}$  is composed of discretizations of terms derived in previous sections.
Since in our setting basis functions are orthogonal on an element and vanish 
outside of it, $M$ is diagonal and its inverse is trivial. We rewrite  
\eqref{eq:time_diff} as
\begin{equation}
	 \fdiff{\mathbf{p}}{t}(t) + M^{-1}\mathcal{L}(\mathbf{p}(t)) = 0
\end{equation}
denoting
\begin{equation}
	\bar{\mathcal{L}} = M^{-1}\mathcal{L} 
\end{equation}
we can write equation 
\eqref{eq:time_diff} in the form
\begin{equation}\label{eq:time_simple}
\fdiff{\mathbf{p}}{t}(t) + \bar{\mathcal{L}}(\mathbf{p}(t)) = 0.
\end{equation}
There is plethora of different schemes for evolving equation 
\eqref{eq:time_simple} we will only present basic forward Euler scheme and so called 
total variations diminishing Runge-Kutta scheme of 3rd order.


\paragraph{Forward Euler scheme} In forward euler scheme we approximate time 
derivation by forward difference i.e.
\begin{equation}
	\fdiff{\mathbf{p}}{t}(t) \approx 	\frac{\mathbf{p}^{n + 1} - 
	\mathbf{p}^{n}}{\Delta t}
\end{equation}
where $n$ denotes time step. Substituting into equation \eqref{eq:time_simple} 
yields
\begin{equation}
	\frac{\mathbf{p}^{n + 1} - \mathbf{p}^{n}}{\Delta t} + 
	 \bar{\mathcal{L}}(\mathbf{p}^n, t^n) = 0
\end{equation}
and after rearranging to obtain explicit equation for $\mathbf{p}^{n + 1}$ we 
get
\begin{equation}
\mathbf{p}^{n + 1} = \mathbf{p}^{n} - {\Delta t} 
\bar{\mathcal{L}}(\mathbf{p}^n, t^n).
\end{equation}
Forward Euler scheme is first order in time. We use it to define the so called 
total variation diminishing property of $\bar{\mathcal{L}}$, that is the total 
variation of the numerical solution in one dimension
\begin{equation}\label{eq:TV}
	TV(p) = \sum_{k} \abs{p_{k + 1} - p_{k}}	
\end{equation}
where $k$ ranges over subsequent 1D mesh elements, does not increase with time i.e.
\begin{equation}
	TV(\mathbf{p}^{n + 1}) \leq TV(\mathbf{p}^{n})
\end{equation}
under update by forward Euler scheme.
This motivates usage of following TVD Runge-Kutta method \cite[p. 73]{Gottlieb2002}. 



\paragraph{TVD Runge-Kutta 3rd order scheme}
Third order total variations diminishing Runge-Kutta scheme \cite{Gottlieb2002} 
is three step scheme that maintains the TVD property while achieving 3rd order 
accuracy in time.
\begin{equation}	
	\begin{aligned}
		\mathbf{p}^{(1)} &= \mathbf{p}^n - \Delta t  
		\bar{\mathcal{L}}(\mathbf{p}^n), \\
		\mathbf{\mathbf{p}}^{(2)} &= \frac{3}{4}\mathbf{p}^n 
		+\frac{1}{4}\mathbf{p}^{(1)} - \frac{1}{4}\Delta t 
		 \bar{\mathcal{L}}(\mathbf{p}^{(1)}),\\
		\mathbf{p}^{(n+1)} &= \frac{1}{3}\mathbf{p}^n 
		+\frac{2}{3}\mathbf{p}^{(2)} - \frac{2}{3}\Delta t 
		 \bar{\mathcal{L}}(\mathbf{p}^{(2)}).
	\end{aligned}
\end{equation}


\subsection{Hyperbolic term stability requirement}

\begin{equation}
	\Delta t \leq c\frac{\Delta x^2}{\varepsilon} \cdot \frac{1}{2M + 1}
\end{equation} 

\subsection{Elliptic term stability requirement}
\begin{equation}\label{eq:cfl_cond}
	\Delta t \leq c\frac{\Delta x}{\norm{\vec{a}}} \cdot \frac{1}{2M + 1}
\end{equation}

\newpage
\section{Initial condition discretization}
Initial condition 
$$
p_0(\vec{x})
$$
is discretized in straight-forward manner as an orthogonal projection into the 
finite dimensional space $Le_{\Omega_h}^{M}$ on domain $\Omega_h$. That is
by solving
\begin{equation}\label{eq:init_proj}
	(P^k_i)^0\elint\psi_i\psi_j = \elint p_0(\vec{x})\psi_j
\end{equation}
for $(P^k_i)^0$. We use the mass matrix notation and the fact that 
it is diagonal and get
\begin{equation}
	(P^k_i)^0 = \frac{1}{M_{ii}}\elint p_0(\vec{x})\psi_i
\end{equation}




\section{Boundary conditions}
Unlike it is common in literature we postponed treatment of the boundary 
conditions (BCs) until now. The reason is to keep the theoretical discussion 
closely tied with implementation. This allows us to clearly demonstrate how 
method works, hopefully providing reader with enough information and 
understanding to modify it. In our implementation treatment of boundary 
condition is separated from term implementation i.e. terms do not have any 
information about boundary conditions, they are merely passed data, which 
already satisfy BCs.

In expressions
\begin{equation}
	  \manglebracs{p} = \frac{p_{in} + p_{out}}{2}
\end{equation}
and 
\begin{equation}
	[p] = p_{in} - p_{out}
\end{equation}
the missing outer value $p_{out}$ in boundary elements is substituted by
\begin{itemize}
	\item value of Dirichlet boundary condition, 
	\item value in corresponding neighbor cell on periodic boundary,
	\item calculated value to satisfy Newton boundary condition,
\end{itemize} 
whenever there is no direct neighbor.
In implementations this is ensured in terms themselves by getting corresponding values 
from \pysauce{DGField} through method \pysauce{get_both_facet_base_vals}..

\newpage
\section{Limiters}
\label{se:limiters}
In high order DG FEM oscillations, which can significantly decrease quality of 
solution, occur even when Courant-Friedrichs-Levy condition \eqref{eq:cfl_cond} is met. 
To combat this limiter needs to be used, in the following section we present moment 
limiters for 1D and 
2D problems.

\subsection{Moment limiter}
Moment limiter due to Krivodonova \cite{Krivodonova2007} leverages idea that 
coefficients for higher order basis functions in hierarchical basis represent 
derivatives of lower order data and uses this to limit the derivative of 
order $i$ in a given cell using derivatives of order $i - 1$ in neighboring 
cells. This kind of limiter unlike others does not reduce the solution to 
first-order accuracy.Unfortunately this kind of limiting is so far available 
only in one dimensional problems and in two dimensional problems with tensor 
product meshes.

\subsubsection{One-dimensional limiting}
\label{sse:moment_lim_1D}
We limit the solution in cell $T^k$
\begin{equation}
\label{eq:el_lin_comb_lim}
p_h^k(t, \vec{x}) = \sum\limits_{i=0}^{N_{base}} P_i^k\psi_i(\vec{x})
\end{equation}
by limiting its coefficients $P_i^k$, starting with the coefficients of 
highest order i.e. $i = N_{base}$ we subsequently replace 
$P_i^k$ with
\begin{equation}
\label{eq:limiter_1D}
	\tilde{P}_i^k = \text{minmod}\left(P_i^k, 
						\alpha_i(P_{i-1}^{k+1} - P_{i-1}^k), 
						\alpha_i (P_{i-1}^k - P_{i-1}^{k-1})\right)
\end{equation}
stopping when $P_i^k = \tilde{P}_i^k$. In the definition of limiter \eqref{eq:limiter_1D} 
$\text{minmod}$ is a function of three 
variables
\begin{equation}
	\text{minmod}(a,b,c) = 
	\begin{cases}
			\text{sign}(a)\min(\abs{a}, \abs{b}, \abs{c}) & \;\text{if}\; 	
															\text{sign}(a) =
															\text{sign}(b) = 
															\text{sign}(c)\\
														0 & \;\text{otherwise}
	\end{cases}.
\end{equation}
and $\alpha_i$ is limiting coefficient dependent on the order, Krivodonova 
\cite{Krivodonova2007} proposes to take $\alpha_i$ from range
\begin{equation}
	\frac{1}{2(2i -1)} \leq \alpha_i \leq 1.
\end{equation}
Choosing $\alpha_i$ outside this region results in either loss of accuracy or numerical 
instability \cite[p. 882]{Krivodonova2007}. Lower bound of the interval corresponds to 
the strictest limiting, whereas $\alpha_i = 1$ is the mildest limiter possible \cite[p. 
882]{Krivodonova2007}. The one-dimensional limiter is implemented in 
\pysauce{dg.limiters.MomentLimiter1D}, see Section \ref{se:i_moment_lim_1D} for details.

\subsubsection{Two-dimensional limiting}
In this section we describe extension of moment limiter to tensor-product meshes in two 
space dimensions.
[\todo element indicies]
[\todo basis indicies]
\begin{multline}
\tilde{P}^{k,m}_{i,j} =
	\text{minmod}\big(P^{k,m}_{i,j}, 
								 \alpha_j(P^{k,m+1}_{i,j-1} - P^{k,m}_{i,j-1}),
							     \alpha_j(P^{k,m}_{i,j-1} - P^{k,m-1}_{i,j-1}),\\
								 \alpha_i(P^{k+1,m}_{i-1,j} - P^{k+1,m}_{i-1,j}),
								 \alpha_i(P^{k-1,m}_{i-1,j} - P^{k,m}_{i-1,j})\big)
\end{multline}


\begin{equation}
	\frac{1}{2\sqrt{4n^2 - 1}} \leq \alpha_n \leq \sqrt{\frac{2n - 1}{2n + 1}}
\end{equation} 


%$$
%I(x) = \Bigg\lbrace
%\begin{array}{ll}
%1 & \text{pro } c < x < d \\
%0 & \text{jinde}.
%\end{array}
%$$




\chapter{Discontinous Galerkin Method implementation}
\label{ch:implementation}
In this chapter we explore in detail Sfepy package application interface (API) 
as well as its inner workings in order to explain implementation details of the 
method. We will show several usage examples and hopefully provide enough 
information for users to use the method effectively and even modify it.

% !TeX spellcheck = en_US
%%
%% Text of diploma thesis
%%
%% Tomáš Zítka
%%
\section{Problem specification}
Before we delve into inner workings of \sfepy{} numerical code lets 
introduce the so-called declarative problem specification format. The format relies on 
Python dictionaries, see Listing \ref{lst:laplace} for model problem specification for 2D 
Laplace equation on domain $[0, 1] \times [0, 1]$ simplified from \Cref{ex:laplace}. It 
contains dictionaries declaring components of the problem like regions in a geometric 
domain, field governing the used FE method, state, and test variables, boundary 
conditions, material constants, and functions, etc. Detailed and more general treatment 
of the format can be found in \cite{Cimrman_Lukes_Rohan_2019} here we focus on the 
specification of a problem to be solved using DG FE.

\setcounter{lstannotation}{0}
\begin{lstlisting}[language=Python, 
caption=\pysauce{example\_dg\_laplace.py}, 
label={lst:laplace}]
regions = {'Omega'     : 'all',  /*!\annotation{lsta:laplace_reg}!*/
'left' : ('vertices in x == 0', 'edge'),
'right': ('vertices in x == 1', 'edge'),
'top' : ('vertices in y == 1', 'edge'),
'bottom': ('vertices in y == 0', 'edge')}
fields = {'f': ('real', 'scalar', 'Omega', /*!\lann{lsta:laplace_field}!*/
                 str(approx_order) + 'd', 'DG', 'legendre')}

variables = {'p': ('unknown field', 'f', 0, 1),
             'v': ('test field', 'f', 'p')} /*!\lann{lsta:var}!*/

def analytic_sol(coors): 
  x_1, x_2 = coors[..., 0], coors[..., 1]
  res = 1/2*x_1**2 - 1/2*x_2**2 - a*x_1 + b*x_2 + c
  return res

@local_register_function
def bcs(ts, coors, bc, problem): /*!\annotation{lsta:laplace_bcf}!*/
  x_1, x_2 = coors[..., 0], coors[..., 1]
  res = nm.zeros(x_1.shape)
  if bc.diff == 0:
      res[:] = analytic_sol(coors)
  elif bc.diff == 1:
      res = nm.stack((x_1 - a, -x_2 + b), axis=-2)
  return res

dgebcs = { /*!\annotation{lsta:laplace_bcs}!*/
  'p_left' : ('left', {'p.all': "bcs", 'grad.p.all': "bcs"}),
  'p_right' : ('right', {'p.all': "bcs", 'grad.p.all': "bcs"}),
  'p_bottom' : ('bottom', {'p.all': "bcs", 'grad.p.all': "bcs"}),
  'p_top' : ('top', {'p.all': "bcs", 'grad.p.all': "bcs"})}

materials = {'D': ({'val': [diffcoef], '.Cw': cw},)} /*!\lann{mat}!*/
integrals = {'i': 2 * approx_order}

equations = {'the_equation': /*!\annotation{lsta:laplace_eq}!*/
  "dw_laplace.i.Omega(D.val, v, p) " 
  " - dw_dg_diffusion_flux.i.Omega(D.val, p, v)" 
  " - dw_dg_diffusion_flux.i.Omega(D.val, v, p)" 
  " + dw_dg_interior_penalty.i.Omega(D.val, D.Cw, v, p)"
  "= 0"}

solvers = {'ls': ('ls.auto_direct', {}),/*!\lann{lsta:laplace_solv}!*/
           'newton': ('nls.newton', {})}
options = {'nls'           : 'newton', /*!\lann{lsta:laplace_opts}!*/
           'ls'            : 'ls',
           'output_format' : 'msh'
           'format_variant': 'gmsh-dg'}
\end{lstlisting}


\begin{itemize}
  \item[\ref{lsta:laplace_reg}] \pysauce{regions} dictionary 
  specifies different regions used in boundary conditions specification, \pysauce{Omega} 
  region is required for setting up fields, \pysauce{'edge'} regions are needed for BCs.
  \item[\ref{lsta:laplace_field}]  \pysauce{fields} determine 
  discretization spaces of variables and are defined using tuple of strings.
  \shellcmd{(data type, number of components, region name, approximation order, field 
  type, polyspace)} here field \pysauce{'f'} is field for discretization of real scalar 
  variable in region \pysauce{"Omega"} using discontinuous Galerkin method of order 
  \pysauce{approx_order} in space of Legendre polynomials. 
  \item[\ref{lsta:var}]  Here \pysauce{'p'} is unknown state variable we are solving for 
  and \pysauce{'v'} is test variable, they are both discretized using field \pysauce{'f'} 
  defined above.
  \item[\ref{lsta:laplace_bcf}]  Function \pysauce{bcs} is called during solution, it is 
  supposed to produce values of boundary conditions.
  \item[\ref{lsta:laplace_bcf}] \pysauce{dgebcs} dictionary sets up boundary 
  conditions specifically for DG FE methods, it creates map between 
  boundary regions and, variables and values of functions that determine boundary  
  conditions.
  \item[\ref{mat}] In materials dictionary we specify diffusion coefficient $D$, the dot 
  notation \pysauce{'.Cw'} causes material not to be broad-casted to quadrature points, 
  which is convenient for constants parameterizing terms like $C_w$ or $\alpha$ in 
  \eqref{eq:diff_penalty_sigma} resp. \eqref{eq:lax-frieflux}.
  \item[\ref{lsta:laplace_eq}] Equation to solve is composed from terms derived in 
  Chapter \ref{ch:theory}.
  \item[\ref{lsta:laplace_solv}] Linear and non-linear solvers to use, \sfepy{} supports 
  various solvers including \pysauce{mumps} \cite{MUMPS:2}
  \item[\ref{lsta:laplace_opts}] Various options, \pysauce{'output_format' : 'msh'} and 
  \pysauce{'format_variant' : 'gmsh'} ensure output in format suitable for 
  post-processing using \pysauce{Gmsh} (\url{http://gmsh.info/}, \cite{Remacle2007}).
\end{itemize}


\section{\sfepy{} architecture}
Components in the problem specification file are parsed into various 
Python 
objects  
brought together in the \pysauce{Problem} object, the most important 
are:
\begin{itemize}
    \item \pysauce{Equation} -- representing the equation to be solved,
    \item \pysauce{EssentialBC} -- representing Dirichlet boundary conditions,
    \item \pysauce{PeriodicBC} -- representing periodic boundary conditions,
    \item \pysauce{InitialCondition} -- representing the initial condition, in case of 
    transient problems,
    \item \pysauce{TimeSteppingSolver} -- specifying time discretization scheme, in case 
    of transient problems.
\end{itemize}
The \pysauce{Equation} object is built by combining \pysauce{Term} 
objects these represent individual integral terms that are evaluated in the course of 
solving a problem.Due to the interpreted nature of CPython 
(\url{https://github.com/python/cpython}) in which \sfepy{} is mainly run and which is 
generally too slow for high-performance numerical computation due to overhead from the 
interpreter, \sfepy{} relies on various approaches to speed up the computation, in 
general, it uses fast vectorized operations provided by NumPy and SciPy 
\cite{SciPy-NMeth2020}. C and Cython are used in places where vectorization is not 
possible, or is too difficult or unreadable \cite{Cimrman_Lukes_Rohan_2019}. Our 
implementation relies on NumPy vectorization, especially \pysauce{einsum} function (mode 
details bellow). Terms keep references to other objects:
\begin{itemize}
    \item \pysauce{Variable} -- representing state and test variables,
    \item \pysauce{Material} -- representing various material constants or functions,
    \item \pysauce{Integral} -- representing gauss quadrature rule.
\end{itemize}
The \pysauce{Variable} object in turn contains reference to \pysauce{Field} object that 
manages the chosen discretization -- finite dimensional space represented by 
\pysauce{PolySpace} object and provides method needed to work with it along with the 
computational domain (\pysauce{Domain} object) which stores geometry 
including\pysauce{Region} objects used in definition of \pysauce{EssentialBC} and 
\pysauce{PeriodicBC}.
% !TeX spellcheck = en_US
%%
%% Text of diploma thesis
%%
%% Tomáš Zítka
%%
\section{DG method components}
Having laid out the structure of the \sfepy{} problem and objects needed to create
it and work with it we now present classes needed to implement the DG FEM. Following the
architecture of \sfepy{}, the DG FE method implementation comprises of:
\begin{itemize}
    \item \pysauce{DGField},
    \item \pysauce{LegendrePolySpace} and its subclasses,
    \item \pysauce{LegendreTensorProductPolySpace} and
    \item \pysauce{LegendreSimplexPolySpace};
\end{itemize}
DG specific terms are summarized in the top portion of \Cref{tab:terms}.
\begin{landscape}
    \begin{table}[p!]
%        \centering
        \caption{Table of terms used in DG method.}
        \label{tab:terms}
        \begin{tabular}{lccc}
            \toprule
            Class & Name & Symbol & Expression  \\
            \midrule
            \pysauce{AdvectionDGFluxTerm} &
            \pysauce{"dw_dg_advect_laxfrie_flux"}
            & $a^F_\mathrm{adv}(\mathbf{p})$
            & $\int_{\partial{T^k}} \vec{n} \cdot \vec{f}^{*} (p_{in},
            p_{out})\cdot\psi_j$ \\
            \pysauce{NonlinearHyperbolicDGFluxTerm} &
            \pysauce{"dw_dg_nonlinear_laxfrie_flux"}
            & $a^F_\mathrm{hyp}(\mathbf{p})$
            & $\int_{\partial{T^k}} \vec{n} \cdot \vec{f}^{*} (p_{in},
            p_{out})\cdot\psi_j$ \\
            \pysauce{NonlinearScalarDotGradTerm} &
            \pysauce{"dw_ns_dot_grad_s"}
            & $a^C_\mathrm{hyp}(\mathbf{p})$
            & $\int_{T^k} \vec{f}(P_i^k\psi_i)\cdot\nabla\psi_j$ \\
            \pysauce{DiffusionDGFluxTerm} & \pysauce{"dw_dg_diffusion_flux"}
            & \begin{tabular}{c}
                $a^R_\mathrm{diff}(\mathbf{p})$\\
                $a^L_\mathrm{diff}(\mathbf{p})$
            \end{tabular}
            & \begin{tabular}{c}
                $\elbint D \frac{\nabla \psi_j}{2}\cdot \vec{n}[P^k_i\psi_i]$\\
                $\elbint D \manglebracs{P^k_i\nabla\psi_i} \cdot \vec{n}\psi_j$
            \end{tabular}  \\
            \pysauce{DiffusionInteriorPenaltyTerm}&
            \pysauce{"dw_dg_interior_penalty"}
            & $a^P_\mathrm{diff}(\mathbf{p})$ & $\elbint \sigma
            [P^k_i\psi_i]\psi_j$ \\
            \midrule

            \pysauce{ScalarDotMGradScalarTerm} & \pysauce{"dw_s_dot_mgrad_s"}
            & $a^C_\mathrm{adv}(\mathbf{p})$
            & $\int_{T^k} \vec{a}P_i^k\psi_i\cdot\nabla\psi_j$
            \\
            \pysauce{LaplaceTerm} & \pysauce{"dw_laplace"}
            & $a^C_\mathrm{diff}(\mathbf{p})$
            & $\elint D\nabla
            P^k_i\psi_i \nabla \psi_j$ \\
            \pysauce{DotProductVolumeTerm} & \pysauce{"dw_volume_dot"} & -- &
            $\fdiff{P^k_i}{t}(t)\elint\psi_i\psi_j$ \\
            \bottomrule
        \end{tabular}
    \end{table}
\end{landscape}
\noindent DG specific boundary conditions:
\begin{itemize}
    \item \pysauce{DGEssentialBC},
    \item \pysauce{DGPeriodicBC};
\end{itemize}
multistage time-stepping solvers:
\begin{itemize}
    \item abstract base class \pysauce{DGMultiStageTS} and two solvers used in numerical experiments:
    \item \pysauce{EulerStepSolver},
    \item \pysauce{TVDRK3StepSolver}.
\end{itemize}
Finally limiters were implemented as subclasses of \pysauce{DGLimiter} abstract class (which has no
counterpart in \sfepy{}):
\begin{itemize}
    \item \pysauce{IdentityLimiter} -- provided for convenience to enable
    easily disabling limiter without changing
    syntax,
    \item \pysauce{MomentLimiter1D} -- for 1D problems only,
    \item \pysauce{MommentLimiter2D} -- only for 2D problems on regular tensor
    product meshes.
\end{itemize}
The limiters are used in the problem composition as post-stage hooks passed to 
time-stepping solvers. For technical reasons we also created the \pysauce{DGVariable} 
class in order to bypass the classical FE treatment of boundary conditions, otherwise it 
is similar to the original \sfepy{} \pysauce{Variable} class and we omit its detailed 
description.


\section{DG Field}
The \pysauce{DGField} class inherits from the \pysauce{Field} base class. This provides it with the basic
functionality needed to be used in problem specification. From methods implemented in
\pysauce{DGField}, the most relevant to DG FEM are:
\begin{itemize}
    \item \pysauce{get_both_facet_state_vals} -- which returns values of state on opposing sides of
    the boundary for each element
    \item \pysauce{get_both_facet_base_vals} -- which returns values of basis functions on opposing
    sides of the boundary for each element
    \item \pysauce{get_facet_neighbor_idx} -- which returns indices of cell neighbors for individual
    facets along with index of the facet within the neighboring cell
    \item \pysauce{get_bc_facet_values}
    \item \pysauce{get_facet_boundary_idx}
    \item \pysauce{get_facet_vols}
    \item \pysauce{get_facet_qp}
    \item \pysauce{get_nodal_values}
\end{itemize}

\subsection{Legendre polynomial spaces implementation}
Legendre polynomial spaces are implemented in two classes
\pysauce{LegendreTensorProductPolySpace} and
\pysauce{LegendreSimplexPolySpace}. Both are derived from the abstract class
\pysauce{LegendrePolySpace} which inherits from \sfepy{}
\pysauce{PolySpace}. It implements the method \pysauce{_eval_base} which is used to get values of basis
functions as well as their derivatives. It also contains methods for evaluating Legendre and Jacobi
polynomials common to tensor-product and simplex subclasses. These classes are accompanied by the
function \pysauce{get_n_el_nod}, which returns number of basis functions for the given order, dimension
and type of basis, and the generator \pysauce{iter_by_order} (\ref{lst:iter_by_order}) which generates
tuples of $r$ and $s$ in desired hierarchical order. For example, for the approximation order $2$ and the
tensor-product basis this is:
\pysauce{(0, 0),
    (0, 1),
    (1, 0),
    (0, 2),
    (1, 1),
    (2, 0),
    (2, 1),
    (1, 2),
    (2, 2)}.
\setcounter{lstannotation}{0}
\begin{lstlisting}[language=Python, caption= Iteration over $r$ and $s$
indicies of basis functions \label{lst:iter_by_order}.]

def iter_by_order(order, dim, extended=False):
  
  ...
  
  porder = order + 1
  for k in range(porder):
    for r in range(k + 1):
      yield r, k - r /*!\lann{lsta:yield}!*/
    if not extended: return /*!\lann{lsta:extended}!*/
    for s in range(1, porder):
      for r in range(1, porder):
        if r + s <= porder - 1:
          continue
        yield r, s
\end{lstlisting}
\begin{itemize}
    \item[\ref{lsta:yield}] \pysauce{yield} keyword turns a function into a generator usable in for
    cycles, for example in Listing \ref{lst:limiter_2D}.
    \item[\ref{lsta:extended}] \pysauce{extended} flag distinguishes the simplex basis from
    tensor-product one which uses more basis functions.
\end{itemize}
To obtain values of Jacobi polynomials, we used implementations provided by SciPy in the
\pysauce{special} module.


\section{DG Terms}
Besides terms listed in \Cref{tab:terms} we implemented the abstract class 
\pysauce{DGTerm} from which the other terms inherit. Methods \pysauce{eval_real} and 
\pysauce{call_function} implemented in this class manage calling the method named 
\pysauce{function}, which each term implements, and returning the results to the 
evaluation engine of \sfepy{}. The \pysauce{function} method comes from architecture of 
the terms already present in \sfepy{} where it is used to call extensions programmed and 
optimized using C programing language. This method is called whenever value of the term 
is needed either to build residual vector (i.e. right-hand side of an equation) or to get 
terms contribution to the matrix form of $\mathcal{L}$ (in case of implicit problem). The 
method returns the residual values corresponding to the individual DOFs and in the matrix 
mode also the indices to build the sparse matrix representation.

\subsection{Hyperbolic flux term implementation}
\label{se:adv_flux_term_imp}
\pysauce{AdvectionDGFluxTerm} corresponds to the discretized term \eqref{eq:hyp_flux_app}
where $\vec{f}(p) = \vec{a}p$. The part of the \pysauce{function} capturing computation of
cell fluxes can be seen in Listing \ref{lst:adv_flux} below.
\setcounter{lstannotation}{0}
\begin{lstlisting}[language=Python, caption=Computation of advection cell
fluxes. \label{lst:adv_flux}]
def function(self, out, state, diff_var, field, region, advelo):

  fc_n = field.get_cell_normals_per_facet(region)
  # get maximal wave speeds at facets
  C = nm.abs(nm.einsum("ifk,ik->if", fc_n, advelo))  /*!\lann{lsta:einsm}!*/
    
  if diff_var is not None: /*!\lann{lsta:mtx_mode}!*/
  
    nbrhd_idx = field.get_facet_neighbor_idx(region, state.eq_map)
    active_cells, active_facets = nm.where(nbrhd_idx[:, :, 0] >= 0)
    active_nrbhs = nbrhd_idx[active_cells, active_facets, 0]
    
    in_fc_b, out_fc_b, whs = field.get_both_facet_base_vals(state, region)
    
    inner_diff = nm.einsum("nfk, nfk->nf",  /*!\lann{lsta:diff}!*/
                           fc_n,
                           advelo[:, None, :]
                           + nm.einsum("nfk, nf->nfk",
                             (1 - self.alpha) * fc_n, C)) / 2.
    outer_diff = nm.einsum("nfk, nfk->nf",
                           fc_n,
                           advelo[:, None, :]
                           - nm.einsum("nfk, nf->nfk",
                           (1 - self.alpha) * fc_n, C)) / 2.
    
    inner_vals = nm.einsum("nf, ndfq, nbfq, nfq -> ndb",  /*!\lann{lsta:vals}!*/
                           inner_diff,
                           in_fc_b,
                           in_fc_b,
                           whs)
    outer_vals = nm.einsum("i, idq, ibq, iq -> idb",
                          outer_diff[active_cells, active_facets],
                          in_fc_b[active_cells, :, active_facets],
                          out_fc_b[active_cells, :, active_facets],
                          whs[active_cells, active_facets])
    
    vals = nm.vstack((inner_vals, outer_vals))
    vals = vals.flatten()
    
    # compute positions within matrix
    iels = self._get_nbrhd_dof_indexes(active_cells, active_nrbhs, field)
    
    out = (vals, iels[:, 0], iels[:, 1], state, state)
    
  else:
  
    facet_base_vals = field.get_facet_base(base_only=True)
    in_fc_v, out_fc_v, weights = field.get_both_facet_state_vals(state, region)
    
    # reshape facet base to (n_el_nod, n_el_facet, n_qp)
    fc_b = facet_base_vals[:, 0, :, 0, :].T
    
    fc_v_avg = (in_fc_v + out_fc_v)/2.
    fc_v_jmp = in_fc_v - out_fc_v
    
    central = nm.einsum("ik,ifq->ifkq", advelo, fc_v_avg) 
    upwind = (1 - self.alpha)/2. * nm.einsum("if,ifk,ifq->ifkq",
                                           C, fc_n, fc_v_jmp)
    
    cell_fluxes = nm.einsum("ifk,ifkq,dfq,ifq->id",
                          fc_n, central + upwind, fc_b, weights)
    out [0, 0, :, 0] = cell_fluxes
  
  return out
\end{lstlisting}
\begin{itemize}
    \item[\ref{lsta:einsm}] \pysauce{numpy.einsum} uses the Einstein summation 
    notation for expressing tensor contractions, for details see \cite{einsum-doc}.
    \item[\ref{lsta:mtx_mode}] The presence of \pysauce{diff_var} denotes an evaluation 
    in the matrix mode.
    \item[\ref{lsta:diff}, \ref{lsta:vals}] Variables \pysauce{inner_diff}, 
    \pysauce{outer_diff} and \pysauce{inner_vals} and \pysauce{outer_vals} correspond to 
    the decomposition of the term \eqref{eq:hyp_flux_app}
    \begin{equation}
        \int_{\partial{T^k}} \vec{n} \cdot \vec{f}^{*} (p_{in}.
            p_{out})\cdot\psi_j
    \end{equation}
    First we substitute the flux $\vec{f}^*$ from ,\eqref{eq:lax-frieflux} using 
    $\vec{f}(p) = \vec{a}p$ we get
    \begin{equation}
        \int_{\partial{T^k}} \vec{n} \left(\frac{\vec{a}p_{in} + \vec{a}p_{out}}{2}  
        + (1-\alpha) \vec{n}\frac{C}{2}(p_{in} - p_{out}) \right)\psi_j
    \end{equation}
    and then expand and split the integral
    \begin{equation}
        \int_{\partial{T^k}} \vec{n}\frac{\vec{a}p_{in}}{2}\psi_j
      + \int_{\partial{T^k}} \vec{n}\frac{\vec{a}p_{out}}{2}\psi_j
      + \int_{\partial{T^k}} \vec{n} (1-\alpha) \vec{n}\frac{C}{2}p_{in}\psi_j
      - \int_{\partial{T^k}} \vec{n} (1-\alpha) \vec{n}\frac{C}{2}p_{out}\psi_j.
    \end{equation}
    Rearranging yields
    \begin{equation}
      \int_{\partial{T^k}} \vec{n}\frac{\vec{a}p_{in}}{2}\psi_j
    + \int_{\partial{T^k}} \vec{n} (1-\alpha) \vec{n}\frac{C}{2}p_{in}\psi_j
    + \int_{\partial{T^k}} \vec{n}\frac{\vec{a}p_{out}}{2}\psi_j
    - \int_{\partial{T^k}} \vec{n} (1-\alpha) \vec{n}\frac{C}{2}p_{out}\psi_j,
    \end{equation}
    \begin{equation}
    \int_{\partial{T^k}}  \frac{1}{2}(\vec{n}\vec{a}
        + \vec{n} (1-\alpha) \vec{n}C)p_{in}\psi_j
    +\int_{\partial{T^k}}  \frac{1}{2}(\vec{n}\vec{a}
        - \vec{n} (1-\alpha) \vec{n}C)p_{out}\psi_j,
    \end{equation}
    after substituting corresponding outer and inner values of $p$ we get the coefficient 
    for $P^k_{i}$ in the form
     \begin{equation}
    \overbrace{\int_{\partial{T^k}}  \underbrace{\frac{1}{2}\vec{n}(\vec{a}
    +  (1-\alpha) 
    \vec{n}C)}_{\text{\pysauce{inner_diff}}}\psi^{k}_i\psi_j}^{\text{\pysauce{inner_vals}}}
    +\overbrace{\int_{\partial{T^k}} \underbrace{ \frac{1}{2}\vec{n}(\vec{a} - (1-\alpha) 
    \vec{n}C)}_{\text{\pysauce{outer_diff}}}\psi^{k(l)}_i\psi^k_j}^{\text{\pysauce{outer_vals}}},
    \end{equation}
    where $k(l)$ denotes the coefficients in the neighboring element like in 
    \eqref{eq:hyp_int_facets}.
\end{itemize}
The general hyperbolic term is implemented in the class
\pysauce{NonlinearHyperbolicDGFluxTerm} unlike linear advection term above it does not 
support evaluation in matrix mode.



\subsection{Diffusion flux term implementation}
\label{se:diff_flux_term_imp}

Implementation of the diffusion flux terms follows the same course as the implementation 
of hyperbolic flux terms, with the important difference that 
\pysauce{DiffusionDGFluxTerm} 
implements both terms in \eqref{eq:diff_left_approx} and \eqref{eq:diff_right_approx}. 
This is thanks to two modes in which it can be used in an equation --- this has already 
been demonstrated for the Laplace equation in Listing \ref{lst:laplace} where 
\pysauce{"dw_dg_diffusion_flux.i.Omega(D.val,
p, v)"} corresponds to $a^R_\mathrm{diff}(\mathbf{p})$ and mode \pysauce{'avg_state'}
(\ref{lsta:avg_state}), and \pysauce{"dw_dg_diffusion_flux.i.Omega(D.val, v, p)"}
corresponds to $a^L_\mathrm{diff}(\mathbf{p})$ and mode \pysauce{'avg_virtual'}
(\ref{lsta:avg_virtual}). The implementation of residual mode computation is presented in 
Listing \ref{lst:diffusion_flux}.
\setcounter{lstannotation}{0}
\begin{lstlisting}[language=Python, caption=Computation of diffusion cell
fluxes. \label{lst:diffusion_flux}]
if self.mode == 'avg_state': /*!\lann{lsta:avg_state}!*/
  avgDdState = (nm.einsum("ikl,ifkq->ifkq",
                          D, inner_facet_state_d) +
                nm.einsum("ikl,ifkq->ifkq",
                          D, outer_facet_state_d)) / 2.
  # outer_facet_base is in DG zero
  # hence the jump is inner value
  jmpBase = inner_facet_base

  cell_fluxes = nm.einsum("ifkq ,ifk,idfq,ifq->id",
                          avgDdState, fc_n, jmpBase, weights)

elif self.mode == 'avg_virtual': /*!\lann{lsta:avg_virtual}!*/
  avgDdbase = (nm.einsum("ikl,idfkq->idfkq",
                         D, inner_facet_base_d)) / 2.

  jmpState = inner_facet_state - outer_facet_state
  cell_fluxes = nm.einsum("idfkq, ifk, ifq , ifq -> id",
                          avgDdbase, fc_n, jmpState, weights)
\end{lstlisting}


\subsection{Difusion penalty term implementation}
\label{se:diff_penal_term_imp}
\setcounter{lstannotation}{0}
\begin{lstlisting}[language=Python, caption=Computation of penalty cell
fluxes.]
approx_order = field.approx_order

inner_facet_base, outer_facet_base, whs = \
    field.get_both_facet_base_vals(state, region, derivative=False)
facet_vols = nm.sum(whs, axis=-1)

# nu characterizes diffusion tensor, so far we use diagonal average
nu = nm.trace(diff_tensor, axis1=-2, axis2=-1)[..., None] / \
                        diff_tensor.shape[1]
sigma = nu * Cw * approx_order ** 2 / facet_vols

inner_facet_state, outer_facet_state, whs = \
    field.get_both_facet_state_vals(state, region,
                                    derivative=False)

inner_facet_base, outer_facet_base, _ = \
    field.get_both_facet_base_vals(state, region,
                                   derivative=False)

jmp_state = inner_facet_state - outer_facet_state
jmp_base = inner_facet_base  # - outer_facet_base

n_el_nod = nm.shape(inner_facet_base)[1]
cell_penalty = nm.einsum("nf,nfq,ndfq,nfq->nd",
                         sigma, jmp_state, jmp_base, whs)

\end{lstlisting}

\section{Limiters implementation}
Following design patterns used in \sfepy{} and Python in general, the limiters are
implemented as classes. The base class providing only the constructor is called \pysauce{DGLimiter}, its
subclasses then implement the abstract method \pysauce{__call__} --- this makes all limiters callable
objects, allowing one to pass them as post-step or post-stage or other hooks to time-stepping
solvers. For convenience the identity limiter which does not alter the solution is 
implemented in the class \pysauce{IdentityLimiter}.

\subsubsection{Moment limiter -- 1D}
\label{se:i_moment_lim_1D}
The code listing below shows the implementation of the moment limiter introduced in Section~\ref{sse:moment_lim_1D},
omitting some details for brevity.
\setcounter{lstannotation}{0}
\begin{lstlisting}[language=Python, caption=Moment limiter for 1D.]
idx = nm.arange(nm.shape(u[0, 1:-1])[0])

nu = nm.copy(u)
tilu = nm.zeros(u.shape[1:])
for ll in range(self.n_el_nod - 1, 0, -1):
  tilu[idx] = minmod(nu[ll, 1:-1][idx],
                     nu[ll-1, 2:][idx] - nu[ll-1, 1:-1][idx],
                     nu[ll-1, 1:-1][idx] - nu[ll-1, :-2][idx]) /*!\lann{lsta:lim1}!*/

  idx = idx[nm.where(abs(tilu[idx] - nu[ll, 1:-1][idx])
            > MACHINE_EPS)[0]] /*!\lann{lsta:lim2}!*/
  if len(idx) == 0:
    break /*!\lann{lsta:lim3}!*/
  nu[ll, 1:-1][idx] = tilu[idx] /*!\lann{lsta:lim4}!*/
\end{lstlisting}
\begin{itemize}
    \item [\ref{lsta:lim1}] Compute the limiting value $\tilde{u}$.
    \item [\ref{lsta:lim2}] Extract indicies where the limiting value is
    larger than the
    current solution.
    \item[\ref{lsta:lim3}] If none of the coefficients requires limiting we
    stop.
    \item [\ref{lsta:lim3}] Replace old values with limited ones.

\end{itemize}

\subsubsection{Moment limiter -- 2D}
\label{se:i_moment_lim_2D}
We list the implementation of the 2D limiter for reference in Listing \ref{lst:limiter_2D}. The Limiter is
implemented according to \Cref{se:limiters}.
\setcounter{lstannotation}{0}
\begin{lstlisting}[language=Python, caption=Moment limiter for
cartesian grid. \label{lst:limiter_2D}]
for ll, (ii, jj) in enumerate(
                     iter_by_order(self.field.approx_order,
                                   2,   # dim
                                   extended=ex)):
  nu[ii, jj, ...] = u[ll] /*!\lann{lsta:indx_to}!*/

for ii, jj in reversed(list(
                        iter_by_order(
                            self.field.approx_order, 2,
                            extended=ex))):
  minmod_args = [nu[ii, jj, idx]]
  nbrhs = nbrhd_idx[idx]
  if ii - 1 >= 0:
    alf = nm.sqrt((2 * ii-1) / (2 * ii + 1))
    # right difference in x axis
    dx_r = alf*(nu[ii-1, jj, nbrhs[:, 1]] - nu[ii-1, jj, idx])
    # left differnce in x axis
    dx_l = alf*(nu[ii-1, jj, idx] - nu[ii-1, jj, nbrhs[:, 3]])
    minmod_args += [dx_r, dx_l]
  if jj - 1 >= 0:
    alf = nm.sqrt((2 * jj - 1) / (2 * jj + 1))
    # right i.e. element "up" difference in y axis
    dy_up = alf*(nu[ii, jj-1, nbrhs[:, 2]] - nu[ii, jj-1,  idx])
    # left i.e. element "down" difference in y axis
    dy_dn = alf*(nu[ii, jj-1,  idx] - nu[ii, jj-1,  nbrhs[:, 0]])
    minmod_args += [dy_up, dy_dn]

  tilu[idx] = minmod_seq(minmod_args)
  idx = idx[nm.where(abs(tilu[idx] - nu[ii, jj, idx]) > MACHINE_EPS)[0]]

  if len(idx) == 0:
    break
  nu[ii, jj, idx] = tilu[idx]

resu = nm.zeros(u.shape)
for ll, (ii, jj) in enumerate(
                     iter_by_order(self.field.approx_order,
                                   2,   # dim
                                   extended=ex)):
  resu[ll] = nu[ii, jj] /*!\lann{lsta:indx_from}!*/
\end{lstlisting}
\begin{itemize}
    \item[\ref{lsta:indx_to}] Reshape the solution array for indexing using $r$ and $s$ indicies,
    effectively removing need for the explicit inverse of index mapping from \eqref{eq:bindx}.
    \item [\ref{lsta:indx_from}] Convert back to the linear index.
\end{itemize}


\section{Time-stepping solvers implementation}
As demonstrated in \Cref{se:time_theory}, the explicit DG FEM requires explicit time stepping solvers
with multiple stages in one time step. These had not been part of the rich collection of time-stepping
solvers included in \sfepy{}, so two new solvers were implemented: the basic Euler solver, the total-variations
diminishing Runge-Kutta of the 3rd order (TVD RK-3). Again following the structure of \sfepy{}, they are
implemented as subclasses of \pysauce{TimeSteppingSolver}. The abstract class
\pysauce{DGMultiStageTS} extends the basic \pysauce{TimeSteppingSolver} with the option to provide
pre-stage and post-stage hooks, allowing to apply limiters between stages. The two time-stepping
solvers are then implemented in classes \pysauce{EulerStepSolver} and \pysauce{TVDRK3StepSolver}.


\chapter{Numerical experiments}
\label{ch:convergence}
In this chapter we first introduce PDEs used to study behavior of DG FE method and 
provide short guide to convergence study setup. 
%and post-processing and visualization options using Gmsh \cite{Remacle2007} and custom 
%python script. 
Finally we present results of convergence studies for various problem setups.

% !TeX spellcheck = en_US
%%
%% Text of diploma thesis
%%
%% Tomáš Zítka
%%
\section{Example PDEs}
In the following examples we will busy ourselves solving following equations:\\
Transient advection equation in one resp. two dimensions
\begin{equation}
    \label{eq:ex_advection}
    \pdiff{p}{t} + a\pdiff{p}{x} = 0,
\end{equation}
resp.
\begin{equation}
    \pdiff{p}{t} + \vec{a}\cdot\nabla{p} = 0.
\end{equation}
After applying discretizations devised in \Cref{ch:theory} we obtain both 
equations in the same form
\begin{equation}
         \fdiff{P^k_i}{t}(t)\elint\psi_i\psi_j 
         - \int_{T^k} \vec{a}P_i^k\psi_i\cdot\nabla\psi_j 
         + \int_{\partial{T^k}} \vec{n}
        \cdot f^{*} (p_{in}, p_{out})\cdot\psi_j = 0
\end{equation}
Equation expressed in \sfepy{} declarative notation can be found in Listing 
\ref{lst:advection} below;
\setcounter{lstannotation}{0}
\begin{lstlisting}[language=Python, caption={Advection equation}
\label{lst:advection}]
equations = {'Advection': 
  # transient
  "dw_volume_dot.i.Omega(v, p)"
 
  "- dw_s_dot_mgrad_s.i.Omega(a.val, p[-1], v)" /*!\lann{lsta:history}!*/
  "+ dw_dg_advect_laxfrie_flux.i.Omega(a.flux, a.val, v, p[-1])"  /*!\lann{lsta:flux}!*/
  "= 0"
  }
\end{lstlisting}
\begin{itemize}
    \item[\ref{lsta:history}] \pysauce{"p[-1]"} ensures the variable object 
    stores history one step backwards in time facilitating forward nature of 
    time stepping solvers,
    \item[\ref{lsta:flux}] \pysauce{"a.flux"} is an optional material argument 
    representing $\alpha$ coefficient in \eqref{eq:lax-frieflux}.
\end{itemize}
Laplace equation in two dimensions
\begin{equation}
    \label{eq:ex_laplace}
    - D \left(\pdiff{^2 p}{x^2} + \pdiff{^2 p}{y^2} \right)= 0.
\end{equation}
Employing symmetric discretization of diffusion term and adding diffusion 
penalty term yields equation in the form
\begin{multline}
    \elint D\nabla P^k_i\psi_i \nabla \psi_j
    - \elbint D \manglebracs{P^k_i\nabla\psi_i} \cdot \vec{n}\psi_j 
    - \elbint D \frac{\nabla \psi_j}{2}\cdot \vec{n} [P^k_i\psi_i] \\ 
    + \nu \elbint \sigma [P^k_i\psi_i]\psi_j 
    = 0.
\end{multline}
Discretization of this equation using \sfepy{} terms can be found in Listing 
\ref{lst:laplace};\\
Static advection-diffusion equation with right hand side of the form
\begin{equation}
\label{eq:ex_advdiff}
\pdiff{p}{x} + \pdiff{p}{y} - D \cdot \left( \pdiff{^2 p}{x^2} + \pdiff{^2 
p}{y^2} \right) = g,
\end{equation}
i.e.,
\begin{equation}
\vec{a} \cdot \nabla p - D \Delta p = g,
\end{equation}
where $\vec{a} = [1, 1]^T$ is advection velocity, $D$ is diffusion coefficient 
and $g$ is a source function. Combining discretization of the two previous 
equations we obtain discretized form
\begin{multline}
- \int_{T^k} \vec{a}P_i^k\psi_i\cdot\nabla\psi_j 
+ \int_{\partial{T^k}} \vec{n}\cdot f^{*} (p_{in}, p_{out})\cdot\psi_j\\
+ \elint D\nabla P^k_i\psi_i \nabla \psi_j
- \elbint D \manglebracs{P^k_i\nabla\psi_i} \cdot \vec{n}\psi_j
- \elbint D \frac{\nabla \psi_j}{2}\cdot \vec{n} [P^k_i\psi_i] \\
+ \nu \elbint \sigma [P^k_i\psi_i]\psi_j
- \elint g\cdot \psi_j
= 0.
\end{multline}
In \sfepy{} declarative notation this equation has the form presented in 
Listing \ref{lst:advdiff}.
\setcounter{lstannotation}{0}
\begin{lstlisting}[language=Python, caption=Static advection-diffusion equation
\label{lst:advdiff}]
equations = {'adv_diff' :
    # advection
    "- dw_s_dot_mgrad_s.i.Omega(a.val, p, v)"
    "+ dw_dg_advect_laxfrie_flux.i.Omega(a.flux, a.val, v, p)"
    # diffusion
    "+ dw_laplace.i.Omega(D.val, v, p) "
    "- dw_dg_diffusion_flux.i.Omega(D.val, p, v)"
    "- dw_dg_diffusion_flux.i.Omega(D.val, v, p)"
    # penalty
    "+ dw_dg_interior_penalty.i.Omega(D.val, D.cw, v, p)"
    # source
    "- dw_volume_lvf.i.Omega(g.val, v)"
    "= 0"
    }
\end{lstlisting}
Transient viscous Burgers equation in one resp. two dimensions
\begin{equation}
\pdiff{p}{t} + \frac{1}{2}\pdiff{p^2}{x} - D \cdot\pdiff{^2 p}{x^2} = g,
\end{equation}
resp.
\begin{equation}
\label{eq:ex_burgers}
    \pdiff{p}{t} + \frac{1}{2}\left(\pdiff{p^2}{x} + \pdiff{p^2}{y}\right)  - 
    D \cdot \left( \pdiff{^2 p}{x^2} + \pdiff{^2 p}{y^2} \right) 
    = g,
\end{equation}
i.e.,
\begin{equation}
    \pdiff{p}{t} + \nabla \cdot \vec{f}(p) - D\Delta p = g.
\end{equation}
with $\vec{f}(p) = \frac{1}{2}[p^2, p^2]^T = \frac{1}{2}\vec{a} p^2$.
Discretizing using all the terms derived before we get the same form for both 1D and 2D
\begin{multline}
    \fdiff{P^k_i}{t}(t)\elint\psi_i\psi_j 
    - \int_{T^k} \vec{f}(P_i^k\psi_i)\cdot\nabla\psi_j 
    + \int_{\partial{T^k}} \vec{n} \cdot f^{*} (p_{in}, p_{out})\cdot\psi_j\\
    + \elint D\nabla P^k_i\psi_i \nabla \psi_j
    - \elbint D \manglebracs{P^k_i\nabla\psi_i} \cdot \vec{n}\psi_j
    - \elbint D \frac{\nabla \psi_j}{2}\cdot \vec{n} [P^k_i\psi_i] \\
    + \nu \elbint \sigma [P^k_i\psi_i]\psi_j
    - \elint g\cdot \psi_j
    = 0.
\end{multline}
In \sfepy{} declarative notation this equation has the form presented in 
Listing \ref{lst:burgers}.
\setcounter{lstannotation}{0}
\begin{lstlisting}[language=Python, caption=Viscous Burgers equation \label{lst:burgers}]
burg_velo = nm.array([1., 1.])

def f(p):
    return .5*burg_velo * p[..., None] ** 2

def f_d(p):
    return burg_velo * p[..., None]

equations = {'burgers':
    # transient
    "dw_volume_dot.i.Omega(v, p)"
    #  non-linear hyperbolic terms
    "- dw_ns_dot_grad_s.i.Omega(f, f_d, p[-1], v)" /*!\lann{lsta:nl1}!*/
    "+ dw_dg_nonlinear_laxfrie_flux.i.Omega(f, f_d, v, p[-1])" /*!\lann{lsta:nl2}!*/
    #  diffusion
    "+ dw_laplace.i.Omega(D.val, v, p[-1])"
    "- dw_dg_diffusion_flux.i.Omega(D.val, p[-1], v)"
    "- dw_dg_diffusion_flux.i.Omega(D.val, v, p[-1])"
    # penalty
    "+ dw_dg_interior_penalty.i.Omega(D.val, D.Cw, v, p[-1])"
    # source
    "- dw_volume_lvf.i.Omega(g.val, v)"
    " = 0"
    }
\end{lstlisting}
\begin{itemize}
    \item[\ref{lsta:nl1}, \ref{lsta:nl2}] Nonlinear terms require as parameters function 
    and its derivative.
\end{itemize}


\section{Examples}

\paragraph{Measuring convergence} We define convergence rate $r$ as is common 
in literature
\begin{equation}
%    num_order = nm.log(row["diff_l2"] / last_err) \
%    / nm.log(row["h"] ** dim / last_h ** dim)
r = \frac{\log\left(\dfrac{\Ltwonorm{p - p_{h_{n}}}}{\Ltwonorm{p - p_{h_{n-1}}}}\right)}
{\log\left(\dfrac{h^d_{n}}{h^d_{n-1}}\right)}.
\end{equation}
Inspired by \cite{Kucera} we present plots depicting average convergence rate 
over several mesh refinements, this might not be ideal measure of the method 
behavior, nevertheless it still provides us with convenient indicator. 
Accompanied with plots of $L^2$ error it allows us to reason about method over 
several varying parameters, notably it reveals relationship between diffusion 
coefficient and penalty term in examples including diffusion terms.

In \Cref{ex:adv1D} we explore behavior of DG FE method for 1D pure advection, 
time dependent problem with and without limiting, examples \ref{ex:laplace}, 
\ref{ex:quart1}, \ref{ex:quart2} and \ref{ex:quart3} importance of diffusion 
penalty terms. All the test problems studied in Chapter \ref{ch:convergence} 
are specified using this declarative approach, codes can be found in (\todo 
Attachment? \url{https://github.com/zitkat/dg_examples})


\section{Examples}
Inspired by \cite{Kucera} we present plots depicting average order over several 
mesh refinements, this might not be ideal measure of the method behavior, nevertheless it 
still provides us with convenient indicator. Accompanied with plots of $L^2$ error it 
allows us to reason about method over several varying parameters, notably it reveals 
relationship between diffusion coefficient and penalty term in examples including 
diffusion terms.

In \Cref{ex:adv1D} we explore behavior of DG FE method for 1D pure advection time 
dependent problem with and without limiting, in examples \ref{ex:quart1}, \ref{ex:quart2} 
and \ref{ex:quart3} we ...

\begin{example}[Advection 1D]
\label{ex:adv1D}
In $\Omega = \langle 0, 1 \rangle$ we will solve equation \eqref{eq:ex_advection}.
%\begin{equation}
%\pdiff{u}{t} + \fdiff{u}{x} = 0.
%\end{equation}
We set two the initial conditions $u(0, x)$ to obtain two different solutions:
\begin{equation}
u_{smooth} = \begin{cases}
g(x),\quad &0.1 < x < .3\\
0, \quad &\text{elsewhere}
\end{cases}
\end{equation}
where
\begin{equation}
g(\xi) = \exp\left(\frac{1}{10(\xi - 0.2)^2 - 1}+ 1\right),
\end{equation}
and
\begin{equation}
u_{step} = \begin{cases}
\dfrac{1}{2},\quad &0.1 < x < .3\\
0, \quad &\text{elsewhere}
\end{cases}.
\end{equation}
And prescribe periodic boundary condition at $x = 0 $ and $x = 1$  this 
results in the solution at time $t = 1$ to be the same as initial condition 
i.e.
\begin{equation}
u(1, x) = u(0, x).
\end{equation} 
%First we evaluate approximation of initial condition. As seen from figure 
%\ref{fig:conv_0adv1D} the approximation corresponds to theoretical orders.
We then compare these to test the convergence, see Figures 
\ref{fig:adv_conv_1D} and \ref{fig:adv_conv_1D_step}. For smooth initial condition 
limiting increases error of the solution due to artificial diffusion, higher order 
methods are capable of counteracting this effect though. For discontinuous initial 
condition limiting significantly improves behavior of the method removing oscillations 
and basically enabling use of high order methods, which suffer from them the most. The 
resulting errors are still significant as limiting introduces prominent 
smoothing. Note that for both $u_{smooth}$ a $u_{step}$ with and without limiting the 
order of the method is impacted by used 3rd order TVD Runge-Kutta time-stepping solver.
\end{example}

\begin{figure}[p!]
	\centering
	\includegraphics[width=1.1\textwidth]{../figs/parametric/advection_1D/advection_1D_smooth_reduced.png}
	\caption{\Cref{ex:adv1D} convergence plots for smooth initial condition 
		$u_{smooth}$}
	\label{fig:adv_conv_1D}
\end{figure}


\begin{figure}[p!]
	\centering
	\includegraphics[width=1.1\textwidth]{../figs/parametric/advection_1D/advection_1D_step_reduced.png}
	\caption{\Cref{ex:adv1D} convergence plots for discontinuous initial 
	condition $u_{step}$}
	\label{fig:adv_conv_1D_step}
\end{figure}

\newpage
\begin{example}[Diffusion 2D]
\label{ex:laplace}
Inspired by \cite[cv. 8.4 (3), p. 150]{Holubova2011}, in $\Omega = \langle 0, 1 
\rangle^2$ 
we will solve Laplace equation \eqref{eq:ex_laplace}.
%\begin{equation}
%\pdiff{^2 u}{x^2} + \pdiff{^2 u}{y^2} = 0
%\end{equation}
%i.e
%\begin{equation}
%\Delta u = 0.
%\end{equation}
We setup boundary condition in such way that the exact solution 
$u_{exact}$ is polynomial
\begin{equation}
u_{exact} = \frac{1}{2}x^2 - \frac{1}{2}y^2 - ax + by + c
\end{equation}
We set boundary conditions to match analytical solution as follows
\begin{equation}
	\begin{aligned}
		&u_x(0, y) = -a, & u_x(a, y) = 0\\
		&u_y(x, 0) = b, & u_y(x, b) = 0.
	\end{aligned}
\end{equation}
In our setting we chose $a=1$, $b=1$, $c=0$. Again different values of 
coefficient $C_w$ in penalty term yield different convergence behavior as demonstrated in 
Figures \ref{fig:conv_laplace} and \ref{fig:orders_lapalce}.  \Cref{fig:orders_lapalce} 
may suggest that high order method to not meet expected convergence rate, they however 
still attain lowest error as illustrated in \Cref{fig:conv_laplace}. This is due to 
polynomial solution which can be approximated very accurately even on coarse mesh and 
refining does not provide much benefit.
\end{example}

\begin{figure}[h!]
	\centering
	\begin{tabular}{p{0.5\textwidth} p{0.5\textwidth}}
		\vspace{0pt} 
		
\includegraphics[width=0.49\textwidth]{../figs/parametric/diffusion_2D/ord_laplace_2_4}
		&
		\vspace{0pt} 
		
\includegraphics[width=0.49\textwidth]{../figs/parametric/diffusion_2D/ord_laplace_2_3}
	\end{tabular}
	\caption{\Cref{ex:laplace} average order for different choice of $C_w$}
	\label{fig:orders_lapalce}
\end{figure}

%\begin{figure}[h!]
%	\centering
%	\begin{subfigure}{.5\textwidth}	
%		\centering	
%		
%\includegraphics[width=\linewidth]{../figs/err-sols/0_1_0_0_0_0_0_0_0_sol-h256o02.0.png}
%		\caption{$C_w = 1$}
%	\end{subfigure}%
%	\begin{subfigure}{.5\textwidth}
%		\centering	
%		
%\includegraphics[width=\linewidth]{../figs/err-sols/2_1_2_0_0_0_0_0_0_sol-h256o02.0.png}
%		\caption{$C_w = 15$}
%	\end{subfigure}
%	\caption{Solutions of \Cref{ex:kucera} for different values of $C_w$ }
%\end{figure}

\begin{figure}[p!]
	\centering
	\includegraphics[height=\textheight]{../figs/parametric/diffusion_2D/laplace}
		
	\caption{\Cref{ex:laplace} Convergence graph for different choice of $C_w$}
	\label{fig:conv_laplace}
\end{figure}
\newpage
\begin{example}[Advection-diffusion 2D]
\label{ex:quart1}
Based on Example 1 from \cite{Antonietti2013},
we will solve the equation \eqref{eq:ex_advdiff} in $\Omega = [0, 1]^2$.
%\begin{equation}
%	\pdiff{u}{x} + \pdiff{u}{y} - D \cdot \left( \pdiff{^2 u}{x^2} + \pdiff{^2
%u}{y^2} \right) = g
%\end{equation}
%i.e
%\begin{equation}
%	\vec{a} \cdot \nabla u - D \Delta u = g
%\end{equation}
%where $\vec{a} = [1, 1]^T$ is advection velocity, $D$ is diffusion coefficient
%and $g$ is a source function.
We set up the boundary conditions and source function $g$ in such a way that
the exact solution $u_{exact}$ is
\begin{equation}
	u_{exact}(x,y) =  -{\left(y^{2} - y\right)} \sin\left(2 \, \pi x\right).
\end{equation}
Solving for $g$ yields
\begin{equation}
	g = \\
	 -2 \, \pi {\left(y^{2} - y\right)} \cos\left(2 \, \pi x\right) - 2 \, {\left(2 \,
	 \pi^{2}
	{\left(y^{2} - y\right)} \sin\left(2 \, \pi x\right) - D\sin\left(2 \, \pi
	x\right)\right)}
	 - {\left(2 \, y - 1\right)} \sin\left(2 \, \pi x\right).
\end{equation}
Matching boundary conditions are
\begin{equation}
u(x) = 0, \quad \nabla u(x) = [-2\pi(y^2 - y)\cos(2 \pi x), -(2 y - 1)\sin(2\pi  x)]^T,
\quad x \in \partial\Omega.
\end{equation}
Different values of the coefficient $C_w$ in the penalty term then yield different
convergence behavior as demonstrated in \Cref{fig:orders_quarteroni1}
and \ref{fig:qconv1}. Both figures illustrate the "gluing" effect of the penalty term
which increases with $C_w$ and counteracts discontinuities between elements
which are the main source of error in this example. In \Cref{fig:sol_quart1} this effect
is clearly visible in numerical solutions. With the growing $C_w$ the convergence
behavior of the method improves, with the exception of the 0th order approximation for which it
has no effect as expected. Antonietti et al. \cite{Antonietti2013} report convergence for
order 1 and 2, these are in accord with ours.

\begin{figure}[h!]
	\centering
	\begin{tabular}{p{0.5\textwidth} p{0.5\textwidth}}
		\vspace{0pt}
		\includegraphics[width=0.4\textwidth]{../figs/parametric/advdiff_2D/ord_quarteroni1_2_4}
		&
		\vspace{0pt}
		\includegraphics[width=0.4\textwidth]{../figs/parametric/advdiff_2D/ord_quarteroni1_2_3}
	\end{tabular}
	\caption{\Cref{ex:quart1}. Average convergence rates for different choice of $C_w$
	for quadrilaterals (left) and triangles (right).}
	\label{fig:orders_quarteroni1}
\end{figure}


\begin{figure}[h!]
	\centering
	\begin{subfigure}{.45\textwidth}
		\centering
		\includegraphics[width=\linewidth]{../figs/sols/quart1-03100-sol-h1024o04.png}
		\caption{$C_w = 1$}
	\end{subfigure}%
	\begin{subfigure}{.45\textwidth}
		\centering
		\includegraphics[width=\linewidth]{../figs/sols/quart1-53100-sol-h1024o04.png}
		\caption{$C_w = 10^5$}
	\end{subfigure}
	\caption{\Cref{ex:quart1}. Solutions on quadrilateral mesh for $D = 1$ and for
	different values of $C_w$.}
	\label{fig:sol_quart1}
\end{figure}

\end{example}
\begin{figure}[p!]
	\centering
	\includegraphics[height=\textheight]{../figs/parametric/advdiff_2D/quarteroni1.png}
	\caption{\Cref{ex:quart1}. Relative errors for different choices of $C_w$ for
	quadrilaterals (left) and triangles (right).}
	\label{fig:qconv1}
\end{figure}

\newpage
\begin{example}[Advection-diffusion 2D]
\label{ex:quart2}
Based on Example 2 \cite{Antonietti2013},
in $\Omega = \langle 0, 1 \rangle^2$ we will again solve equation \eqref{eq:ex_advdiff}
%\begin{equation}
%	\pdiff{u}{x} + \pdiff{u}{y} - D \cdot \left( \pdiff{^2 u}{x^2} + \pdiff{^2 
%	u}{y^2} \right) = g
%\end{equation}
%i.e
%\begin{equation}
%	\vec{a} \cdot \nabla u - D \Delta u = g
%\end{equation}
%where $\vec{a} = [1, 1]^T$ is advection velocity and $D$ is diffusion 
%coefficient and $g$ is a source function. 
We set up boundary condition and source function in such a way that the exact 
solution $u_{exact}$ is
\begin{equation}
	u_{exact} =  -\arctan\left(\frac{4 \, {\left(2 \, x - 1\right)}^{2} + 4 \, {\left(2 
	\, y - 1\right)}^{2} - 
	1}{16 \, \sqrt{\mathit{D}}}\right).
\end{equation}
We omit analytical forms of $g$ and boundary conditions for brevity, they can be found in 
the code.
Different values of coefficient $C_w$ in penalty term yield different 
convergence behavior as demonstrated in Figure \ref{fig:conv_qart2} and 
\Cref{fig:orders_quarteroni2}.
\end{example}

\begin{figure}[h!]
\centering
\begin{tabular}{p{0.5\textwidth} p{0.5\textwidth}}
	\vspace{0pt} 
	\includegraphics[width=0.49\textwidth]{../figs/parametric/advdiff_2D/ord_quarteroni2_2_4}
	&
	\vspace{0pt} 
	\includegraphics[width=0.49\textwidth]{../figs/parametric/advdiff_2D/ord_quarteroni2_2_3}
\end{tabular}
\caption{\Cref{ex:quart2} average order for different choice of $C_w$}
\label{fig:orders_quarteroni2}
\end{figure}


\begin{figure}[p!]
	\centering
	\includegraphics[height=\textheight]{../figs/parametric/advdiff_2D/quarteroni2.png}
	
	\caption{\Cref{ex:quart2} convergence graphs for different choice of $C_w$}
	\label{fig:conv_qart2}
\end{figure}

\newpage
\begin{example}[Advection-diffusion 2D]
\label{ex:quart3}
Based on Example 3 in \cite{Antonietti2013},
in $\Omega = [0, 1]^2$ we will once again solve the equation \eqref{eq:ex_advdiff}
%\begin{equation}
%	\pdiff{u}{x} + \pdiff{u}{y} - D \cdot \left( \pdiff{^2 u}{x^2} + \pdiff{^2
%	u}{y^2} \right) = g
%\end{equation}
%i.e
%\begin{equation}
%	\vec{a} \cdot \nabla u - D \Delta u = g
%\end{equation}
%where $\vec{a} = [1, 1]^T$ is advection velocity and $D$ is diffusion
%coefficient and $g$ is a source function.
We set up the boundary condition and source function in such a way that the exact
solution $u_{exact}$ is
\begin{equation}
	u_{exact} = -xy + x +y + \frac{\exp{\left(-\frac{{\left(x - 1\right)} {\left(y -
	1\right)}}{D}\right)} -
	\exp{\left(-\frac{1}{D}\right)}}{\exp{\left(-\frac{1}{D}\right)}
	- 1}.
\end{equation}
We omit analytical forms of $g$ and boundary conditions for brevity, they can found in
the code. Different values of the coefficient $C_w$ in the penalty term yield different
convergence behavior as demonstrated in Figures \ref{fig:orders_quarteroni3} and 
\ref{fig:conv_qart3}. Antonietti et al. \cite{Antonietti2013} report convergence
for order 1 and 2, these are in accord with ours. \todo

\begin{figure}[h!]
    \centering
    \begin{subfigure}{.5\textwidth}
        \centering
        \includegraphics[width=.8\linewidth]{../figs/sols/quart3-00000-sol-h1024o04}
        \caption{$C_w = 1$}
    \end{subfigure}%
    \begin{subfigure}{.5\textwidth}
        \centering
        \includegraphics[width=.8\linewidth]{../figs/sols/quart3-30000-sol-h1024o04}
        \caption{$C_w = 10^3$}
    \end{subfigure}
    \caption{\Cref{ex:quart3}. Solutions using quadrilateral mesh for $D = 0.001$ and
        for different values of $C_w$ for quadrilaterals (left) and triangles (right).}
    \label{fig:sol_quart3}
\end{figure}


\begin{figure}[h!]
    \centering
    \begin{tabular}{p{0.5\textwidth} p{0.5\textwidth}}
        \vspace{0pt}
        \includegraphics[width=0.5\textwidth]{../figs/parametric/advdiff_2D/ord_quarteroni3_2_4}
        &
        \vspace{0pt}
        \includegraphics[width=0.5\textwidth]{../figs/parametric/advdiff_2D/ord_quarteroni3_2_3}
    \end{tabular}
    \caption{\Cref{ex:quart3}. Average convergence rate for different choices of $C_w$}
    \label{fig:orders_quarteroni3}
\end{figure}
\end{example}


\begin{figure}[p!]
	\centering
	\includegraphics[height=\textheight]{../figs/parametric/advdiff_2D/quarteroni3.png}

	\caption{\Cref{ex:quart3}. Relative errors for different choice of $C_w$ for
	quadrilaterals (left) and triangles (right).}
	\label{fig:conv_qart3}
\end{figure}

\newpage
\begin{example}[Viscous Burgers 1D]
\label{ex:burgers_hest}
Based on \cite[Section 7.1.2, Example 7.5,  p. 255]{Hesthaven2008}.
On $\Omega = [-1, 1]$ we will solve the viscous Burgers’ equation
\eqref{eq:ex_burgers} with the zero source function.
%\begin{equation}
%	\pdiff{u}{t} + \pdiff{}{x}\left(\frac{u^2}{2}\right) = D \pdiff{^2u}{x^2}
%\end{equation}
%where $D$ is diffusion coefficient.
This equation has an exact solution of a traveling wave
\begin{equation}
	u_{exact} =  -\tanh\left(-\frac{2 \, t - 2 \, x - 1}{4 \,D}\right) + 1.
\end{equation}
We set boundary conditions to match the solution
\begin{equation}
	\begin{aligned}
	& u(-1, t) = -\tanh\left(-\frac{2 \, t  + 1}{4 \,D}\right) + 1,
	&  u(-1, t) = -\tanh\left(-\frac{2 \, t - 3}{4 \,D}\right) + 1,\\
	&u_x(-1, t) = \frac{1}{2D}\tanh\left(-\frac{2 \, t + 1}{4 \, D}\right)^{2} -
	\frac{1}{2D},
	&u_x(1, t) = \frac{1}{2D}\tanh\left(-\frac{2 \, t - 3}{4 \, D}\right)^{2} -
	\frac{1}{2D}.
	\end{aligned}
\end{equation}
We will study the solution at time $t = 1$ with $D = 0.001$ and $D = 0.01$. The exact
solution is shown in \Cref{fig:burgers_hesthaven_ext}.
\begin{figure}[h]
	\centering
%	\begin{tabular}{p{0.5\textwidth} p{0.5\textwidth}}
%		\vspace{0pt}
%		\includegraphics[width=0.4\textwidth]{../figs/burgess_hesthaven_exact_t1_e001.png}
%		&
%		\vspace{0pt}
%		\includegraphics[width=0.4\textwidth]{../figs/burgess_hesthaven_exact_t1_e01.png}
%	\end{tabular}
%
	\includegraphics[scale=0.45]{../figs/burgers_hesthaven_exact_t1.png}
	\caption{\Cref{ex:burgers_hest}. Exact solution at $t = 1$.}
	\label{fig:burgers_hesthaven_ext}
\end{figure}
Different values of the coefficient $C_w$ in the penalty term yield different convergence
behavior as demonstrated in \Cref{fig:burgers_conv}. In this case increasing $C_w$ is detrimental to the accuracy of the solution for $D=0.001$ as it
develops into a steep step, see \Cref{fig:burgers_hesthaven_ext}, whose approximation
requires discontinuity in the approximate solution. For $D=0.01$ the diffusion leads to a much
more gradual solution and the penalty term counteracting discontinuity between elements
is beneficial, it also helps to counteract oscillations similarly to the limiter in
\Cref{ex:adv1D}. \todo Figure \ref{fig:burgers_sol} demonstrates where the errors come 
from.

\begin{figure}[h!]
    \centering
    \begin{subfigure}{.5\textwidth}
        \centering
        \includegraphics[width=\linewidth]{../figs/sols/burg1D-0002000100000-sol-h4o4}
        \caption{Limit: False}
    \end{subfigure}%
    \begin{subfigure}{.5\textwidth}
        \centering
        \includegraphics[width=\linewidth]{../figs/sols/burg1D-0002000000000-sol-h4o4}
        \caption{Limit: True}
    \end{subfigure}
    \caption{\Cref{ex:burgers_hest}. 4th order solution for $D=0.001$ with and without
        limiting.}
    \label{fig:burgers_sol}
\end{figure}

\end{example}

%\begin{figure}[h!]
%	\centering
%	\begin{tabular}{p{0.5\textwidth} p{0.5\textwidth}}
%		\vspace{0pt}
%
%\includegraphics[width=0.49\textwidth]{../figs/parametric/burgers_1D/orders_unlimited}
%		&
%		\vspace{0pt}
%
%\includegraphics[width=0.49\textwidth]{../figs/parametric/burgers_1D/orders_limited}
%	\end{tabular}
%	\caption{\Cref{ex:burgers_hest} average convergence rates for different choices of
%	$C_w$}
%	\label{fig:burgess_orders}
%\end{figure}

\begin{figure}[p!]
	\centering
	\includegraphics[width=\textwidth]{../figs/parametric/burgers_1D/convergences}
	\caption{\Cref{ex:burgers_hest}. Relative errors for different choices of $C_w$}
	\label{fig:burgers_conv}
\end{figure}
\clearpage

%\begin{figure}
%	\centering
%	\includegraphics[width=\textwidth]{../figs/err-sols/burgess_hesthaven-err-sol-i20cw1_d001_t2}
%	\includegraphics[width=\textwidth]{../figs/err-sols/burgess_hesthaven-err-sol-i20cw10_d001_t2}
%	\caption{Example 9 Exact solution (gray), numerical solution (orange) and their absolute difference (red) for
%	different orders and $h$. The left $y$ axes correspond to the solutions, the right ones to their difference.}
%	\label{fig:err_sol_burges_hest}
%\end{figure}

\newpage
\begin{example}[Viscous Burgers 2D]
\label{ex:kucera}
Based on example in \cite[Section 1.6]{Kucera},
we will solve equation \eqref{eq:ex_burgers} in $\Omega = \langle 0, 1 \rangle^2$.
%\begin{equation}
%		\pdiff{u}{t} + \frac{1}{2}\left(\pdiff{u^2}{x} + \pdiff{u^2}{y}\right)  - 
%		D \cdot \left( \pdiff{^2 u}{x^2} + \pdiff{^2 u}{y^2} \right) 
%		= g
%\end{equation}
%where $D$ is diffusion coefficient and $g$ is a source function. 
We setup boundary condition and source function in such way that the exact 
solution $u_{exact}$ is
\begin{equation}
	u_{exact} =  \ -{\left(e^{\left(-t\right)} - 1\right)} {\left(\sin\left(5 \,x 
	y\right) + \sin\left(-4 \, 
	x y + 4 \,x + 4 \, y\right)\right)}.
\end{equation}
We omit analytical forms of $g$ and boundary conditions for brevity.
Different values of coefficient $C_w$ in penalty term then yield only slightly different 
convergence behavior as demonstrated in Figure \ref{fig:kucera_conv} and 
\Cref{fig:kucera_orders}. In this case the solution does not feature any sharp steps and 
increase in penalty coefficient leads to increase in accuracy. In \Cref{fig:sol_kucera} 
this effect is clearly visible in sample solution, similarly to \Cref{ex:quart1}
\end{example}

\begin{figure}[h!]
	\centering
	\begin{tabular}{p{0.5\textwidth} p{0.5\textwidth}}
		\vspace{0pt} 
		\includegraphics[width=0.49\textwidth]{../figs/parametric/burgers_2D/orders_2_4}
		&
		\vspace{0pt} 
		\includegraphics[width=0.49\textwidth]{../figs/parametric/burgers_2D/orders_2_3}
	\end{tabular}
	\caption{\Cref{ex:kucera} average orders for different choice of $C_w$ in 
	tensor-product geometry (left) and simplex geometry (right)}
	\label{fig:kucera_orders}
\end{figure}


\begin{figure}[h!]
	\centering
	\begin{subfigure}{.5\textwidth}	
		\centering	
		\includegraphics[width=\linewidth]{../figs/sols/0_1_0_0_0_0_0_0_0_sol-h256o02}
		\caption{$C_w = 1$}
	\end{subfigure}%
	\begin{subfigure}{.5\textwidth}
		\centering	
		\includegraphics[width=\linewidth]{../figs/sols/2_1_2_0_0_0_0_0_0_sol-h256o02}
		\caption{$C_w = 15$}
	\end{subfigure}
	\caption{Solutions of \Cref{ex:kucera} for different values of $C_w$}
	\label{fig:sol_kucera}
\end{figure}

\begin{figure}[p!]
	\includegraphics[width=\textwidth]{../figs/parametric/burgers_2D/convergence_symmetry}
	\caption{\Cref{ex:kucera} convergence plots for different choice of $C_w$}
	\label{fig:kucera_conv}
\end{figure}

\chapter{Conclusion}
\label{ch:conclusion}
% !TeX spellcheck = en_US
%%
%% Text of diploma thesis
%%
%% Tomáš Zítka
%%
\paragraph{Implementation}
In this work we implemented discontinuous Galerkin method into SfePy package. 
SfePy uses term based syntax for building discretizations of equations, we implemented 
several new terms, namely linear advection flux term, general hyperbolic flux term, 
diffusion flux and diffusion penalty term, and general term for computing integral
$$
\int_{T^k} \vec{f}(P_i^k\psi_i)\cdot\nabla\psi_j. 
$$
Along with wide range of terms already present in SfePy this allows users to 
discretize variety of useful equations. To enable solving transient equations we 
implemented two explicit time stepping solvers, forward Euler and TVD Runge-Kutta of 3rd 
order solver. Moreover we implemented moment limiters for 1D and 2D transient 
problems.

\paragraph{Analysis}
To study properties of the method we measured convergence rates for seven 
example 
problems chosen from literature. For some of them we present results to our 
knowledge not 
available in literature. For transient problems the performance of the method 
is hindered 
by used time stepping solvers, nevertheless the method still performs to the 
expectations.
[\todo more on examples when they are finished]

This fulfills all the major goals of this work. There are, however, still many possible 
improvements and opportunities for future work. Although from the time and memory 
requirements perspective the implementation of the method scales well enough with SfePy 
capabilities, there is still room for improvement. Calls of \pysauce{numpy.einsum} could 
use optimized tensor contraction path, which could be retained between individual term 
evaluation (i.e. between time steps). The implementation of limiters is rather ad-hoc and 
refactoring it to bring it in line with design of other SfePy elements would help making 
code more readable and their usage simpler and more versatile. 

One important feature available in SfePy missing from this DG FEM 
implementation is ability to solve systems of PDEs. This lack could inspire future work 
as it would requires substantial modification of flux terms as well as implementation of  
strategies for evolving systems of interest like Euler or Navier-Stokes 
equations \cite{Hesthaven2008}. Further besides implemented Lax-Friedrichs flux there is 
variety of other fluxes for examples Godunov flux \cite{DiPietro2012} or fluxes designed 
specifically for solving Euler equations \cite[Section 3.3]{Kucera} which could be added 
to broaden tools available in SfePy. Thanks to versatility of problem specification this 
would also unlock large potential for further study of the method behavior. 



\section*{Acknowledgment}
 DPP v rámci řešení projektu GACR 16-03823S

%\section{Doc test}
%\begin{equation}
%	C = \max_{p \in [1, 2]}\left\lvert n_x \frac{\partial p}{\partial a_1} +
%	n_y\frac{\partial p}{\partial a_2} \right\rvert =
%	\max_{p \in [1, 2]} \left\lvert  \ul{n} \cdot \nabla p \cdot \ul{a} \right\rvert
%\end{equation}

% TODO resolve citation format
%\nocite{*}

\addcontentsline{toc}{chapter}{Bibliography}
\bibliographystyle{plain}
\bibliography{dg_fem_literature}


%\glsaddall % list all
%\renewcommand*{\acronymname}{Seznam použitých zkratek}
%\addcontentsline{toc}{chapter}{Seznam použitých zkratek}
%\printnoidxglossaries % print acronym without having to call indexer(?)
\addtocontents{toc}{\protect\enlargethispage{2\baselineskip}}
\begin{appendix}
	

	
\end{appendix}


\end{document}